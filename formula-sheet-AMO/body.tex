% Matteo Veneziano -- Atomic and Molecular Physics and Optics Formula Sheet
\graysec{Light propagation, Optics}
\begin{squishlist}
    \item Helmholz equation $\Delta \vec{E} - \frac{n^2}{c^2} \partial_t^2 \vec{E} = 0$
    \item Monochromatic waves: $\vec{E}({\vec{r}, t}) = \Re[\vec{E}(\vec{r}e^{-i \omega t})]$
    \item \textbf{Angular spectrum representation} (2D Fourier transform in plane at position $z$): $\tilde{\vec{E}}(k_x, k_y, z) = \tilde{\vec{E}}(k_x, k_y, 0) e^{\pm i k_z z}$ \squishsep if $\Im(k_z) \geq 0$ evanescent waves for $z \rightarrow \infty$
\end{squishlist}
\graypar{Paraxial approximation}
\begin{squishlist}
    \item Smooth, slow spatial variations $k_x, k_y \ll k$
    \item $k_z \approx k - \tfrac{k_x^2 + k_y^2}{2k} \quad \Rightarrow \quad e^{ik_z z} = e^{ikz}e^{-i \tfrac{k_x^2 + k_y^2}{2k}z}$ \textbf{Gaussian beams}
    \item $\vec{E}(x,y,0) = \vec{E_0} \exp\left({-\tfrac{x^2 + y^2}{w_0^2}}\right) \quad \tilde{\vec{E}}(k_x,k_y,0) = \vec{E_0} \frac{w_0^2}{4\pi} \exp \left(- \frac{(k_x^2 + k_y^2)w_0^2}{4}\right)$
    \item $\vec{E}(x,y,z) = \vec{E_0} \frac{e^ikz}{1 + \tfrac{2iz}{kw_0^2}} \exp\left({-\tfrac{x^2 + y^2}{w_0^2}\left(\tfrac{1}{1 + \tfrac{2iz}{kw_0^2}}\right)}\right)$
    \item $\vec{E}(x,y,z) = \vec{E_0} \frac{w_0}{w(z)}\exp\left(-\frac{x^2 + y^2}{w(z)^2}\right) \exp \left(i \left(kz - \eta(z) + \frac{k(x^2 + y^2)}{2 R(z)}\right)\right)$
    \item $w_0$ (waist) is a measure of the width of the beam at its narrowest point.
    \item $\eta(z) = \arctan (z/z_R)$ Gouy phase
    \item $w(z) = w_0 \sqrt{1 + \left(\frac{z}{z_R}\right)^2}$ (hyperbolic relation), where $z_R = \frac{\pi n w_0^2}{\lambda}$ is the Rayleigh range.
    \item At a distance $z_R$ from the waist, the width $w$ of the beam is $\sqrt{2}$ larger than at the focus where $w=w_0$.
    \item The wavefronts have zero curvature (radius $R = \infty$) at the waist. Curvature increases away from the waist upt to $z_R$ then decreases approaching zero as $z \rightarrow \infty$
    $\frac{1}{R(z)} = \frac{z}{z^2 + z^2_R}$
    \item Complex beam parameter $q(z) = z + i z_R$ (when propagating in free space) \\ $\frac{1}{q(z)} = \frac{1}{R(z)} - i \frac{\lambda}{n \pi w^2(z)}$
\end{squishlist}

\graypar{Fresnel}
\begin{squishlist}
    \item \textbf{Fresnel diffraction} $\vec{E}(x,y,z) = \int \vec{E}(x',y',0) h(x-x', y-y', z)$ \\
    $h = \frac{2\pi k}{iz} e^{ikz} \exp \left(i \frac{k(u^2 + v^2)}{2z}\right)$
\end{squishlist}
\graypar{Paraxial optics}
\begin{squishlist}
    \item Thin lens: no diffraction inside $ \Rightarrow \vec{E}_{\text{out}}(x,y) = t(x,y) \vec{E}_{\text{in}}(x,y)$
    \item Thickness function $\Delta(x,y) = \Delta_0 - \frac{x^2 + y^2}{2} \left( \frac{1}{R_1} - \frac{1}{R_2} \right)$
    \item Define $\frac{1}{f} = (n-1) \left( \frac{1}{R_1} - \frac{1}{R_2} \right)$ then $t(x,y) = \exp(-i \frac{k}{2f}(x^2 + y^2)) P(x,y)$
    \item In $f$ we find Fourier transform of what we had as input in position $d$
    \item For a finite aperture (ex. $P(x,y)=1 \text{ for } x,y \in [-1,1]$) best resolution $w \sim \frac{\lambda f}{L}$
\end{squishlist}

\graysec{Light-matter interaction}
\graypar{Out-of-equilibrium atom}
\begin{squishlist}
    \item $\kpsi = \frac{1}{N}(\ket{1s} + \epsilon \ket{2p}) \; \Rightarrow \; \ket{\psi(t)} = \ket{1s} + \epsilon e^{-i\omega_{2p}t}\ket{2p} \; \Rightarrow \; \rho(t) = -e |\psi|^2 = \rho_0(\vec{r}) + \epsilon \delta \rho(\vec{r}, t)$
    \item $\delta(\vec{r},t) = -e R_{1s}(\vr) R_{2p} \left\{
    \begin{aligned}
        &\frac{2z}{r}\cos(\omega t) \quad m=0 \\
        &\frac{2}{r} (x \cos(\omega t) \pm y\sin(\omega t)) \quad m=\pm 1
    \end{aligned}
    \right.$
    \item Radiation emitted (frequency $\omega$), energy leaving atom. $E_{\text{initial}} \sim \braket{\psi|H|\psi} = \epsilon^2 \hbar \omega_{2p}$
    \item Decay of atom in $\ket{2p}$ at rate $\Gamma$: \textbf{spontaneous emission}
\end{squishlist}

\graypar{Semi-classical theory}
\begin{squishlist}
    \item $\hat{H} = \frac{(\hat{p} + e\vec{A})^2}{2m} - e\phi \quad \overset{\text{long wavelength}}{\Longrightarrow} \quad \hat{H} = \hat{H}_{\text{atom}} - {\hat{\vec{d}}\cdot \vec{E}(\vr=0,t)} = \hat{H}_{\text{atom}} - \hat{H}_{\text{lm}}(t)$
    \item Perturb. th. $\Rightarrow P_{g\rightarrow e} = P_{e \rightarrow g} = \frac{T^2}{4 \pi \hbar^2}\left| \braket{g|\hat{H}_{lm}|e}\right|^2 \left(\frac{\sin (\delta T / 2)}{\delta T / 2}\right)^2 \quad \delta = \omega - \frac{|E_g - E_e}{\hbar}$  \\
    respectively absorption and stimulated emission.
\end{squishlist}

\graysec{Molecular spectroscopy}
\graypar{Rotational spectroscopy}
\begin{squishlist}
    \item Assume free rigid linear rotor
    \item $\hat{H} Y_J^M(\theta, \phi) = E_J Y_J^M(\theta, \phi)$ with $E_J = \frac{\hbar^2}{2I} J (J+1) = B J(J+1)$. $B$ rotational constant
    \item Equilibrium occupations $P(J) = \frac{1}{q(T)} (2J+1) \exp(-E_J / k_B T)$ \\
    with $q(T) = \sum_{J=0}^{\infty} (2J+1) \exp(-E_J / k_B T)$ the partition function. \\ 
    \textbf{Note when drawing:} shouldn't be $0$ at $J=0$.

    \item $E_f = E_i \pm \hbar \omega$ ($+$: absorption, $-$: emission)
    \item State $\ket{\varepsilon} \ket{J, m_J}$ (electron vibrational part, rotational part)
    \item \textbf{Gross selection rule}: purely rotational spectra are observed only for molecules with $\vec{d_e} = \braket{\varepsilon|\vec{d}|\varepsilon} \neq 0$. AB molecules have $\vec{d_e} \neq 0$. A$_2$ have $\vec{d_e} = 0$.
    \item \textbf{Particular selection rules:} $\Delta J = \pm 1$, $\Delta m_J = 0, \pm 1$ ($J=J'$ not observable)
    \item \textbf{Centrifugal distorsion:} $\Delta E(J) = 2B (J+1) - 4D(J+1)^3 \longrightarrow$ unequal spacing between spatial lines
\end{squishlist}

\graypar{Vibrational spectroscopy}
\begin{squishlist}
    \item $\psi(q) = N_{\nu} H_{\nu}(q) \exp(-q^2/2)$ with $q = \left(\frac{\mu k}{\hbar^2}\right)^{1/4}x$ \squishsep $E_{\nu} = \hbar \omega (\nu + \frac{1}{2})$
    \item \textbf{Gross selection rule:} $\left.\dder{\vec{d}}{q}\right|_0 \neq 0$
    \item \textbf{Specific selection rule:} $\Delta \nu = \pm 1$ (but $\Delta \nu = \pm 1$ transitions are always accomp. by $\Delta J = \pm 1$ trans.)
\end{squishlist}

\graypar{Rovibrational spectroscopy}
\begin{squishlist}
    \item $E_{\nu, J} = (\nu + \frac{1}{2}) \hbar \omega + B_{\nu} J(J+1)$. Higher contrib. from $\nu$
    \item $\Delta \nu = \pm 1, \Delta J = \pm 1$.
    \item R-Branch: $\nu \rightarrow \nu + 1,\, J \rightarrow J+1$ \quad P-Branch:  $\nu \rightarrow \nu + 1, \,  J \rightarrow J-1$
    \item Absorption \quad R: $\Delta E = \hbar \omega + 2B(J+1)$, P: $\Delta E = \hbar \omega - 2BJ$
    \item \textbf{Anharmonic Morse potential} $V(R) = D_e (1 - e^{-ax}), \, a = \sqrt{\frac{k}{2D_e}}$ \\
    $E_{\nu} = (\nu+\frac{1}{2}) \hbar \omega - (\nu - \frac{1}{2})^2 \hbar \omega \chi_e$, with $\chi_e$ anharmonicity constant \\
    Compresses R-branch and stretches P-branch. Allows for $\Delta \nu = \pm 2, \pm 3$
\end{squishlist}

\graypar{Raman spectroscopy}