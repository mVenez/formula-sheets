% Matteo Veneziano -- Atomic and Molecular Physics and Optics Formula Sheet

\graysec{Atomic physics}
\graypar{Hydrogenic atom spectrum}
\begin{squishlist}
    \item $H = - \frac{\hbar^2}{2m}\nabla^2_r - \frac{Z e^2}{4 \pi \varepsilon r} $ \qquad 
    $\psi_{n'lm} = R_{n'l}(r) Y_l^m(\theta, \phi)$ \\
    $ - \frac{\hbar^2}{2m_r}u_{n'l}'' + \left[-\frac{1}{4\pi \varepsilon_0}\frac{Ze^2}{r} + \frac{\hbar^2}{2m_r}\frac{l(l+1)}{r^2}\right] u_{n'l} = \epsilon_{n'l}u_{n'l}$ \quad $u_{n'l} = r R_{n'l} \quad m_r = \frac{m_nm_{\elec}}{m_n + m_{\elec}}$ \\
    if $\rho = \frac{r}{a_0^*}$, $\tilde{u}_{n'l}$ dimensionless solut.\ : $\tilde{u}_{n'l}'' + 2(\frac{z}{\rho} - \frac{l(l+1)}{\rho^2})\tilde{u}_{n'l} = \frac{m_r a_0^*2}{\hbar^2} \varepsilon_{n'l} \tilde{u}_{n'l}$
    \item $ a = \frac{a_0}{Z} = \frac{1}{Z} \frac{4\pi \varepsilon_0 \hbar^2}{m_e e^2}$ \quad $a_0^* = \frac{m_e}{m_r}a_0$
    \squishsep $\braket{r}_{nl} = \frac{a}{2Z} [3n^2 - l(l+1)]$
    \item $\frac{1}{Z^2}H_{\text{Hydrogenic}}\left(\rho:=\frac{r}{Z}\right) = H_{\text{Hydrogen}} \Longrightarrow$ lengths scale as $Z^{-1}$
    \item Ground state $\phi_{1s} = \left(\frac{1}{\pi a_0^3}\right)^{1/2} e^{-|\vr|/a_0} $
    \squishsep $s$-orbitals have $\left|\psi_{nl}(0)\right|^2 \neq 0$
    \item \textbf{Bohr:} $E_n = - \frac{z^2}{n^2} R_y = - \frac{z^2}{n^2}h c R_{\infty}$ \quad $n = n' + l + 1$ \quad $R_{\infty} = \frac{m_e e^4}{8 \varepsilon_0^2 h^3 c}$, $R_y = \frac{\alpha^2 m_e c^2 }{2} = 13.6$ eV\\
     $\frac{1}{\lambda_n} = - R_{\infty} \frac{m_r}{m_e} \left[ \frac{1}{n^2} - \frac{1}{n_0^2}\right]$ when trans.between states

    \item \textbf{Degeneracies:} (1) Different $m$ are degenerate. (2) $n'+l =$cst degenerate (accidental, only for $V(r) = -\frac{1}{r}$) $\Longrightarrow$ $n$ labels electron shells
    

    \item \textbf{Selection rules} (electric dipole transitions)
        \begin{enumerate}
            \item The only allowed transitions involv a change in parity
            \item $\Delta l = 0, \pm1 \overset{(1)}{\Longrightarrow} \Delta l = \pm 1$ \quad 3. $\Delta m = 0, \pm 1$ \quad 4. $\Delta j = 0,\pm 1$ (excl.\ $j=j'=0$)
        \end{enumerate}
    
        \item \textbf{Orbital magnetic moment:} the orb. ang. moment. of $e^-$ causes \\
        $m_z = \frac{-e}{2m_e}l_z \Longrightarrow \vec{m} = \gamma_e \vec{l}$ \quad with $\gamma_e = -\frac{e}{2m_e}$\\
        $m_z = \gamma_e m_l \hbar = -\mu_B m_l$ \quad with $m_l = l, l-1, \ldots, -l \quad \mu_B = \frac{e\hbar}{2m_e}$

        \item \textbf{Spin magnetic moment:} \\ $\vec{m} = g_e \gamma_e \vec{s} \Longrightarrow m_z = -g_e \mu_B m_s$ \quad with $m_s=\pm1/2$ and $g_e = 2.0023$ the $g$-factor \\
        In general $\hat{m}_J = - g_J \frac{\mu_B}{\hbar} \hat{J}$
    \end{squishlist} 
    
    \squishline

\textbf{Relativistic effects}
\begin{squishlist}
    \item \textbf{Spin-orbit coupling:} a particle of mass $m_e$ and charge $-e$ moving at $\vec{v}$ in $\vec{E}$ experiences magnetic field $\vec{B} = \frac{1}{c^2}\vec{E}\times \vec{v}$. There is energy of interaction between $\vec{B}$ and the magnetic moment of particle: $H_{so} = \xi(r) \vec{l}\cdot \vec{s}$ \qquad $\Delta_E = \mu_0 B \propto Z^4$\\
    Causes \textbf{Fine structure}: $\left\{\begin{aligned}
        \text{parall.\ spin and orb.\ ang.\ mom.\ } &\Rightarrow \text{High-energy arrangem.\ } \\
        \text{antiparall.\ spin and orb.\ ang.\ mom.\ } &\Rightarrow \text{Low-energy arrangem.\ }
    \end{aligned} \right.$
    The energy of level $j = l+1/2$ lies above $j = l-1/2$
    \item Other relativistic effects (e.g.\ Darwin term for $s$ wavefunctions)
    \item $\Delta E_{1,nl} = - E_n \frac{\alpha^2}{n^2}\left[\frac{3}{4}-\frac{n}{l+1/2}\right]$ \quad $\Delta E_{2,nl} = -E_n \frac{\alpha^2}{2nl(l+1)(l+1/2)}$ times $l$ or $-l-1$ $(j=l\pm\frac{1}{2})$
    \item $E_{nj} = E_n \left[1 + \frac{\alpha^2}{n^2}\left(\frac{n}{j+1/2}-\frac{3}{4}\right)\right]$
    \item Breaks degeneracy of $n$-states, but degeneracy on $m_j$ and accidental degeneracy on states of same $j$ and $n$ (e.g. $2p_{1/2}$ - $2s_{1/2}$)

    \item Notation for $e^-$ orbital/level: $n^{2S+1}L_J$ (e.g. $2^2 p_{1/2}$)
\end{squishlist}

\squishline

\textbf{Nuclear effects}
\begin{squishlist}
    \item Finite size and mass (charge rad.\ is not 0) $\Rightarrow$
    Isotope shift $\Delta E = \frac{Ze^2}{4 \pi \varepsilon_0 10}R^2 |R_{nl}(0)|^2$ \\ $R_{nl}(0) \neq 0$ only for $l=0$ \quad $|R_{nl}(0)|^2 = 4\pi |\psi_{nlm}(0)|^2$ \quad Does not lift $m$ degen.\
    \item \textbf{Hyperfine structure}: the nucleus has finite angular momentum $\vec{I}$ thus magnetic moment $\vec{\mu_N} = g_I \mu_N \frac{\vec{I}}{\hbar}$ with $\mu_N$ the nucleon magneton.
    Interaction between magnetic moments of nucleus and of $e^-$:
    $H_{HFS} \sim \vec{\mu_N}\cdot \vec{\mu_{e^-}} \propto Z^3$ \\
    $H_{HFS} = a_{HFS} \frac{h}{\hbar^2} \; \hat{J}_{\text{elec}}\cdot \hat{I}_{\text{nucl}}$
    \item \textbf{Quadruple moment of nucleus} (only states $l \neq 0$, $I \geq 1$). Charge distrib.\ is no longer spheric.\ symmetr.\. Depends on $m$ thus lifts fine structure degeneracy.
\end{squishlist}

\graypar{Atoms in external fields}
\begin{squishlist}
    \item \textbf{Zeeman effect:} the magnetic moments of the electrons interact with externally applied magnetic fields
    \begin{itemize}
        \item \textbf{Normal} (zero spin angular momentum) \\
        $H^{(1)} = \frac{\mu_B}{\hbar} \hat{L}\cdot \vec{B_0} = - \hat{m}_L B = -\gamma_e l_z B$ (plus diamag.\ term, but small for small $m$) \\
        $\Delta E^{(1)} = \braket{nlm| \frac{\mu_B}{\hbar} L_z B_z |nlm} = \mu_B B_0 m_L$
        \item \textbf{Anomalous} (non-zero spin angular momentum, more common) \\
        $H^{(1)} = \frac{\mu_B}{\hbar}(\hat{L} + 2 \hat{S}) \cdot \vec{B_0}= -\gamma_e (\vec{L} + 2\vec{S}) \cdot \vec{B}$ \qquad
        $\Delta E^{(1)} = \mu_B g_J B_0 m_j$
    \end{itemize}

    \item \textbf{Stark effect:} interaction with electric field of strength $E$ in $z$-direction \\
    $H^{(1)} = - \mu_z E = e z E$ where $\mu_z = -e z$ is $z$-compon./ of electric dipole moment operator. \\
    $\Delta E^{(1)} \propto \int z |R_{nl}(r)|^2 \ud r^3 = 0$ by parity $\Longrightarrow$ must use \textbf{2nd order} perturbation \\
    \textbf{Exception:} The correction to energy is linear in the case of a degeneracy between the two wavefunctions that the perturbation mixes, e.g.\ $2s_{1/2}, 2p_{1/2}$
\end{squishlist}

\graypar{Rydberg atoms (Hydrogen-like atom excited to a very high $n$)}

\begin{squishlist}
    \item $\psi_{n,l,j,m_j}(r,\theta, \phi) = R_{n,l}(r) A_{l,j,m_j}(\theta,\phi)$ \quad $E_{n,l,j} = - \frac{E_I}{(n - \delta_{nlj})^2}$
    \item $\braket{n,l,j,m_j|\vec{r}|n',l',j',m_j'} = \vec{A}(ljm_j, l'j'm_j')R(nl,n'l')$ \, $\braket{||\vec{r}||} = R(nl,nl)$ \, $R(nl,n'l') \propto nn'$
\end{squishlist}

\graypar{Helium atom}
\begin{squishlist}
    \item $\hat{H} = - \frac{\hbar^2}{2m_e}(\nabla^2_1 + \nabla^2_2) - \frac{2e^2}{4\pi \varepsilon_0 r_1} - \frac{2e^2}{4\pi \varepsilon_0 r_2} + \frac{e^2}{4\pi \varepsilon_0 r_{12}} = \hat{H}_{01} + \hat{H}_{02} + \hat{H}_c$
    \item \textbf{Non-interacting spectrum} ($H_c = 0$): $\ham$ is sum of two indep.\ terms, thus wavefunc.\ of the two $e^-$ is the product of the two hydrogenic wavefunctions \\
    $ \ham = \hat{H}_{01} + \hat{H}_{02} \Longrightarrow \psi(r_1,r_2) = \phi_{n_1 l_1 m_1}(r_1) \; \phi_{n_2 l_2 m_2}(r_2)$ \qquad $E = -z^2 R_y \left( \frac{1}{n_1^2}+ \frac{1}{n_2^2} \right)$ \\
    Ground state $E_0 = -8 R_y = -108.8$ eV (exp.\ $-2\times 2.904$). First ionisation treshold $4$ Ry. \\
    Degeneracy for different $l$.

    \item \textbf{Electron-electron interactions} (repulsion term):
    \begin{squishitemize}
        \item First order correction to energy $\Rightarrow$ Coulomb integral \\
        $ \displaystyle \delta E = J = j_0 \int \left|\psi_{n_1 l_1 m_1}(\vec{r_1})\right|^2 \left(\frac{1}{r_{12}}\right) |\psi_{n_2 l_2 m_2}(\vec{r_2})|^2  \ud \tau_1 \ud \tau_2 = \frac{e^2}{4 \pi \varepsilon_0 a_0} \frac{5}{8}Z$ \\
        $E_{1s^2} = 2 R_y [-z^2 + 5/8 z]$ \qquad $\Longrightarrow E = -2\times 2.75$ Ry for $z=2$
        \item We can also use \textbf{variational method} to find an effective charge $z_e$ which minimises the energy. Guess function $\psi(\vr_1 \vr_2) = \frac{z_e^3}{\pi}\exp\left(-\frac{z_e}{a_0}(\vr_1 + \vr_2) \right) = \phi_{1s,Z_e}(\vr_1)\phi_{1s,Z_e}(\vr_2)$ \\
        We find $E_{\min} = -2 R_y (z - 5/16)^2$ \qquad $\Longrightarrow E = -2\times 2.848$ Ry for $z=2$ 
    \end{squishitemize}

    \item \textbf{Excited states:} when two $e^-$ occupy different orbitals the wavefuncs.\ are either
    $\psi_{n_1l_1m_1}(\vec{r}_1) \psi_{n_2l_2m_2}(\vec{r}_2)$ or $\psi_{n_1l_1m_1}(\vec{r}_2) \psi_{n_2l_2m_2}(\vec{r}_1)$. \\
    $\delta E_{a/s} = \braket{\phi_{a,s}|h_c|\phi_{a,s}}$ with
    $\ket{\phi_{a,s}} = \frac{1}{\sqrt{2}} \left( \ket{1s}\ket{nlm} \mp \ket{n l m}\ket{1s}\right)$ \\
    $\Rightarrow \delta E_a = J - K \qquad \delta E_s = J + K$ \qquad $J$: direct term, $K$: exchange term (both $> 0$) \\
    $J_{nl} = \braket{1s,nlm|\hat{H}_c|1s, nlm} \propto \int \ud^3 r_1 \ud^3 r_2 \left|\phi_{1s}(\vr_1)\right|^2 \frac{1}{|\vr_1 - \vr_2} \left|\phi_{nlm}(\vr_2)\right|^2$ \\
    $K_{nl} = \braket{1s,nlm|\hat{H}_c|nlm,1s} \propto \int \phi_{1s}^*(\vr_1) \phi_{nlm}^*(\vr_2) \frac{1}{|\vr_1 - \vr_2}\phi_{1s}^*(\vr_2) \phi_{nlm}^*(\vr_1) $ \\
    $\Longrightarrow$ antisym has lower energy: $S=1$ is favored

    \item Notation: $n^{2S_{\text{tot}}+1} L_{\text{tot}\; J_{\text{tot}}}$
\end{squishlist}

\graypar{Many-electron atoms}
\begin{squishlist}
    \item $\ham = \sum_{i=1}^{Z}\left( - \frac{\hbar^2}{2m}\Delta_i - \frac{Ze^2}{4\pi \varepsilon_0 r_i}\right) + \sum_{i < j} \frac{e^2}{4\pi \varepsilon_0 |\vr_i - \vr_j}$ 
    acts on $\mathcal{H} = \otimes_i \mathcal{H}_i,\, \mathcal{H}_i =  \mathcal{H}_{i, orb} \otimes \mathcal{H}_{i, spin}$
    \item \textbf{Central field approximation}: every electron sees an identical potential $V(r)$ that is only a function of its distance from the nucleus.
    
    \item The inter-electronic contribution is replaced by a screening of the nucleus  \\ $\Longrightarrow Z_{\mathrm{eff}}(r)$ \quad $Z_{\mathrm{eff}}(r) \overset{r_i\rightarrow 0}{\rightarrow} Z$ \quad $Z_{\mathrm{eff}}(r) \overset{r_i\rightarrow \infty}{\rightarrow} 1$\quad $V(r) \sim \frac{Z_{\mathrm{eff}}(r)}{r}$
    \item E.g.\ Thomas-Fermi atoms: many-electron atom with screening function $F(r)$ \\
    $V(r) = -\frac{1}{4 \pi \varepsilon_0} \frac{Z e^2}{r} F(r)$ locally constant \qquad
    $r \propto Z^{-1/3}$
    \item Symetrisation $\rightarrow$ Slater determinants: write wavefuncs of many \elec as product of wavefuncs of \elec, all antisymetr.
    \item The best atomic orbitals are found by numerical solution of the \textbf{Hartree-Fock} eqs.:
    $\left[\hat{h}_k + \hat{J}_k + \hat{K}k\right]\psi_K = \varepsilon_k \psi_k$ \quad
    Schrödinger + Coulomb repulsion + purely quantum contribution.
    The wavefuncs.\ are expressed as Slater determinants.

    \item \textbf{Aufbau principle:} fill the orbitals with increasing values of $n+l$
    \item \textbf{Hund's rules:} (1) Maximise total spin (2) Maximise total $L$ (3) Minimise/Maximise total $J$ if shell is less/more than half filled
\end{squishlist}

\graysec{Molecular physics}
\textbf{Born-Oppenheimer approximation}: nuclei are heavier, thus slower and electrons can respond almost instant.\ to displacement of the nuclei. We treat nuclei as fixed in position ("parametrically").

\squishline

\textbf{Energy scales:} (i) Electrons: $E_e = \frac{\hbar^2}{m_e a^2}\sim 1-10$ eV (UV, vis, IR)
(ii) Vibrations of nuclei $E_V = \sqrt{\frac{m_e}{M}}E_e \sim 0.1$ eV (IR) (iii) Rotations $E_r \approxeq \frac{\hbar^2}{Ma^2} \sim 1$ meV (far IR, MW)
\textbf{Time scales:} $\tau = \hbar/E$ (i) $\tau_e \sim 10^{-16}$ s (ii) $\tau_V \sim 10^{-14}$ s (iii) $\tau_r \sim 10^{-12}$ s
\columnbreak

\textbf{H$_2^+$ molecule}: $\ham_e = - \frac{\hbar^2}{2m_e} \nabla^2 - \frac{j_o}{|\vr - \vR_A|} - \frac{j_o}{|\vr - \vR_B|} + \frac{j_0}{R}$ \quad $R$ the dist.\ betw.\ nuclei 
\begin{squishlist}
    \item \textbf{Variat.\ method} with \textbf{trial wavefunction} $\psi_i(\vr) = a_i \phi_A(\vR) + b_i \phi_B(\vR)$ \\ with $\phi_A(\vr)=\phi_{1s}(\vr - \vR_A) \quad \phi_B(\vr)=\phi_{1s}(\vr - \vR_B)$ 
    \item \textbf{Eigenfuncs:} $\psi_{\pm}(\vr) = N_{\pm} (\phi_A(\vr)\pm \phi_B(\vr))$ \quad $N_{\pm} = 1/\sqrt{2(1+S(R))} $
     \\
    $\psi_+$ is bonding, $\psi_-$ is antibonding. An antibonding orbital is more antibonding than a bonding orb.\ is bonding.
    \item \textbf{Eigenenergs:} $\epsilon_{\pm} = \frac{\epsilon_{AA} \pm \epsilon_{AB}}{1\pm S}$ \quad  $S(R) = \int \ud r^3 \phi_A(\vr) \phi_B(\vr) = \left(1 + \frac{R}{a_0} + \frac{R^2}{3a_0^2}\right) e^{-R/a_0}$\\
    $\epsilon_{AA} = \epsilon_{BB} = \int \ud r^3 \phi_{A}(\vr) H \phi_{A}(\vr) = \epsilon_H + \frac{e^2}{4\pi\varepsilon_0 R}\left(1 + \frac{R}{a_0}e^{-2R/a_0}\right)$ \\
    $\epsilon_{AB} = \int dr^3 \phi_A(\vec{r}) H \phi_B(\vec{r}) = \left(\epsilon_H +  \frac{e^2}{4 \pi \epsilon_0R}\right) S - \frac{e^2}{4 \pi \epsilon_0a_0} \left( 1 + \frac{R}{a_0} \right) e^{-R/a_0}$
    \item Then use variational parameter $a$ instead of $a_0$
\end{squishlist}

\squishline

\begin{squishlist}
    \item \textbf{H$_2$ molecule:} \\
    $H = -\frac{\hbar^2}{2m}\left(\nabla_1^2 + \nabla_2^2\right) + \frac{e^2}{|\vr_1 - \vr_2|} + \frac{e^2}{R} - \left(\frac{e^2}{|\vr_1 - \vR_A|} + \frac{e^2}{|\vr_1 - \vR_B|} + \frac{e^2}{|\vr_2 - \vR_A|} + \frac{e^2}{|\vr_2 - \vR_B|}\right)$
    \item Molecular Orbital approach: $\psi(\vr_1,\vr_2) = \psi_+(\vr_1)\psi_+(\vr_2)$
    
    \item Heitler-London (Valence bond) theory: neglect terms $\phi_{A/B}(\vr_1)\phi_{A/B}(\vr_2)$ in $\psi(\vr_1, \vr_2)$ \\
    $\psi_S(1,2) = \frac{1}{\sqrt{2(1+s^2)}} \left[\phi_A(1)\phi_B(2) + \phi_B(1)\phi_A(2)\right] \chi_S$ \\
    $\psi_T(1,2) = \frac{1}{\sqrt{2(1-s^2)}} \left[\phi_A(1)\phi_B(2) - \phi_B(1)\phi_A(2)\right] \chi_T$
    \item $\epsilon_{S/T} = \frac{\epsilon_{ABAB} \pm \epsilon_{ABBA}}{1 \pm S^2} \quad \epsilon_{ABAB} = 2 \epsilon_H + \epsilon_C \quad \epsilon_{ABBA} = S^2 2 \epsilon_H + \epsilon_{X}$ \squishsep $\epsilon_S < \epsilon_T$
    \item $\epsilon_C(R) = \int d^3r \int d^3r' \, \phi_A^2 (\vec{r}) \phi_B^2 (\vec{r}') \left[ \frac{j}{|\vec{r} - \vec{r}'|}  - \frac{j}{\epsilon_0|\vec{r}' - \vec{R}_A|} - \frac{j}{|\vec{r} - \vec{R}_B|} \right]  + \frac{j}{|\vec{R}_A - \vec{R}_B|}\\
    \epsilon_{X}(R) = \int d^3r \int d^3r' \, \phi_A (\vec{r}) \phi_B (\vec{r}) \phi_A (\vec{r}') \phi_B (\vec{r}')  \left[ \frac{j}{|\vec{r} - \vec{r}'|}  - \frac{j}{|\vec{r}' - \vec{R}_A|}  - \frac{j}{|\vec{r} - \vec{R}_B|} \right]$
    \item Possible improvements by introducing more degrees of freedom e.g. $a$ variat.\ param., MO-LCAO: $\psi = \sum_{R,n,l,m}c_{R,n,l,m} \phi_{n,l,m}(\vr - \vR_i)$, configuration interaction
\end{squishlist}

\squishline

\begin{squishlist}
    \item \textbf{Other diatomic molecules:} Rayleigh-Ritz method (need to solve secular equations)\\
    $\psi = \sum_i c_i \phi_i \rightarrow \epsilon = \frac{\int \psi^* H \psi \ud \vr}{\int \psi^* \psi \ud \vr} = \frac{\sum_{i,j}c_ic_j \int \phi_i^* H \phi_j \ud \vr}{\sum_{i,j}\int \phi_i^* \phi_j \ud \vr} = \frac{\sum_{i,j}c_i c_j H_{ij}}{\sum_{i,j}c_i c_j S_{ij}} \quad S_{AA} = S_{BB} = 1$\\
    Variat.\ meth.\: find min by $\D{c_k}{\epsilon} = 0$ \\
    Leads to \textbf{secular equations} $\sum_i c_i (H_{ik - \epsilon S_{ik}}) = 0 \Longleftrightarrow \det(H - \epsilon S) = 0$\\
    $\det \begin{pmatrix}
        \alpha_A - E & \beta - ES \\
        \beta - ES   & \alpha_B - E
    \end{pmatrix} = 0; \quad \alpha_A = \braket{A|H|A};\; \alpha_B = \braket{B|H|B};\; \beta = \braket{A|B};\; S = \braket{A|B}$ \\
    \textbf{A$_2$ molec.:} $E_{\pm} = \frac{\alpha \pm \beta}{1\pm S}$ \qquad
    \textbf{AB molec.\ ($S=0$):} $E_- = \alpha_A - \frac{\beta^2}{|\alpha_B - \alpha_A|};\; E_+ = \alpha_B + \frac{\beta^2}{|\alpha_B - \alpha_A|}$  \\Atomic orbitals closest in energy dominate the bonding. Shift is lower as $\alpha_B - \alpha_A$ is higher.

    \squishline

    \item \textbf{Bond order} $b = \frac{1}{2} (N_{\text{bonding}} - N_{\text{antibonding}})$ \quad ($N$ nb of electrons)
    \item A non-bonding(core) orbital is a molec.\ orbit.\ whose occupation by \elec does not increase or decrease the bond order (equiv.\ of lone pairs)
    
    \item
    \begin{minipage}{0.7\columnwidth}
        $m=0 : $ symmetry $ \Sigma (s, p_z) \;  \Longrightarrow \sigma$-bonding $(s-s, s-p_z, p_z-p_z) \\
        |m|=1 : $ symmetry $ \Pi (p_x, p_y) \; \Longrightarrow \pi$-bonding $(p_x-p_x, p_y-p_y)$
    \end{minipage}
    \vline
    \hspace{0.05cm}
    \begin{minipage}{0.3\columnwidth}
        Other combinations give null matrix elements
    \end{minipage}
    \item A $\sigma$ bond is formed by the overlap of orbitals in an end-to-end fashion, with the \elec density concentrated between the nuclei of the bonding atoms. A $\pi$  bond is formed by the overlap of orbitals in a side-by-side fashion with the \elec density concentrated above and below the plane of the nuclei of the bonding atoms.
    \item $|\beta(\sigma)| > |\beta(\pi)|$ \qquad $\alpha(2p) - \alpha(2s) > |\beta| > 0$
\end{squishlist}

\squishline

\begin{squishlist}
    \item \textbf{$\mathbf{N>2}$ molecules:} Hybridisation $sp^{n-1}$,\, $n$ \textbf{Steric nb} (nb bonding partn.\ + nb lone pairs)
    \item LUMO: lowest unoccupied MO, HOMO: highest occupied MO
    \item \textbf{Hückel model}: (1) Only $\pi$-conjugated molecules (2) Consider only $p_z$ orbitals (3) $S_{ij} = \delta_{ij}$ (4) $H_{ii} = \alpha$ (5) Interact.\ entre C $=\beta$. \quad Then set $\det =0$ to find eigenenergies
    \item The nb of sign changes in the eigenvectors increases with increasing energy
\end{squishlist}

\newpage

\graysec{Light propagation, Optics}
\begin{squishlist}
    \item Helmholz equation $\Delta \vec{E} - \frac{n^2}{c^2} \partial_t^2 \vec{E} = 0$ \hspace{-0.3cm}
    \squishsep Monochrom.\ waves: $\vec{E}({\vec{r}, t}) = \Re[\vec{E}(\vec{r}e^{-i \omega t})]$
    \item \textbf{Angular spectrum representation} (2D Fourier transform in plane at position $z$): $\tilde{\vec{E}}(k_x, k_y, z) = \frac{1}{4\pi} \int \ud x \ud y \vec{E}(x,y,z) e^{-i(k_xx+k_yy)} \overset{\text{Helmh.}}{=} \tilde{\vec{E}}(k_x, k_y, 0) e^{\pm i k_z z}$ \\
    $\vec{E}(x,y,z) = \int \ud k_x \ud k_y \tilde{\vec{E}}(k_x,k_y,0)e^{\pm ik_z z} e^{i(k_x x + k_y y)}$
    \item $e^{\pm i k_z z}$ is the propag.\ in recipr.\ space. \squishsep if $\Im(k_z) \geq 0$ evanescent waves for $z \rightarrow \infty$ \\
    If large $k_x,k_y \rightarrow$ evanescent waves
\end{squishlist}
\graypar{Paraxial approximation}
\begin{squishlist}
    \item Consider fields in $z=0$ plane with smooth, slow spatial variations $k_x, k_y \ll k$
    \item $k_z = \pm k \sqrt{1 - \left(\frac{k_x}{k}\right)^2 - \left(\frac{k_y}{k}\right)^2}\approx k - \tfrac{k_x^2 + k_y^2}{2k} \Rightarrow \quad e^{ik_z z} = e^{ikz}e^{-i \tfrac{k_x^2 + k_y^2}{2k}z}$ \textbf{Gaussian beams}
    \item $\vec{E}(x,y,0) = \vec{E_0} \exp\left({-\tfrac{x^2 + y^2}{w_0^2}}\right) \quad \tilde{\vec{E}}(k_x,k_y,0) = \vec{E_0} \frac{w_0^2}{4\pi} \exp \left(- \frac{(k_x^2 + k_y^2)w_0^2}{4}\right)$
    \item $\vec{E}(x,y,z) = \vec{E_0} \frac{e^{ikz}}{1 + \tfrac{2iz}{kw_0^2}} \exp\left({-\tfrac{x^2 + y^2}{w_0^2}\left(\tfrac{1}{1 + \tfrac{2iz}{kw_0^2}}\right)}\right)$
    \item $\vec{E}(x,y,z) = \vec{E_0} \frac{w_0}{w(z)}\exp\left(-\frac{x^2 + y^2}{w(z)^2}\right) \exp \left(i \left(kz - \eta(z) + \frac{k(x^2 + y^2)}{2 R(z)}\right)\right)$
    \item $w_0$ (waist) is a measure of the width of the beam at its narrowest point.
    \item $\eta(z) = \arctan (z/z_R)$ Gouy phase
    \item $w(z) = w_0 \sqrt{1 + \left(\frac{z}{z_R}\right)^2}$ (hyperbolic relation), where $z_R = \frac{\pi n w_0^2}{\lambda}$ is the Rayleigh range.
    \item At a distance $z_R$ from the waist, the width $w$ of the beam is $\sqrt{2}$ larger than at the focus where $w=w_0$.
    \item The wavefronts have zero \textbf{curvature} (radius $R = \infty$) at the waist. Curvature increases away from the waist upt to $z_R$ then decreases approaching zero as $z \rightarrow \infty$ \\
    $R(z) = z \left(1 + \frac{z_R^2}{z^2}\right)$ \qquad$\frac{1}{R(z)} = \frac{z}{z^2 + z^2_R}$

    \item \textbf{Region I} (focal region $z \ll z_R$): $w(z) \approx w_0 \quad \eta=0 \quad R(z) \gg z_R$ \quad The phase does not depend on $x,y$. Looks like plane wave ($w_0 \rightarrow \infty$) propag.\ along $z$
    \item \textbf{Region II} (far-field $z \gg z_R$): $w(z) = w_0 \frac{z}{z_R} \quad R(z) \sim z$ \quad Waist diverges, phase (initially const) forms parabolas.
    \item \textbf{Spheric.\ wave} in parax.\ approx.: $\frac{1}{r}e^{ikr} = \frac{1}{\sqrt{\rho^2 + z^2}}e^{ik \sqrt{\rho^2 + z^2}} \approx \frac{1}{z \left(1 + \frac{p^2}{2z^2}\right)}e^{ikz+ik\frac{\rho^2}{2z}}$
    \item Complex beam parameter $q(z) = z + i z_R$ (when propagating in free space) \\ $\frac{1}{q(z)} = \frac{1}{R(z)} - i \frac{\lambda}{n \pi w^2(z)}$
\end{squishlist}

\graypar{Diffraction}
\begin{squishlist}
    \item \textbf{Fresnel diffraction} $\tilde{\vec{E}}(k_x,k_y,z) = \tilde{\vec{E}}(k_x,k_y,0)e^{ikz} \exp\left({-i\frac{(k_x^2 + k_y^2)z}{2k}}\right)$\\
    $\vec{E}(x,y,z) = \int \ud x' \ud y' \vec{E}(x',y',0) h(x-x', y-y', z)$ \quad
    $h (u,v,z)= \frac{2\pi k}{iz} e^{ikz} \exp \left(i \frac{k(u^2 + v^2)}{2z}\right)$
    \item \textbf{Fraunhofer condition} (far-field) $z \gg \frac{k(x'^2 + y'^2)}{2}$ \\
    $\vec{E}(x,y,z\rightarrow \infty) = \frac{2\pi k}{iz} e^{ikz}e^{ik\frac{(x^2+y^2)}{2z}} \tilde{\vec{E}}(\frac{kx}{z}, \frac{ky}{z},0)$ \\ 
    with $\tilde{\vec{E}}(\frac{kx}{z}, \frac{ky}{z},0) = \int \ud x' \ud y'\vec{E}(x', y', 0) e^{-i \left(\frac{kxx'}{z} + \frac{kyy'}{z}\right)} \cancel{e^{ik\frac{x'^2+y'^2}{}}}$\\
    Interpretation: each point of the $z=0$ plane emits a plane wave in the $k_x,k_y$ direction
\end{squishlist}
\graypar{Paraxial optics}
\begin{squishlist}
    \item Thin lens: no diffraction inside $ \Rightarrow \vec{E}_{\text{out}}(x,y) = t(x,y) \vec{E}_{\text{in}}(x,y)$
    \item Thickness function $\Delta(x,y) = \Delta_0 - \frac{x^2 + y^2}{2} \left( \frac{1}{R_1} - \frac{1}{R_2} \right)$
    \item Define $\frac{1}{f} = (n-1) \left( \frac{1}{R_1} - \frac{1}{R_2} \right)$ then $t(x,y) = \exp(-i \frac{k}{2f}(x^2 + y^2)) P(x,y)$
    \item In $f$ we find Fourier transform of what we had as input in position $d$: \\
    ${\vec{E}}(x,y) = \tilde{\vec{E}}_{\text{obj}}(\frac{kx}{f},\frac{ky}{f})\exp \left(i\frac{k(x^2 + y^2)}{2}\left[\frac{1}{f}-\frac{1}{d}\right]\right)$
    \item$\vec{E}_{\text{image}}(x_i,y_i) = Ae^{i\Phi} \int \ud x \ud y P(x,y) \exp \left(-\frac{ik}{z_2} [(x_i- M x_0)x + (y_i - M y_0)y]\right)$ \quad $M = -\frac{z_2}{z_1}$ \\
    $\vec{E}_{\text{image}}(x_i,y_i) \propto \delta [x_i - M x_0, y_i - M y_0]$
    \item For a finite aperture (e.g.\ $P(x,y)=1 \text{ for } x,y \in [-1,1]$) the image of a point source is the FT of pupil function, with $w = \frac{2\pi}{k}\frac{z_2}{L} = \frac{\lambda z_2}{L}$ and the min value of $z_2$ is $f$ ($z_1 = \infty$) $\Longrightarrow$
    best resolution $w \sim \frac{\lambda f}{L}$ 
\end{squishlist}

\graysec{Light-matter interaction}
\graypar{Out-of-equilibrium atom}
\begin{squishlist}
    \item $\kpsi = \frac{1}{N}(\ket{1s} + \epsilon \ket{2p}) \; \Rightarrow \; \ket{\psi(t)} = \ket{1s} + \epsilon e^{-i\omega_{2p}t}\ket{2p} \;  \Rightarrow \; \rho(t) = -e |\psi|^2 = \rho_0(\vec{r}) + \epsilon \delta \rho(\vec{r}, t)$
    \item $\delta \rho(\vec{r},t) = -e R_{1s}(\vr) R_{2p} \left\{
    \begin{aligned}
        &\frac{2z}{r}\cos(\omega t) \quad m=0 \\
        &\frac{2}{r} (x \cos(\omega t) \pm y\sin(\omega t)) \quad m=\pm 1
    \end{aligned}
    \right.$
    Time-dep.\ charge distribution
    \item Radiation emitted (frequency $\omega$), energy leaving atom. $E_{\text{initial}} \sim \braket{\psi|H|\psi} = \epsilon^2 \hbar \omega_{2p}$
    \item Decay of atom in $\ket{2p}$ at rate $\Gamma$: \textbf{spontaneous emission}
\end{squishlist}

\graypar{Semi-classical theory}
\begin{squishlist}
    \item $\hat{H} = \frac{(\hat{p} + e\vec{A})^2}{2m} - e\phi  = \ham _{\text{atom}} + \ham_{\text{lm}} \qquad \ham_{\text{lm}}= \frac{e}{m}\vec{p}\cdot \vec{A}_{\text{ext}} + \frac{e^2}{2m}\vec{A}_{\text{ext}}^2$ \quad ($\vec{A}^2$ neglig.\ )\\
    \textbf{Long wavelength approx.} \quad$ \hat{H} = \hat{H}_{\text{atom}} - {\hat{\vec{d}}\cdot \vec{E}(\vr=0,t)} = \hat{H}_{\text{atom}} + \hat{H}_{\text{lm}}(t)$ \quad $\vec{d} = e\vr$
    \item Perturb. th. $\Rightarrow P_{g\rightarrow e} = P_{e \rightarrow g} = \frac{T^2}{4 \pi \hbar^2}\left| \braket{g|\hat{H}_{lm}|e}\right|^2 \left(\frac{\sin (\delta T / 2)}{\delta T / 2}\right)^2 \quad \delta = \omega - \frac{|E_g - E_e}{\hbar}$  \\
    respectively absorption and stimulated emission.
\end{squishlist}

\graypar{Atomic transitions}
\begin{squishlist}
    \item Pert.\ theory: $P_{i\rightarrow f}(T) = \frac{1}{4\pi \hbar^2} \left|\braket{f|\ham_{lm}|i}\right|^2 T^2 \left|\frac{\sin(\delta T / 2)}{\delta T / 2}\right|^2$ \quad $\delta = \omega - \left|\frac{E_i - E_f}{\hbar}\right|$ \\
    $\Longrightarrow$ \textbf{Absorption} and \textbf{stimulated emission} caused by EM field interacting. \\
    Atoms absorb at discrete freq., which are the $E$ diff.\ between levels.
    \item \textbf{Selection rules}: $\braket{f|\ham_{\text{lm}}|i} = - \braket{f| \vec{d}|i} \cdot \vec{E}(t)$ \quad $d$ a vector operator $\Rightarrow$ Wigner-Eckart \\
    $z$-polarised light: \quad $j-1 \leq j' \leq j+1$ \quad $m = m'$ \quad ($\pi$-transitions) \\
    circ.\ polarised light: \quad $j-1 \leq j' \leq j+1$ \quad $m = m' \pm 1$ \quad ($\sigma$-transitions)
\end{squishlist}

\graypar{Two-level atom}
\begin{squishlist}
    \item $\ham = \ham_{\text{atom}} + \ham_{\text{lm}} = \hbar \omega_e \ket{e}\bra{e} \cancel{+ \hbar \omega_g \ket{g}\bra{g}} - \hat{\vec{d}}\cdot \vec{E}_0 \cos(\omega t)$ \quad $\hat{d} = d_0 (\ket{g}\bra{e} + \ket{e}\bra{g})$ \\
    $\Rightarrow \ham = \hbar \omega_e \ket{e}\bra{e} - \hbar \Omega_1 \cos(\omega t)(\ket{g}\bra{e} + \ket{e}\bra{g})$ \quad $\hbar \Omega_1 = d_0 E_0$ \textbf{Rabi frequency (pulsation)}
    \item $\kpsi = \gamma_g(t)\ket{g} + \gamma_e(t)e^{-i\omega_e t} \ket{e}$ and solve Schröd.\ $i\hbar \partial_t \kpsi = \ham \kpsi$
    \item $\tilde{\gamma_g}(t) = \cos(\Omega \frac{t}{2}) - i \frac{\delta}{\Omega} \sin( \Omega \frac{t}{2}) \quad \tilde{\gamma_e}- i \frac{\Omega_i}{\Omega} \sin( \Omega \frac{t}{2})$ \qquad $\Omega = \sqrt{\omega_1^2 + \delta^2}$ \quad $\gamma_{g/e} = \tilde{\gamma}_{g/e}e^{\pm \delta t / 2}$ \\
    $P_e(t) = |\tilde{\gamma}_e(t)|^2 = \frac{\Omega_1^2}{\omega_1^2 + \delta^2} \sin^2(\Omega \frac{t}{2})$ \quad \textbf{Rabi oscillations} \qquad $\delta = \omega - \omega_e$

    \squishline

    \textbf{Populations of two-level atoms}
    \item $N = N_e + N_g$ \quad \textbf{Decay:} $\der{t}{N_{e/g}} = - \Gamma_D N_{e/g}$ \quad \textbf{Injection:} $\der{t}{N_{e/g}} = \Lambda_{e/g}$ \\
    \textbf{Equilibrium:} $\bar{N}_{e/g} = \frac{\Lambda_{e/g}}{\Gamma_D}$
    \item $P_e(t \,| \,g \text{ at } t=t_0) = P_{g\rightarrow e}(t\, |\, g \text{ at } t=t_0) \; P(\text{survival between } t,t_0) = \text{Rabi} \cdot e^{-\Gamma_D (t-t_0)}$ \\
    $N_e(t) = \int_{-\infty}^{t} \ud t' \Lambda_g P_e \Longrightarrow N_e = \frac{N}{2} \frac{\Omega_1^2}{\Omega_1^2 + \delta^2 + \Gamma_D^2}$ \qquad $N_{e,\max} = \frac{N}{2} \Longrightarrow N_e < N_g$

    \squishline 

    \textbf{Dielectric susceptibility} \qquad for atom in $g$ at $t=t_0$:
    \item $\vec{P} \cdot V  = \Lambda_g \int_{-\infty}^{t} \braket{\hat{\vec{d}}}(t,t_0) e^{{-\Gamma_D(t-t_0)}} \ud t_0 = - \frac{N}{V}d_0 \frac{\Omega_1}{2} \frac{-\delta + i \Gamma_D}{\Gamma_D^2 + \Omega_1^2 + \delta^2}e^{-i\omega t} + c.c.$ 
    \item $\chi = \chi' + i \chi'' \quad \vec{P} = \varepsilon_0 \chi \frac{\vec{E_0}}{2} e^{-i \omega t} +c.c. = \varepsilon_0 (\chi' \cos(\omega t) + \chi'' \sin(\omega t)) \vec{E_0} $\\
    $\chi = \frac{N}{V} \frac{d^2}{\varepsilon_0 \hbar} \frac{\omega_e - \omega + i\Gamma_D}{\Gamma_D^2 + \Omega_1^2 + (\omega - \omega_e)^2}$ \quad $n^2 = 1 + \chi$. In most situat.\ $\chi \ll 1 \Rightarrow n \approx 1 \Rightarrow n = 1 + \frac{\chi'}{2} + \frac{\chi''}{2}$
    \item $k = \frac{n \omega}{c} \quad k'=\left(1 + \frac{\chi'}{2}\right)\frac{\omega }{c} \quad k'' = \frac{\chi''}{2}\frac{\omega}{c} \quad \Longrightarrow E(z,t) = E_0 e^{-k'' z} \cos(k'z - \omega t)$
    
    \squishline
    
    \item \textbf{Laser operation}:
    \textbf{Population inversion} $N_e - N_g > 0$ for light to be amplified
\end{squishlist}

\graysec{Molecular spectroscopy}
\graypar{Rotational spectroscopy}
\begin{squishlist}
    \item Assume free rigid linear rotor
    \item $\hat{H} Y_J^M(\theta, \phi) = E_J Y_J^M(\theta, \phi)$ with $E_J = \frac{\hbar^2}{2I} J (J+1) = B J(J+1)$. $B$ rotational constant
    \item Equilibrium occupations $P(J) = \frac{1}{q(T)} (2J+1) \exp(-E_J / k_B T)$ \\
    with $q(T) = \sum_{J=0}^{\infty} (2J+1) \exp(-E_J / k_B T)$ the partition function. \\ 
    \textbf{Note when drawing:} shouldn't be $0$ at $J=0$.

    \item $E_f = E_i \pm \hbar \omega$ ($+$: absorp.\, $-$: emiss.) \hspace{-0.3cm}
    \squishsep State $\ket{\varepsilon} \ket{J, m_J}$ (\elec vibrat.\ part, rotat.\ part)
    \item \textbf{Gross selection rule}: purely rotational spectra are observed only for molecules with $\vec{d_e} = \braket{\varepsilon|\vec{d}|\varepsilon} \neq 0$. AB molecules have $\vec{d_e} \neq 0$. A$_2$ have $\vec{d_e} = 0$.
    \item \textbf{Particular selection rules:} $\Delta J = \pm 1$, $\Delta m_J = 0, \pm 1$ ($J=J'$ not observable)
    \item \textbf{Centrifugal distors.:} $\Delta E(J) = 2B (J+1) - 4D(J+1)^3 \rightarrow$ uneq.\ spacing betw.\ spatial lines
\end{squishlist}

\graypar{Vibrational spectroscopy}
\begin{squishlist}
    \item $\psi(q) = N_{\nu} H_{\nu}(q) \exp(-q^2/2)$ with $q = \left(\frac{\mu k}{\hbar^2}\right)^{1/4}x$ \squishsep $E_{\nu} = \hbar \omega (\nu + \frac{1}{2})$
    \item \textbf{Gross selection rule:} $\left.\der{\vec{d}}{q}\right|_0 \neq 0$ \quad as $\vec{d} = \vec{d}_0 + \left. \D{q}{\vec{d}}\right|_0$
    \item \textbf{Specific selection rule:} $\Delta \nu = \pm 1$ (but $\Delta \nu = \pm 1$ transitions \\ are always accomp. by $\Delta J = \pm 1$ trans.)
\end{squishlist}

\graypar{Rovibrational spectroscopy}
\begin{squishlist}
    \item $E_{\nu, J} = (\nu + \frac{1}{2}) \hbar \omega + B_{\nu} J(J+1)$. Higher contrib. from $\nu$ \squishsep if rigid rotor $B_{\nu} = B$
    \item $\Delta \nu = \pm 1, \Delta J = \pm 1$.
    \item R-Branch: $\nu \leftrightarrow \nu + 1,\, J \leftrightarrow J+1$ \quad P-Branch:  $\nu \leftrightarrow \nu + 1, \,  J \leftrightarrow J-1$
    \item Absorption \quad R: $\Delta E = \hbar \omega + 2B(J+1)$, P: $\Delta E = \hbar \omega - 2BJ$ (rigid rotor)
    \item \textbf{Anharmonic Morse potential} $V(R) = D_e (1 - e^{-ax}), \, a = \sqrt{\frac{k}{2D_e}}$ \\
    $E_{\nu} = (\nu+\frac{1}{2}) \hbar \omega - (\nu - \frac{1}{2})^2 \hbar \omega \chi_e$, with $\chi_e$ anharmonicity constant \\
    Compresses R-branch and stretches P-branch. Allows for $\Delta \nu = \pm 2, \pm 3$
\end{squishlist}

\graypar{Raman spectroscopy}
\begin{squishlist}
    \item $\Delta E = \hbar \omega_i - \hbar \omega_s$ (incident/scattering)
    \item Selection rules are based on the polarisability tensor $\vec{d} = \bar{\bar{\alpha}} \vec{E}$
    \item Rotational: $\Delta J = \pm 1, \pm 2$ (only $\pm 2$ for lin.molec.); Vibrational $\Delta \nu  =\pm 1$; \\ Rovibrational $\Delta J = 0, \pm 2$
\end{squishlist}

\graypar{Rotational spectroscopy for $N>2$ molecules}
\begin{squishlist}
    \item Total ang.\ moment.\ $L^2 = L_a^2 + L_b^2 + L_c^2$, \quad $L_a = I_a \omega_a$ \quad $\Rightarrow$  \quad$\ham_{\text{rot}} = \frac{\hat{L}_a^2}{2I_a} + \frac{\hat{L}_b^2}{2I_b} + \frac{\hat{L}_c^2}{2I_c}$
    \item Linear rotor: $I_c = I_b, I_a=0$ \quad Prolate (rugby ball): $I_c = I_b > I_a$  \\
    Oblate (pancake): $I_c > I_b = I_a$ \quad Spherical: $I_a = I_b=I_c$ \quad Asym.:  $I_a \neq I_b \neq I_c$
    \item e.g. for prolate: $\ham_{\text{rot}} = \frac{\hat{L}_a^2}{2I_a} + \frac{\hat{L}_b^2}{2I_b} + \frac{\hat{L}^2 - \hat{L}_a^2 - \hat{L}_b^2}{2I_b} $ \quad $E_{\text{rot}} = B J(J+1) + (A - B)K^2$
\end{squishlist}

\graypar{Vibrations of $N>2$ molecules}
\begin{squishlist}
    \item Sum of $3N-6$ indip.\ harm.\ oscillat.\ ($3N-5$ if linear) \quad $E_V = \sum_{i=1}^{3N-6} \hbar \omega_i (\nu_i + \frac{1}{2})$
    \item Normal coordinates $Q_i = \sum_{k=1}^{3N} c_{ik} q_k$ \squishsep Select.\ rules: $\left. \D{Q_i}{\vec{d}}\right|_0 \neq 0 $ and $\Delta \nu_i = \pm 1$
    \item \textbf{Electronic transit.\ rates:} $\Gamma_{fi} \propto |\braket{\psi_f | \vec{d}|\psi_i}|^2, \, \vec{d} = \sum_i z_i \vR_i + \sum_i e \vr_i \equiv \vec{d}_N(\{\vR\}) + \vec{d}_e(\{\vr\}) $\\
    $\kpsi = \ket{\psi^{\text{vib}}} \ket{\psi^{\text{el}}} \Rightarrow \braket{\psi|\vec{d}_N|\psi} = 0 \Longrightarrow \braket{\psi_f | \vec{d}|\psi_i} = \braket{\psi_f^{\text{vib}}|\psi_i^{\text{vib}}} D^{\text{el}}(\vR_{\text{eq}})$ \\
    $D^{\text{el}}(\vR_{\text{eq}}) \neq 0$ electr.\ trans.\ dipole moment \quad $\braket{\psi_f^{\text{vib}}|\psi_i^{\text{vib}}} \neq 0$ Franck-London factors
\end{squishlist}

\graysec{Mixed}
\begin{squishlist}
    \item \textbf{Viriel theorem:} if $V = \alpha x^s \Longrightarrow \braket{T} = \frac{1}{2}s \braket{V}$
    \squishsep $\frac{1}{2m}\hat{p}^2 = - \frac{\hbar^2}{2m} \dder{x}{}$
    \qquad $[\hat{x}, \hat{p}] = i \hbar$
    \item $Y_{lm}(-\vr) = (-1)^l Y_{lm}(\vr)$ \squishsep $J_{\pm} \ket{J,m_J} = \hbar \sqrt{(J \mp m_J)(J \pm m_J + 1)} \ket{J,m_J \pm 1 }$
    \item \textbf{Wigner-Eckart theorem:} if $T_{kq}$ is $q$ component of a spherical tensor of rank $k$ \\
    $\braket{n j m_j | T_{kq} | n' j' m_j'} = \frac{1}{\sqrt{2j'+1}} \underbrace{\braket{n j | T_k | n'j'}}_{\text{Red.\ matr.\ elem.}} \underbrace{\braket{j m_j ; k q | j' m_j'}}_{\text{Clebsch-Gordan}}$
    \item \textbf{Pert.\ th.:} $E_n^{(1)}=\braket{\phi_n|V|\phi_n} \quad \ket{\psi^1_n} = \sum_{m\neq n}\frac{\braket{\phi_m|V|\phi_n}}{\varepsilon_n - \varepsilon_m} \ket{\phi_m}$ \quad
    $E_n^{(2)} = \sum_{m\neq n}\frac{|\braket{\phi_m|V|\phi_n}|^2}{\varepsilon_n - \varepsilon_m}$
    \item \textbf{Elliptical coordinates:} $\mu = \frac{1}{R} \left(|\vec{r}-\vec{R}_A| + | \vec{r} - \vec{R}_B| \right) \quad 
    \nu = \frac{1}{R} \left(|\vec{r}-\vec{R}_A| - | \vec{r} - \vec{R}_B| \right) \\
    \int d^3r = \int_1^\infty d\mu \int_{-1}^1 d \nu \int_0^{2\pi} d \phi \frac{R^3}{8} ( \mu^2 - \nu^2).$
    \item $\frac{1}{|\vr_1 - \vr_2|} = \sum_{l=0}^{\infty} \sum_{m=-l}^{l} \frac{4\pi}{2l+1} \frac{r_<^l}{r_>^{l+1}} Y_l^{-m}(\theta_1, \phi_1) Y_l^{m}(\theta_2, \phi_2) \quad r_< = \min(r_1,r_2)$
    \item $\vec{u} = \sqrt{\frac{4\pi}{3}}\left[Y_1^1(\theta,\phi) \vec{e}_+^* + Y_1^0(\theta,\phi)\vec{e}_0 + Y_1^{-1}(\theta,\phi)\vec{e}_-^*\right]$ où $\vec{e}_{\pm} = \frac{1}{\sqrt{2}}(\vec{e}_x \pm \vec{e}_y)$, $\vec{e}_0 = \vec{e}_z$
\end{squishlist}

\graysec{Integration stuff}
\begin{squishlist}
    \item $\int_{0}^{b} x^2 e^{-ax} \ud x = \frac{1}{a^3}\left[2 - (2 + a^2 b^2 + 2ab)e^{-ab}\right]$ 
    \squishsep $\int_{b}^{\infty} x e^{-ax} \ud x = \frac{1}{a^2} \left(1 + ab\right)e^{-ab}$
    \item $\int_{0}^{\infty} r^n e^{-ar} \ud r = \frac{n!}{a^{n+1}}$
\end{squishlist}