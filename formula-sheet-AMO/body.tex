% Matteo Veneziano -- Atomic and Molecular Physics and Optics Formula Sheet

\graysec{Atomic physics}
\graypar{Hydrogen atom spectrum}
\begin{squishlist}
    \item \textbf{Bohr:} $E_n = - \frac{z^2}{n^2} R_y = - \frac{z^2}{n^2}h c R_{\infty}$ \quad $R_{\infty} = $\\
     $\frac{1}{\lambda_n} = - R_{\infty} \frac{m_r}{m_e} \left[ \frac{1}{n^2} - \frac{1}{n_0^2}\right]$ when trans.between states
    
    \item \textbf{Selection rules} (electric dipole transitions)
        \begin{enumerate}
            \item The only allowed transitions involv a change in parity
            \item $\Delta l = 0, \pm1 \overset{(1)}{\Longrightarrow} \Delta l = \pm 1$ \quad 3. $\Delta m = 0, \pm 1$
        \end{enumerate}
    
        \item \textbf{Orbital magnetic moment:} the orb. ang. moment. of $e^-$ causes \\
        $m_z = \frac{-e}{2m_e}l_z \Rightarrow \vec{m} = \gamma_e \vec{l}$ with $\gamma_e = -\frac{e}{2m_e}$\\
        $m_z = \gamma_e m_l \hbar = -\mu_B m_l$ with $m_l = l, l-1, \ldots, -l \quad \mu_B = \frac{e\hbar}{2m_e}$

        \item \textbf{Spin magnetic moment:} \\ $\vec{m} = g_e \gamma_e \vec{s} \Longrightarrow mz = -g_e \mu_B m_s$ with $m_s=\pm1/2$ and $g_e = 2.0023$ the $g$-factor
    \end{squishlist} 
    
    \squishline

\textbf{Relativistic effects}
\begin{squishlist}
    \item \textbf{Spin-orbit coupling:} a particle of mass $m_e$ and charge $-e$ moving at $\vec{v}$ in $\vec{E}$ experiences magnetic field $\vec{B} = \frac{1}{c^2}\vec{E}\times \vec{v}$. There is energy of interaction between $\vec{B}$ and the magnetic moment of particle: $H_{so} = \xi(r) \vec{l}\cdot \vec{s}$ \\
    Causes \textbf{Fine structure}: $\left\{\begin{aligned}
        \text{parall.\ spin and orb.\ ang.\ mom.\ } &\Rightarrow \text{High-energy arrangem.\ } \\
        \text{antiparall.\ spin and orb.\ ang.\ mom.\ } &\Rightarrow \text{Low-energy arrangem.\ }
    \end{aligned} \right.$
    The energy of level $j = l+1/2$ lies above $j = l-1/2$

    \item Other relativistic effects (e.g.\ Darwin term for $s$ wavefunctions)
    \item Notation for $e^-$ orbital/level: $n^{2S+1}L_J$ (e.g. $2^2 p_{1/2}$)
\end{squishlist}

\squishline

\textbf{Nuclear effects}
\begin{squishlist}
    \item Finite size (charge radius is not exactly 0)
    \item Finite mass (different isotopes have diff.\ spectra)
    \item \textbf{Hyperfine structure}: the nucleus has finite angular momentum $\vec{I}$ thus magnetic moment $\vec{\mu_N} = g_I \mu_N \frac{\vec{I}}{\hbar}$ with $\mu_N$ the nucleon magneton.
    Interaction between magnetic moments of nucleus and of $e^-$:
    $H_{HFS} \sim \vec{\mu_N}\cdot \vec{\mu_{e^-}}$
    \item Quadruple moment of nucleus
\end{squishlist}

\graypar{Atoms in external fields}
\begin{squishlist}
    \item \textbf{Zeeman effect:} the magnetic moments of the electrons interact with externally applied magnetic fields
    \begin{itemize}
        \item \textbf{Normal} (zero spin angular momentum) \\
        $H^{(1)} = - m_z B = -\gamma_e l_z B$ (there is diamagnetic term too, but small for small $m$)
        \item \textbf{Anomalous} (non-zero spin angular momentum, more common) \\
        $H^{(1)} = -\gamma_e (\vec{L} + 2\vec{S}) \cdot \vec{B}$
    \end{itemize}

    \item \textbf{Stark effect:} interaction with electric field of strength $E$ in $z$-direction \\
    $H^{(1)} = - \mu_z E = e z E$ where $\mu_z = -e z$ is $z$-compon./ of electric dipole moment operator.
    The correction to energy is linear in the case of a degeneracy between the two wavefunctions that the perturbation mixes, e.g.\ $2s_{1/2}, 2p_{1/2}$. Otherwise 2$^{\text{nd}}$ order perturbation.
\end{squishlist}


\graypar{Helium atom}
\begin{squishlist}
    \item $\hat{H} = - \frac{\hbar^2}{2m_e}(\nabla^2_1 + \nabla^2_2) - \frac{2e^2}{4\pi \varepsilon_0 r_1} - \frac{2e^2}{4\pi \varepsilon_0 r_2} + \frac{e^2}{4\pi \varepsilon_0 r_{12}} = \hat{H}_{01} + \hat{H}_{02} + \hat{H}_c$
    \item \textbf{Non-interacting spectrum} ($H_c = 0$): $\ham$ is sum of two indep.\ terms, thus wavefunc.\ of the two $e^-$ is the product of the two hydrogenic wavefunctions \\
    $ \ham = \hat{H}_{01} + \hat{H}_{02} \Longrightarrow \psi(r_1,r_2) = \phi_{n_1 l_1 m_1}(r_1) \; \phi_{n_2 l_2 m_2}(r_2)$ \qquad $E = -z^2 R_y \left( \frac{1}{n_1^2}+ \frac{1}{n_2^2} \right)$ \\
    Ground state $E_0 = -8 R_y = -108.8$ eV. First ionisation treshold $4$ eV. \\
    Degeneracy for different $l$.

    \item \textbf{Electron-electron interactions} (repulsion term): \\
    First order correction to energy $\Rightarrow$ Coulomb integral \\
    $ \displaystyle J = j_0 \int \left|\psi_{n_1 l_1 m_1}(\vec{r_1})\right|^2 \left(\frac{1}{r_{12}}\right) |\psi_{n_2 l_2 m_2}(\vec{r_2})|^2  \ud \tau_1 \ud \tau_2$ [RESULTS?] \\
    We can also use \textbf{variational method} to find an effective charge $z_e$ which minimises the energy. Guess function $\psi(\vr_1 \vr_2) = \frac{z_e^3}{\pi}\exp\left(-\frac{z_e}{a_0}(\vr_1 + \vr_2) \right) = \phi_{1s,Z_e}(\vr_1)\phi_{1s,Z_e}(\vr_2)$
    We find $E_{\min} = -2 R_y (z - 5/16)^2$
    \vspace{0.1cm}

    \item \textbf{Excited states:} when two $e^-$ occupy different orbitals the wavefuncs.\ are either
    $\psi_{n_1l_1m_1}(\vec{r}_1) \psi_{n_2l_2m_2}(\vec{r}_2)$ or $\psi_{n_1l_1m_1}(\vec{r}_2) \psi_{n_2l_2m_2}(\vec{r}_1)$. We use degenerate pert.\ theory $\braket{\phi_{a,s}|h_c|\phi_{a,s}}$ with
    $\ket{\phi_{a,s}} = \frac{1}{\sqrt{2}} \left( \ket{n_1l_1m_1}\ket{n_2l_2m_2} \mp \ket{n_2 l_2 m_2}\ket{n_1l_1m_1}\right)$ \\
    $\Rightarrow \delta E_a = J - K \qquad \delta E_s = J + K$ \qquad $J$: direct term, $K$: exchange term (both $> 0$)

    \item Notation: $n^{2S_{\text{tot}}+1} L_{\text{tot}\; J_{\text{tot}}}$
\end{squishlist}

\graypar{Many-electron atoms}
\begin{squishlist}
    \item $\ham = \sum_{i=1}^{Z}\left( - \frac{\hbar^2}{2m}\Delta_i - \frac{Ze^2}{4\pi \varepsilon_0 r_i}\right) + \sum_{i < j} \frac{e^2}{4\pi \varepsilon_0 |\vr_i - \vr_j}$ \\
    acts on $\mathcal{H} = \otimes_i \mathcal{H}_i,\, \mathcal{H}_i =  \mathcal{H}_{i, orb} \otimes \mathcal{H}_{i, spin}$
    \item \textbf{Central field approximation}: the inter-electronic contribution is replaced by a screening of the nucleus $\Rightarrow Z_{\mathrm{eff}}$ \quad $V(r) \sim \frac{Z(r)}{r}$
    \item Symetrisation $\rightarrow$ Slater determinants: write wavefuncs of many \elec as product of wavefuncs of \elec, all antisymetr.
    \item The best atomic orbitals are found by numerical solution of the \textbf{Hartree-Fock} eqs.\ : Schrödinger = Coulomb repulsion + purely quantum contribution.
    The wavefuncs.\ are expressed as Slater determinants.

    \item \textbf{Aufbau principle:} fill the orbitals with increasing values of $n+l$
    \item \textbf{Hund's rules:} (1) Maximise total spin (2) Maximise total $L$ (3) Minimise/Maximise total $J$ if shell is less/more than half filled
\end{squishlist}

\graysec{Molecular physics}
\textbf{Born-Oppenheimer approximation}: nuclei are heavier, thus slower and electrons can respond almost instant.\ to displacement of the nuclei. We treat nuclei as fixed in position ("parametrically").

\squishline

\begin{squishlist}
    \item \textbf{H$_2^+$ molecule}: $\ham_e = - \frac{\hbar^2}{2m_e} \nabla^2 - \frac{j_o}{|\vr - \vR_A|} - \frac{j_o}{|\vr - \vR_B|} + \frac{j_0}{R}$ \quad $R$ the dist.\ betw.\ nuclei \\
    Solved by variat.\ method with \textbf{trial wavefunction} $\psi_i(\vr) = a_i \phi_A(\vR) + b_i \phi_B(\vR)$, with $\phi_A(\vr)=\phi_{1s}(\vr - \vR_A), \phi_B(\vr)=\phi_{1s}(\vr - \vR_B)$ \\
    We find \textbf{eigenfuncs} $\psi_+(\vr) = N_+ (\phi_A(\vr)+\phi_B(\vr))$ (even) and $\psi_-(\vr) = N_- (\phi_A(\vr)-\phi_B(\vr))$ \\
    $\psi_+$ is bonding, $\psi_-$ is antibonding. An antibonding orbital is more antibonding than a bonding orb.\ is bonding.
\end{squishlist}

\squishline

\begin{squishlist}
    \item \textbf{H$_2$ molecule:} Heitler-London (Valence bond) approach \\
    $\psi_S(1,2) = \frac{1}{\sqrt{2(1+s^2)}} \left[\phi_A(1)\phi_B(2) + \phi_B(1)\phi_A(2)\right] \chi_S$ \\
    $\psi_T(1,2) = \frac{1}{\sqrt{2(1-s^2)}} \left[\phi_A(1)\phi_B(2) - \phi_B(1)\phi_A(2)\right] \chi_T$
\end{squishlist}

\squishline

\begin{squishlist}
    \item \textbf{Other diatomic molecules:} Rayleigh-Ritz method (need to solve secular equations)\\
    $\det \begin{pmatrix}
        \alpha_A - E & \beta - ES \\
        \beta - ES   & \alpha_B - E
    \end{pmatrix} = 0; \quad \alpha_A = \braket{A|H|A};\; \alpha_B = \braket{B|H|B};\; \beta = \braket{A|B};\; S = \braket{A|B}$ \\
    $E_- = \alpha_A - \frac{\beta^2}{|\alpha_B - \alpha_A};\; E_+ = \alpha_B + \frac{\beta^2}{|\alpha_B - \alpha_A|}$ Atomic orbitals closest in energy dominate the bonding. Shift is lower as $\alpha_B - \alpha_A$ is higher.

    \item \textbf{Bond order} $b = \frac{1}{2} (N_{\text{bonding}} - N_{\text{antibonding}})$ ($N$ nb of electrons)
    
    \item $m=0 : \Sigma (s, p_z) \; |m|=1 : \Pi (p_x, p_y) \Longrightarrow \sigma$-bonding $(s-s, s-p_z, p_z-p_z), \pi$-bonding $(p_x-p_x, p_y-p_y)$
\end{squishlist}

\squishline

\begin{squishlist}
    \item \textbf{$\mathbf{N>2}$ molecules:} Hybridisation
    \item LOMO, HOMO, RULES OF HYBRIDISATION
    \item \textbf{Hinckel model}: (1) Only $\pi$-conjugated molecules (2) Consider only $p_z$ orbitals (3) $S_{ij} = \delta_{ij}$ (4) $H_{ij} = \alpha$
\end{squishlist}

\graysec{Light propagation, Optics}
\begin{squishlist}
    \item Helmholz equation $\Delta \vec{E} - \frac{n^2}{c^2} \partial_t^2 \vec{E} = 0$
    \item Monochromatic waves: $\vec{E}({\vec{r}, t}) = \Re[\vec{E}(\vec{r}e^{-i \omega t})]$
    \item \textbf{Angular spectrum representation} (2D Fourier transform in plane at position $z$): $\tilde{\vec{E}}(k_x, k_y, z) = \tilde{\vec{E}}(k_x, k_y, 0) e^{\pm i k_z z}$ \squishsep if $\Im(k_z) \geq 0$ evanescent waves for $z \rightarrow \infty$
\end{squishlist}
\graypar{Paraxial approximation}
\begin{squishlist}
    \item Smooth, slow spatial variations $k_x, k_y \ll k$
    \item $k_z \approx k - \tfrac{k_x^2 + k_y^2}{2k} \quad \Rightarrow \quad e^{ik_z z} = e^{ikz}e^{-i \tfrac{k_x^2 + k_y^2}{2k}z}$ \textbf{Gaussian beams}
    \item $\vec{E}(x,y,0) = \vec{E_0} \exp\left({-\tfrac{x^2 + y^2}{w_0^2}}\right) \quad \tilde{\vec{E}}(k_x,k_y,0) = \vec{E_0} \frac{w_0^2}{4\pi} \exp \left(- \frac{(k_x^2 + k_y^2)w_0^2}{4}\right)$
    \item $\vec{E}(x,y,z) = \vec{E_0} \frac{e^ikz}{1 + \tfrac{2iz}{kw_0^2}} \exp\left({-\tfrac{x^2 + y^2}{w_0^2}\left(\tfrac{1}{1 + \tfrac{2iz}{kw_0^2}}\right)}\right)$
    \item $\vec{E}(x,y,z) = \vec{E_0} \frac{w_0}{w(z)}\exp\left(-\frac{x^2 + y^2}{w(z)^2}\right) \exp \left(i \left(kz - \eta(z) + \frac{k(x^2 + y^2)}{2 R(z)}\right)\right)$
    \item $w_0$ (waist) is a measure of the width of the beam at its narrowest point.
    \item $\eta(z) = \arctan (z/z_R)$ Gouy phase
    \item $w(z) = w_0 \sqrt{1 + \left(\frac{z}{z_R}\right)^2}$ (hyperbolic relation), where $z_R = \frac{\pi n w_0^2}{\lambda}$ is the Rayleigh range.
    \item At a distance $z_R$ from the waist, the width $w$ of the beam is $\sqrt{2}$ larger than at the focus where $w=w_0$.
    \item The wavefronts have zero curvature (radius $R = \infty$) at the waist. Curvature increases away from the waist upt to $z_R$ then decreases approaching zero as $z \rightarrow \infty$
    $\frac{1}{R(z)} = \frac{z}{z^2 + z^2_R}$
    \item Complex beam parameter $q(z) = z + i z_R$ (when propagating in free space) \\ $\frac{1}{q(z)} = \frac{1}{R(z)} - i \frac{\lambda}{n \pi w^2(z)}$
\end{squishlist}

\graypar{Fresnel}
\begin{squishlist}
    \item \textbf{Fresnel diffraction} $\vec{E}(x,y,z) = \int \vec{E}(x',y',0) h(x-x', y-y', z)$ \\
    $h = \frac{2\pi k}{iz} e^{ikz} \exp \left(i \frac{k(u^2 + v^2)}{2z}\right)$
\end{squishlist}
\graypar{Paraxial optics}
\begin{squishlist}
    \item Thin lens: no diffraction inside $ \Rightarrow \vec{E}_{\text{out}}(x,y) = t(x,y) \vec{E}_{\text{in}}(x,y)$
    \item Thickness function $\Delta(x,y) = \Delta_0 - \frac{x^2 + y^2}{2} \left( \frac{1}{R_1} - \frac{1}{R_2} \right)$
    \item Define $\frac{1}{f} = (n-1) \left( \frac{1}{R_1} - \frac{1}{R_2} \right)$ then $t(x,y) = \exp(-i \frac{k}{2f}(x^2 + y^2)) P(x,y)$
    \item In $f$ we find Fourier transform of what we had as input in position $d$
    \item For a finite aperture (ex. $P(x,y)=1 \text{ for } x,y \in [-1,1]$) best resolution $w \sim \frac{\lambda f}{L}$
\end{squishlist}

\graysec{Light-matter interaction}
\graypar{Out-of-equilibrium atom}
\begin{squishlist}
    \item $\kpsi = \frac{1}{N}(\ket{1s} + \epsilon \ket{2p}) \; \Rightarrow \; \ket{\psi(t)} = \ket{1s} + \epsilon e^{-i\omega_{2p}t}\ket{2p} \; \Rightarrow \; \rho(t) = -e |\psi|^2 = \rho_0(\vec{r}) + \epsilon \delta \rho(\vec{r}, t)$
    \item $\delta(\vec{r},t) = -e R_{1s}(\vr) R_{2p} \left\{
    \begin{aligned}
        &\frac{2z}{r}\cos(\omega t) \quad m=0 \\
        &\frac{2}{r} (x \cos(\omega t) \pm y\sin(\omega t)) \quad m=\pm 1
    \end{aligned}
    \right.$
    \item Radiation emitted (frequency $\omega$), energy leaving atom. $E_{\text{initial}} \sim \braket{\psi|H|\psi} = \epsilon^2 \hbar \omega_{2p}$
    \item Decay of atom in $\ket{2p}$ at rate $\Gamma$: \textbf{spontaneous emission}
\end{squishlist}

\graypar{Semi-classical theory}
\begin{squishlist}
    \item $\hat{H} = \frac{(\hat{p} + e\vec{A})^2}{2m} - e\phi \quad \overset{\text{long wavelength}}{\Longrightarrow} \quad \hat{H} = \hat{H}_{\text{atom}} - {\hat{\vec{d}}\cdot \vec{E}(\vr=0,t)} = \hat{H}_{\text{atom}} - \hat{H}_{\text{lm}}(t)$
    \item Perturb. th. $\Rightarrow P_{g\rightarrow e} = P_{e \rightarrow g} = \frac{T^2}{4 \pi \hbar^2}\left| \braket{g|\hat{H}_{lm}|e}\right|^2 \left(\frac{\sin (\delta T / 2)}{\delta T / 2}\right)^2 \quad \delta = \omega - \frac{|E_g - E_e}{\hbar}$  \\
    respectively absorption and stimulated emission.
\end{squishlist}

\graysec{Molecular spectroscopy}
\graypar{Rotational spectroscopy}
\begin{squishlist}
    \item Assume free rigid linear rotor
    \item $\hat{H} Y_J^M(\theta, \phi) = E_J Y_J^M(\theta, \phi)$ with $E_J = \frac{\hbar^2}{2I} J (J+1) = B J(J+1)$. $B$ rotational constant
    \item Equilibrium occupations $P(J) = \frac{1}{q(T)} (2J+1) \exp(-E_J / k_B T)$ \\
    with $q(T) = \sum_{J=0}^{\infty} (2J+1) \exp(-E_J / k_B T)$ the partition function. \\ 
    \textbf{Note when drawing:} shouldn't be $0$ at $J=0$.

    \item $E_f = E_i \pm \hbar \omega$ ($+$: absorption, $-$: emission)
    \item State $\ket{\varepsilon} \ket{J, m_J}$ (electron vibrational part, rotational part)
    \item \textbf{Gross selection rule}: purely rotational spectra are observed only for molecules with $\vec{d_e} = \braket{\varepsilon|\vec{d}|\varepsilon} \neq 0$. AB molecules have $\vec{d_e} \neq 0$. A$_2$ have $\vec{d_e} = 0$.
    \item \textbf{Particular selection rules:} $\Delta J = \pm 1$, $\Delta m_J = 0, \pm 1$ ($J=J'$ not observable)
    \item \textbf{Centrifugal distorsion:} $\Delta E(J) = 2B (J+1) - 4D(J+1)^3 \longrightarrow$ unequal spacing between spatial lines
\end{squishlist}

\graypar{Vibrational spectroscopy}
\begin{squishlist}
    \item $\psi(q) = N_{\nu} H_{\nu}(q) \exp(-q^2/2)$ with $q = \left(\frac{\mu k}{\hbar^2}\right)^{1/4}x$ \squishsep $E_{\nu} = \hbar \omega (\nu + \frac{1}{2})$
    \item \textbf{Gross selection rule:} $\left.\dder{\vec{d}}{q}\right|_0 \neq 0$
    \item \textbf{Specific selection rule:} $\Delta \nu = \pm 1$ (but $\Delta \nu = \pm 1$ transitions are always accomp. by $\Delta J = \pm 1$ trans.)
\end{squishlist}

\graypar{Rovibrational spectroscopy}
\begin{squishlist}
    \item $E_{\nu, J} = (\nu + \frac{1}{2}) \hbar \omega + B_{\nu} J(J+1)$. Higher contrib. from $\nu$ \squishsep if rigid rotor $B_{\nu} = B$
    \item $\Delta \nu = \pm 1, \Delta J = \pm 1$.
    \item R-Branch: $\nu \rightarrow \nu + 1,\, J \rightarrow J+1$ \quad P-Branch:  $\nu \rightarrow \nu + 1, \,  J \rightarrow J-1$
    \item Absorption \quad R: $\Delta E = \hbar \omega + 2B(J+1)$, P: $\Delta E = \hbar \omega - 2BJ$ (rigid rotor)
    \item \textbf{Anharmonic Morse potential} $V(R) = D_e (1 - e^{-ax}), \, a = \sqrt{\frac{k}{2D_e}}$ \\
    $E_{\nu} = (\nu+\frac{1}{2}) \hbar \omega - (\nu - \frac{1}{2})^2 \hbar \omega \chi_e$, with $\chi_e$ anharmonicity constant \\
    Compresses R-branch and stretches P-branch. Allows for $\Delta \nu = \pm 2, \pm 3$
\end{squishlist}

\graypar{Raman spectroscopy}
\begin{squishlist}
    \item $\Delta E = \hbar \omega_i - \hbar \omega_s$ (incident/scattering)
    \item Selection rules are based on the polarisability tensor $\vec{d} = \bar{\bar{\alpha}} \vec{E}$
    \item Rotational: $\Delta J = \pm 1, \pm 2$ (only $\pm 2$ for lin.molec.); Vibrational $\Delta \nu  =\pm 1$; \\ Rovibrational $\Delta J = 0, \pm 2$
\end{squishlist}