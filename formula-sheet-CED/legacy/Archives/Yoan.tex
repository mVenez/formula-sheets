\graysec{Maxwell equations}
$$\boxed{\begin{split}
	\textsc{Coulomb's law} \qquad &\boldsymbol{\nabla} \cdot \boldsymbol{D} = \rho\\
	\textsc{Ampère's law} \qquad &\boldsymbol{\nabla} \wedge \boldsymbol{H} = \boldsymbol{J} + \frac{\partial \boldsymbol{D}}{\partial t}\\
	\textsc{Faraday's law} \qquad &\boldsymbol{\nabla} \wedge \boldsymbol{E} + \frac{\partial \boldsymbol{B}}{\partial t} = 0\\
	\textsc{Absence of magnetic poles} \qquad &\boldsymbol{\nabla} \cdot \boldsymbol{B} = 0
	\end{split}}$$

\graypar{Lorentz Force}
\squishlist
\item $ \boldsymbol{F} = q(\boldsymbol{E}+\boldsymbol{v} \wedge \boldsymbol{B}) $

\item $ \int_{\Sigma}\boldsymbol{E} \cdot \boldsymbol{d\sigma} = \frac{Q}{\epsilon_0}$

\item $\int_{\Gamma} \boldsymbol{E} \cdot \boldsymbol{dl} = -\frac{\text{d}}{\text{d}t}\int_{\Sigma} \boldsymbol{B} \cdot \boldsymbol{d\sigma}$

\item Continuity Equation: $\frac{\partial \rho}{\partial t} + \boldsymbol{\nabla}\cdot\boldsymbol{j} = 0$
\squishend

\graypar{Electromagnetic Potentials} 
\squishlist
\item $\boldsymbol{B} = \boldsymbol{\nabla} \times \boldsymbol{A}$

\item $\boldsymbol{E} = -\boldsymbol{\nabla} \Phi - \frac{\partial \boldsymbol{A}}{\partial t}$
\squishend
\graypar{Lorenz gauge}
$$
\boldsymbol{\nabla} \cdot \boldsymbol{A} + \frac{1}{c^2}\frac{\partial \Phi}{\partial t}=0
$$
\[      \implies
        \begin{cases}
            \frac{1}{c^2}\frac{\partial^2 \Phi }{\partial t^2} - \boldsymbol{\nabla}^2 \Phi = \square \phi = \frac{\rho}{\epsilon_0} \\
            \frac{1}{c^2}\frac{\partial^2 \boldsymbol{A}}{\partial t^2} - \boldsymbol{\nabla}^2 \boldsymbol{A} = \square \vec{A} =  \frac{\boldsymbol{J}}{c^2\epsilon_0}
        \end{cases}
\]

\graypar{Electrostatics.}
$$
\boldsymbol{\nabla}^2\Phi = -\frac{\rho}{\epsilon_0} \quad \text{with} \quad |\boldsymbol{E}| \overset{|\boldsymbol{x}|\to \infty}{\longrightarrow} 0
$$

$$
\Phi(\boldsymbol{x}) = \frac{1}{4\pi \epsilon_0} \int \text{d}^3x' \frac{\rho(\boldsymbol{x})}{|\boldsymbol{x}-\boldsymbol{x}'|}
$$

\graypar{Dirac Functions.}
$$\int \text{d}^nx f(\boldsymbol{x})\delta^n(\boldsymbol{x}-\boldsymbol{x}_0)= f(\boldsymbol{x}_0)
$$
$$
\int \text{d}^nx f(\boldsymbol{x})\boldsymbol{\nabla}\delta^n(\boldsymbol{x}-\boldsymbol{x}_0)= - \boldsymbol{\nabla}f(\boldsymbol{x}_0)
$$

$$
\delta^n(g(\boldsymbol{x})) = \sum_i \frac{\delta(\boldsymbol{x}-\boldsymbol{x}_i)}{\left|\frac{\partial g}{\partial \boldsymbol{x}} \right| } : \quad \boldsymbol{x}_i = \text{ zeros of }g
$$

\graypar{Green functions - static case}
$$
-\boldsymbol{\nabla}_{\boldsymbol{x}}^2 G(\boldsymbol{x}, \boldsymbol{x}') = \delta^3(\boldsymbol{x}-\boldsymbol{x}')
$$

$$
\Phi(\boldsymbol{x}) = \int_V \text{d}^3 x' G(\boldsymbol{x}, \boldsymbol{x}')\rho(\boldsymbol{x}') + \int_{\partial V} \boldsymbol{dS} \cdot \left[G(\boldsymbol{x}, \boldsymbol{x}')\boldsymbol{\nabla}\Phi(\boldsymbol{x}')-\Phi(\boldsymbol{x}')\boldsymbol{\nabla}_{\boldsymbol{x}'}G(\boldsymbol{x}, \boldsymbol{x}')\right]
$$
Find $G$ by the images method in order to remove the $Phi$ term on the edge $\partial V$ that is not given. $G$ can be treated as a potential if $G|_{\partial V}=0$ or if $\boldsymbol{\nabla}G|_{\partial V}=0$. 

\begin{enumerate}
\item If $G|_{\partial V}=0$ we choose $G|_V =0$ and then
$$
\Phi(\boldsymbol{x}) = \int \text{d}^3x' G(\boldsymbol{x}, \boldsymbol{x}')\rho(\boldsymbol{x}')
$$
\item If $\boldsymbol{\nabla}G|_{\partial V}$ is known we put $\boldsymbol{\nabla} G |_{\partial V}=0$ EXCEPTED if $\partial V$ is a closed surface; in this case we choose
$$
\frac{\partial G}{\partial n'}(\boldsymbol{x}, \boldsymbol{x}')= -\frac{1}{\text{surface}}
$$
Then
$$
\Phi(\boldsymbol{x}) = \left\langle \Phi \right\rangle _S + \frac{1}{\epsilon_0}\int \text{d}^3x' G(\boldsymbol{x}, \boldsymbol{x}')\rho(\boldsymbol{x}')+\int_{\partial V} \text{d}S \text{ } G(\boldsymbol{x}, \boldsymbol{x}')\boldsymbol{\nabla}\Phi
$$
$$
\left\langle \Phi \right\rangle _S = \frac{1}{S}\int_{\partial V} \text{d}S \text{ } \Phi(\boldsymbol{x})
$$

\end{enumerate}