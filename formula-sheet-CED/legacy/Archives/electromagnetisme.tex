%\graysec{Electromagnétisme}
% \squishlist
% \item $\vec F = q ( \vec E + {v}\cross{B})$
%  \item charge élémentaire: $|e| = 1.6 \cdot 10^{-19} \quad [C] = [As]$
%  \item densité volumique de charge: $\rho = \frac{\ud q}{\ud \omega} \quad [\frac{C}{m^3}]$
%  \item densité superficielle de charge: $\rho_s = \frac{\ud q}{\ud \sigma} \quad %[\frac{C}{m^2}]$
%  \item permittivité du vide: $\varepsilon_0 = 8.854 \cdot 10^{-12} \ [\frac{C^2}{m^2 N}] = %[\frac{F}{m}]$
%  \item $\emc 1 = 10^{-7} c^2 = 9.0 \cdot 10^9$
% \squishend
 
%\graysec{Electromagnétisme}
 \squishlist
  \item $\vec F = q ( \vec E + \cross{v}{B})$, Epot = $qV$
  \item charge élémentaire: $|e| = 1.6 \cdot 10^{-19} \quad [C] = [As]$
  \item N Avogadro: $6.022\cdot10^{23}$ mol$^{-1}$.
  \item densité volumique de charge: $\rho = \frac{\ud q}{\ud \omega} \quad [\frac{C}{m^3}]$
  \item densité superficielle de charge: $\rho_s = \frac{\ud q}{\ud \sigma} \quad [\frac{C}{m^2}]$
  \item permittivité du vide: $\varepsilon_0 = 8.854 \cdot 10^{-12} \ [\frac{C^2}{m^2 N}] = [\frac{F}{m}]$
  \item $\emc 1 = 10^{-7} c^2 = 9.0 \cdot 10^9$
 \squishend
 
\graypar{Electrostatique}
  \squishlist
     \item Force de Coulomb: $\frac{1}{4\pi\epsilon_0} \frac{q_1 q_2}{r^2}\cdot \hat{\vec{e_r}}$
     \item Champ électrique: $\mathbf{E(x)} = \frac{1}{4\pi\epsilon_0} \int_{\Omega_0} \frac{\rho(\mathbf{x_0})d\omega_0}{r^2}$ 
  \squishend
\textbf{Loi de Gauss}
 \squishlist
  \item $\int_{\Sigma} \vec{E}\cdot d\vec{\sigma} = \frac{Q}{\epsilon_0}$
  \item local: div$\vec{E}(\vec{x}) = \rho(\vec{x})/\epsilon_0$
  \item distribution continue des charges: $\nabla\cdot \vec{E(x)}= \frac{\rho(\vec{x})}{\epsilon_0}$
  \item distribution volumique:
    \squishlist
     \item $\vec{E}(r)=\emc{Q} \frac{1}{r^2} $, $V(r)=\emc{Q}\frac{1}{r}$ si $r\geqslant R$ 
     \item $\vec{E}(r)=\emc{Q} \frac{r}{R^3}$, $V(r) = Q\frac{3R^2-r^2}{8\pi\epsilon_0 R^3}$ si $r<R$
    \squishend
  \item distribution surfacique:
   \squishlist
    \item $\vec{E}(r)=\emc{Q} \frac{1}{r^2} $, $V(r)=\emc{Q}\frac{1}{r}$ si $r\geqslant R$
    \item $\vec{E}(r)=0$, $V(r)=\emc{Q} \frac{1}{R}$ si $r<R$
   \squishend
    \item champ de déplacement électrique: $\vec{D}=\epsilon_0 \vec{E}$
    \item loi de circulation dans le vide: $\oint \vec{E}\cdot \vec{dl} = 0$, local $rot\vec{E}=0$	
    \item potentiel: $V(P)-V(A)= -\int_A^P \vec{E}\cdot\vec{dl}$ [V]
    \item $\vec{E}=-\vec{\nabla} V$
    \item potentiel d'une charge ponctuelle: $V(r)= \emc{1}\frac{q}{r}$
    \item équation de poisson: $\nabla^2 V(\vec{x}) = -\frac{\rho(\vec{x})}{\epsilon_0}$ (laplacien)
    \item équation de Laplace: si $\rho(\vec{x}) = 0, \ \nabla^2 V(\vec{x}) = 0$
 \squishend
 
 

\graypar{Champ électrique et conducteurs}
	\squishlist
		\item densité de charge et courant nuls dans un conducteur.
		\item $\vec{E}$ de surface: $E=\frac{\rho_s}{\epsilon_0}$
		\item force agissant sur la charge de surface: $\vec{dF}=pd\sigma \hat{\vec{n}}$
		\item pression électrostatique: $p = \rho_s^2/2\epsilon_0 = \epsilon_0 E^2/2$
	\squishend
\textbf{Capacité}
	\squishlist
		\item sphère conductrice de rayon $R$: $C=4\pi\epsilon_0 R$
		\item $C=\frac{Q_1}{V1-V2}$ (2 sphères concentriques)
		\item 2 plaques: $C=\frac{Q_1}{V1-V2} = \epsilon_0 S/d$
		\item Energie électrostatique: $E_C = \int_0^Q V(q)dq = 1/2 C V^2$
		\item Densité d'énergie: $e_c= E_C/Vol = 1/2 \epsilon_0 E^2$ [J/m$^3$]
	\squishend

\graypar{Champ électrique dans la matière diélectrique}
  \squishlist
   \item \textbf{Diplôle diélectrique}:
    \squishlist
     \item moment dipolaire él.: $\vec p = q \vec a \ [Cm]$, $\vec a: -q \rightarrow +q$
     \item $V(r) = \emc 1 \left( \frac{q}{r_1} - \frac{q}{r_2} \right)
            = \emc 1 \frac{\vec p \cdot \univec e_r}{r^2}$
     \item $\vec E = \emc 1 \frac{3 \univec e_r (\univec e_r \cdot \vec p) - \vec p}{r^3}$
     \item molécules polaires: $\vec C = \cross{p}{E}, \vec F = (\vec p \cdot \ugrad) \vec E$ \\
           $E_\textrm{pot} = - \vec p \cdot \vec E$
    \squishend

   \item \textbf{Polarisation, susceptibilité électrique}
    \squishlist
     \item polarisation: $\vec P (\vec x) = n \vec p \ [\frac{C}{m^2}]$
     \item en générale: $\vec P(\vec x) = \chi_e \varepsilon_0 \vec E$,
           $\chi_e = \varepsilon_r - 1$
     \item ferroélectriques: $\chi_e \approx 1000$, pol. sans $\vec E$
     \item piézoélectriques: déformation mécanique $\rightarrow$ pol.
     \item pyroélectriques: échauffement $\rightarrow$ pol.
    \squishend

   \item diél. ds un condensateur: $\vec E = \vec E_0 + \vec E'$,
         $\vec E' = - \frac{\vec P}{\varepsilon_0}$ \\
         $\rho_S = \pm P$
   \item Champ él. microsc.: $\vec E(\vec x) = \frac{1}{\Delta V} \int_{\Delta V}
                              \vec E_\mu (\vec x + \vec \eta) \ud^3 \eta$
   \item première loi: $\int_\Sigma (\varepsilon_0 \vec E + \vec P) \cdot \vec{\ud \sigma}= \int_\Omega \rho_l(\vec{x}) d\omega = Q_l$
   \item \textbf{Champ de déplacement}: $\vec D = \varepsilon_0 \vec E + \vec P$
   \item pol. prop. au ch. el.: $\vec F \approx \frac{\varepsilon_0 \chi_e}{2} \ugrad E^2 \cdot \textrm{vol.}$
   \item $\vec E, V, \vec F$: caluculer avec $\varepsilon_0 \rightarrow \varepsilon_0 \varepsilon_r$



   \item Milieu polarisé uniformément: ($\rho = - \udiv \vec P$)
    \squishlist
     \item $V(\vec x) = \frac{1}{\rho_0} \vec P \cdot \vec E_{\textrm{aux}}$
     \item $\vec E_{\textrm{aux}} = \emc 1 \int_\Omega \rho_0 \frac{\vec x - \vec x_0}{\| \vec x - \vec x_0 \|^3} \ud \omega$
     \item $\vec E = - \frac{1}{\rho_0} ( \vec P \cdot \ugrad) \vec E_{\textrm{aux}}$
    \squishend

   \item Sphère uniformément polarisée:
    \squishlist
     \item $V_{\textrm{int}} = \frac{\vec P \cdot \vec r}{3 \varepsilon_0},
            \quad V_{\textrm{ext}} = \frac{R^3}{3 \varepsilon_0 r^3} \vec P \cdot \vec r$
     \item $\vec E_{\textrm{int}} = - \frac{\vec P}{3 \varepsilon_0}, \quad
            \vec E_{\textrm{ext}} = \frac{R^3}{3 \varepsilon_0} \frac{3 ( \vec P \cdot \univec e_r) \univec e_r - \vec P}{r^3}$
     \item $\vec D_{\textrm{int}} = \frac{2 \vec P}{3}, \quad \vec D_{\textrm{ext}} = \varepsilon_0 \vec E_{\textrm{ext}}$
    \squishend

   \item Sphère diélectrique dans un ch. él. ext.:
    \squishlist
     \item $\vec E_{\textrm{int}} = \frac{3 \vec E_0}{3 + \chi_e},
            \vec D_{\textrm{int}} = \frac{3 (1 + \chi_e)}{3 + \chi_e} \varepsilon_0 \vec E_0, \\
            \vec P_{\textrm{int}} = \frac{3 \chi_e}{3 + \chi_e} \varepsilon_0 \vec E_0$
     \item $\vec E_{\textrm{ext}} = \vec E_0 + \frac{R^3}{3 \varepsilon_0}
              \frac{3 ( \vec P_{\textrm{int}} \cdot \univec e_r) \univec e_r - \vec P_{\textrm{int}}}{r^3}, \\
            \vec E_{\textrm{ext}} = \varepsilon_0 \vec E_{\textrm{ext}}, \vec P_{\textrm{ext}} = 0$
     \item $\chi_e = 0$: vide, $chi_e \rightarrow \infty$: conducteur
    \squishend
   \squishend

\graysec{Courant électrique stationnaire}
matériaux conducteurs
	\squishlist
		\item densité de courant: $\vec{J(x},t) = \rho \vec{v}$ [Cs$^{-1}$m$^{-2}$] = [A/m$^2$]
		\item courant électrique: $I=\int\vec{J}\cdot \vec{d\sigma} = dQ/dt$ [A]
		\item loi d'Ohm: $V_A-V_B = RI$, local $\vec{E}=\rho \vec{J}$
		\item $R=\rho L/S$ ($\rho$ la resistivité)
		\item Resistivité en fct de $T$: $\rho(T)=\rho(T_0)[1+\alpha(T-T_0]$
		\item Effet Joule: $dW_{A-B}=dq(V_A-V_B) = Idt(V_A-V_B)=IdtV$
		\item $P=dW/dt = RI^2$ [W]
		\item conductivité: $\sigma = 1/\rho$, $\vec{E} = \rho \vec{J}$ ($\rho$ la resistivité)
		\item série: $R=R_1 + \dots +R_n$, parralèle: $1/R= 1/R_1+\dots+1/R_n$
		\item loi de Kirchoff: $\sum_j I_j=0$,
	\squishend
 
 \graypar{Magnétostatique}
  \squishlist
   \item \textbf{Force de Lorentz}: $\vec F = q(\vec E + {v}\cross{B})$, (v $e^-$)
    \squishlist
     \item force magnétique: $\vec F = q {v}\cross{B}$, $\vec{\overline B}_{\textrm{terre}} = 2 \cdot 10^{-5} T$ \\
           $\vec F_{\textrm{mag}}$ force passive $\rightarrow$ pas de changement de $E^{cin}$
     \item $\vec B$ uniforme et $\perp \vec v_0$: trajectoire cercle de \\
           $R = \frac{mv}{qB} = \frac{\sqrt{2mE^{\textrm{cin}}}}{qB}$, $f_{\textrm{cyclo}} = \frac{1}{2 \pi} \frac{q}{m} B$
    % \item $\vec B$ uniforme et \emph{ne pas} $\perp \vec v_0$: traj. hélicoïdale
    % \item bouteille magnétique: Sym. axiale, $B_{\textrm{extr.}} > B_{\textrm{centre}}$
     \item courant dans un conducteur: $\vec F = I \int_L {\ud l}\cross{B}$
     \item spire rect.: $\vec C = {m}\cross{B}$, moment mag.: $\vec m = IS \univec e_n$ \\
           $\Delta E_{\alpha_0 \rightarrow \alpha}^{\textrm{pot}} = - \|\vec m\| B (\cos \alpha - \cos \alpha_0)$
     \item En. potentielle: $E_{pot} = -\vec{m\cdot B}$
     \item effet Hall: $V_H = \frac{I B}{n q b}$, $n$ dens. de charges libres
    \squishend

   \item \textbf{Champ d'ind. mag. créé par des charges en mvt}
    \squishlist
     \item \textbf{loi de Biot-Savart}: $\vec B(\vec x) = \frac{\mu_0}{4 \pi} I \oint \frac{\univec e_\tau \cross \univec e_r}{r^2} \ud l$
     %\item perméabilité mag. du vide: $\mu_0 = 4 \pi \cdot 10^{-7} \ [\frac{Vs}{mA}]$
     \item fil rectiligne ($l = \infty$): $\vec B = \frac{\mu_0 I}{2 \pi R} \univec e_\phi$
     \item spire (rayon $a$), sur l'axe: $\vec B = \frac{\mu_0 I a^2}{2(a^2 + z^2)^{3/2}} \univec e_z$
     \item \emph{dipôle magnétique}: loin de la spire $\rightarrow B(z) = \frac{\mu_0 m}{2 \pi z^3}$ 
     %\\      $m$ = moment dipolaire magnétique
     %\item bobines de Helmholtz: 2 spires, $d = a$, ch. homogène
     \item solénoïde: $B = \mu_0 I n$, $n = \frac{\textrm{\# spires}}{\textrm{unité de long.}}$
    \squishend

   \item \textbf{Lois fondamentales de la magnétostatique}
    \squishlist
     \item \emph{flux magnétique}: $\phi_{\vec B} = \int_{\Sigma_f} \scalar{B}{\ud \sigma} = 0 \ [\textrm{Weber}]$ \\
           forme locale: $\udiv \vec B = 0$
     \item \emph{loi d'Ampère}: $\oint_L \scalar{B}{\ud l} = \mu_0 \sum_j I_j = \mu_0 \int_\Sigma \scalar{J}{\ud \sigma}$ \\
           forme locale: $\urot \vec B = \mu_0 \vec J$
    \squishend

   \item \textbf{Potentiel vecteur}: $\vec B(\vec x) = \urot \vec A(\vec x)$
    \squishlist
     \item $\vec A' = \vec A + \ugrad \phi \Rightarrow \urot \vec A' = \urot \vec A$
     \item choix de $\vec A \Leftrightarrow$ choix de jauge
     \item $\ugrad (\udiv \vec A) - \ulap \vec A = \mu_0 \vec J$ \\
           jauge de Coulomb ($\udiv \vec A = 0$): $\ulap \vec A = - \mu_0 \vec J$
     \item ext. d'un fil: $\vec A(\vec x) = \frac{\mu_0}{4 \pi} \int_L \frac{I}{\| \vec x - \vec x_0 \|} \vec{\ud l}$
    \squishend

   \item \textbf{Potentiel vecteur + dipôle magnétique}
    \squishlist
     \item $\vec A = \frac{\mu_0}{4 \pi} \left( \frac{\vec m \wedge \univec e_r}{r^2} \right) \univec e_\phi$,
           $\vec m = I \pi R^2 \univec e_z$
     \item $\vec B = \frac{\mu_0}{4 \pi} \frac{3 ( \vec m \cdot \univec e_r) \univec e_r - \vec m}{r^3}$
    \squishend
  \squishend

 \graypar{Aimantation de la matière}
  \squishlist
   \item \textbf{Moment magnétique et aimantation}
    \squishlist
     \item courant particulaire de l'$e^-$: $I = \frac{e}{2 \pi a} v \ [A]$ \\
           moment dipôlaire magnétique: $\vec m = \frac{q}{2} {a}\cross{v} \ [Am^2]$
     \item aimantation: $\vec M (\vec x) = \frac{1}{\Omega} \sum_i \vec m_i [\frac{A}{m}]$,
           \ $M = I_m$ \\
           cylindre long et mince: $\vec B = \mu_0 \vec M$
     \item dipôle mag., translation: $\vec F = -\ugrad E_{pot} = \ugrad (\vec{m\cdot B_{ext}}) = Jac \vec{B}^t_{ext} \vec{m}$
    \squishend

   \item \textbf{Champ magnétique}: $\vec H = \frac{1}{\mu_0} \vec B - \vec M \ [\frac{A}{m}]$
    \squishlist
     \item $\oint \scalar{H}{\ud l} = \sum I_l \qquad \urot \vec H = \vec J_l$
     \item $B_\perp$ et $H_\parallel$ sont continues au passage de 2 matériaux
    \squishend

   \item \textbf{Susceptibilité magnétique et perméabilité relative}
    \squishlist
     \item $\vec M = \chi_m \vec H$, $\chi_m$ = susceptibilité magnétique
     \item $\vec B = \mu \vec H$, $\mu = \mu_r \mu_0 = (1+\chi_m) \mu_0$
     %\\  $\mu$ = perméabilité du milieu, $\mu_r$ = perméabilité relative
    \squishend

   \item diamagnétiques: $\chi_m < 0$, $\mu_r \approx 1$, $M = 0$ sans $B_\textrm{ext}$
   \item paramagnétiques: $\chi_m > 0$, $\mu_r \approx 1$, $M = 0$ sans $B_\textrm{ext}$
   \item ferromagnétiques: $\chi_m \gg 0, \mu_r \gg 1$

   \item pot. vect. + ch. d'ind. mag. + milieu aimanté: \\
         $\vec B = \urot \vec A = \frac{\mu_0 \varepsilon_0}{\rho_0}
          [\vec M \udiv \vec E_\textrm{aux} - Jac \vec{E_{aux}} \vec{M}]$

   \item champs d'une sphère uniformement aimanté: \\
         $\vec B_\textrm{int} = \frac{2}{3} \mu_0 \vec M$,
         $\vec B_\textrm{ext} = \frac{\mu_0 R^3}{3} \frac{3 (\vec M \cdot \univec e_r) \univec e_r - \vec M}{r^3}$ \\
         $\vec H_\textrm{int} = - \frac{\vec M}{3}$, $\vec H_\textrm{ext} = \frac{B_\textrm{ext}}{\mu_0}$

   \item sphère mag. + champs mag.: \\
         intérieur: $\vec B_i = \frac{3 (1+\chi_m)\vec B_0}{3 + \chi_m}, \\
                \vec M = \frac{3 \chi_m}{3 + \chi_m} \frac{\vec B_0}{\mu_0},
                \vec H_i = \frac{3}{3 + \chi_m} \frac{\vec B_0}{\mu_0}$ \\
         extérieur: $\vec H_e = \frac{\vec B_e}{\mu_0}$, \\
                    $\vec B_e = \vec B_0 + \frac{\chi_m}{3+\chi_m}
                     \frac{R^3}{r^3}
                     [3 (\vec B_0 \cdot \univec e_r) \univec e_r - \vec B_0]$
  \squishend

 \graypar{Induction électromagnétique}
  \squishlist
   \item loi de Faraday: $\varepsilon = - \der{t}{\phi_B}$
   \item $\varepsilon = \oint_L (\vec{v_{obj}}\wedge\vec{B})\cdot d\vec{l}$, $u+\varepsilon = Ri$, $d\varepsilon = -Budl$
   \item \textbf{Loi d'induction}: $\oint_L \scalar{E}{\ud l}
         = - \der{t}{} \int_\Sigma \scalar{B}{\ud \sigma}$ \\
         forme locale: $\urot \vec E(\vec x,t) = - \D{t}{\vec B(\vec x,t)}$
   \item \textbf{self-induction/induction mutuelle}:
    \squishlist
     \item coef. de self-ind.: $L_1 = \frac{\phi_{11}}{i_1} \ $ [Henry]
     \item coef. d'ind. mutuelle: $M = \frac{\phi_{12}}{i_1} \ [H]$
    \squishend
   \item Energie méc. d'un self: $E_m = \half L I_0^2$
   \item transformateur: $\frac{v_2(t)}{v_1(t)} = \frac{M}{L} = \frac{N_2}{N_1}
         \quad ( = \frac{i_1(t)}{i_2(t)} )$
  \squishend

% \graypar{Circuits électriques en régime non-stationnaire}
%  \squishlist
%   \item \textbf{régime transitoire}:
%    \squishlist
%     \item condensateur: $v_C(t) = \varepsilon \left( 1 - \exp(-\frac{t}{RC}) \right)$ \\
%           $i_C(t) = \frac{\varepsilon}{R} \exp (-\frac{t}{RC})$,
%           cste de temps: $\tau = RC$
%     \item self: $i_L(t) = \frac{\varepsilon}{R} (1 - \exp(-\frac{R}{L}t))$, \\
%           $v_L(t) = \varepsilon \exp(-\frac{R}{L}t)$, $\tau = \frac{L}{R}$
%    \squishend
%
%   \item \textbf{régime sinusoïdal}:
%    \squishlist
%     \item $v(t) = V \cos(\omega t), i(t) = I \cos (\omega t - \phi)$
%     \item $i(t) = \textrm{Re}(I e^{j(\omega t - \phi)}) \rightarrow$
%           diagramme des phaseurs
%     \item $V_\textrm{eff} = \frac{V}{\sqrt{2}} =
%            \sqrt{\frac{1}{T} \int_0^T v^2(t) \ud t}$
%     \item résistance: $v(t) = R i(t) \Rightarrow i = \frac{V}{R} \cos \omega t$
%     \item condensateur: $i(t) = C \omega V \cos ( \omega t + \frac{\pi}{2})$
%     \item self: $i(t) = \frac{V}{\omega L} \cos ( \omega t - \frac{\pi}{2})$
%     \item RCL série: $i(t) = I \cos (\omega t - \phi)$ \\
%           $\phi = \arctan \left(\frac{\omega L - \frac{1}{\omega C}}{R} \right)$,
%           $I = \frac{V}{\sqrt{R^2 + (\omega L - \frac{1}{\omega C})^2)}}$
%     \item impédance: $Z = |Z| e^{j \delta}$ \\
%           $Z_R = R$, $Z_C = \frac{1}{j \omega C}$, $Z_I = j \omega L$ \\
%           $|Z| = \sqrt{R^2 + (\omega L - \frac{1}{\omega C})^2)}$,
%           $\tan \delta = \frac{\omega L - \frac{1}{\omega C}}{R}$
%     \item puissance active: $P = \frac{V I}{2} \cos \delta$
%    \squishend
%
%   \item filtres électriques:
%    \squishlist
%     \item fréquence de coupure: $\omega_0 \tq P = \frac{1}{2} P_\textrm{max}$
%     \item low-pass (L-R): $\frac{\tilde v_2}{\tilde v_1} = \frac{1}{1 + j\omega / \omega_0}$,
%           $\omega_0 = \frac{R}{L}$
%     \item high-pass (C-R): $\frac{\tilde v_2}{\tilde v_1} = \frac{1}{1 - j\omega_0 / \omega}$,
%           $\omega_0 = \frac{1}{RC}$
%    \squishend
%  \squishend

 \graypar{Equations de Maxwell}
  \begin{tabular}{|@{\ }c@{\ }|@{\ }c@{\ }|}
   \hline
   \textbf{Forme locale} & \textbf{Forme intégrale} \\
   \hline \hline
   $\udiv \vec D = \rho$ & $\int_{\Sigma_{\textrm{fermé}}} \vec D \cdot
   \ud \vec \sigma = \int_\Omega \rho(\vec x) \ud \omega$ \\
   \hline
   $\urot \vec E = -\D{t}{\vec B}$ & $\oint \vec E \cdot \ud \vec l = -
   \frac{\ud}{\ud t} \int_\Sigma \vec B \cdot \ud \vec \sigma$ \\
   \hline
   $\udiv \vec B = 0$ & $\int_{\Sigma_{\textrm{fermé}}} \vec B \cdot
   \ud \vec \sigma = 0$ \\
   \hline
   $\urot \vec H = \vec J + \D{t}{\vec D}$ & $\oint \vec H \cdot \ud
   \vec l = \sum i + \frac{\ud}{\ud t} \int_\Sigma \vec D \cdot \ud
   \vec \sigma$ \\
   \hline
  \end{tabular}

  \squishlist
   \item $\D{t}{\vec D}$: densité de courant de déplacement
   \item $\vec D = \varepsilon_0 \vec E + \vec P$, 
         linéaire, isotrope: $\vec D = \varepsilon_r \varepsilon_0 \vec E = \varepsilon \vec E$
   \item $\vec B = \mu_0 \vec H + \mu_0 \vec M$, 
         lin+isotr: $\vec B = \mu_0 \mu_r \vec H = \mu \vec H$
   \item milieu conducteur: $\vec J = \sigma \vec E$, $\sigma = 1/\rho$ 
   \item $\vec B (\vec x, t) = \urot \vec A(\vec x,t), \vec E = - \ugrad V - \D{t}{\vec A}$
  \squishend

 \graypar{Énergie électromagnétique}
  \squishlist
   \item Travail de la force de Lorentz: $\frac{\delta W}{\ud t \cdot \ud \omega} = \scalar{E}{J}$
   \item $\D{t}{} \int_\Omega u_\textrm{EM} \ud \omega = - \int_{\Sigma_f} \scalar{S}{\ud \sigma}
           - \int_\Omega \scalar{E}{J} \ud \omega$, \\
         $\vec S (\vec x,t)$: courant d'énergie électromagnétique $[\frac{W}{m^2}]$
   \item $\D{t}{u_\textrm{EM}} = - \udiv \vec S + \sigma_\textrm{EM}$ \\
         $\sigma_\textrm{EM} > 0$: prod. d'énergie, $\sigma_\textrm{EM} < 0$: perte d'énergie
   \item \textbf{vecteur de Poynting}: $\vec S = {E}\cross{H}$ \\
         $\D{t}{u_\textrm{EM}} = \vec H \cdot \D{t}{\vec B} + \vec E \cdot \D{t}{\vec D}$
   \item milieux linéaires: $u_\textrm{EM} = \half \frac{B^2}{\mu_0 \mu_r} + \half \varepsilon_0 \varepsilon_r E^2$
  \squishend