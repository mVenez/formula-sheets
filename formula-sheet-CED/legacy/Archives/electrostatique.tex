\graysec{Electromagnétisme}
 \squishlist
  \item $\vec F = q ( \vec E + \cross{v}{B})$, Epot = $qV$
  \item charge élémentaire: $|e| = 1.6 \cdot 10^{-19} \quad [C] = [As]$
  \item N Avogadro: $6.022\cdot10^{23}$ mol$^{-1}$.
  \item densité volumique de charge: $\rho = \frac{\ud q}{\ud \omega} \quad [\frac{C}{m^3}]$
  \item densité superficielle de charge: $\rho_s = \frac{\ud q}{\ud \sigma} \quad [\frac{C}{m^2}]$
  \item permittivité du vide: $\varepsilon_0 = 8.854 \cdot 10^{-12} \ [\frac{C^2}{m^2 N}] = [\frac{F}{m}]$
  \item $\emc 1 = 10^{-7} c^2 = 9.0 \cdot 10^9$
 \squishend
 
\graypar{Electrostatique}
  \squishlist
     \item Force de Coulomb: $\frac{1}{4\pi\epsilon_0} \frac{q_1 q_2}{r^2}\cdot \hat{\vec{e_r}}$
     \item Champ électrique: $\mathbf{E(x)} = \frac{1}{4\pi\epsilon_0} \int_{\Omega_0} \frac{\rho(\mathbf{x_0})d\omega_0}{r^2}$ 
  \squishend
\textbf{Loi de Gauss}
 \squishlist
  \item $\int_{\Sigma} \vec{E}\cdot d\vec{\sigma} = \frac{Q}{\epsilon_0}$
  \item local: div$\vec{E}(\vec{x}) = \rho(\vec{x})/\epsilon_0$
  \item distribution continue des charges: $\nabla\cdot \vec{E(x)}= \frac{\rho(\vec{x})}{\epsilon_0}$
  \item distribution volumique:
    \squishlist
     \item $\vec{E}(r)=\emc{Q} \frac{1}{r^2} $, $V(r)=\emc{Q}\frac{1}{r}$ si $r\geqslant R$ 
     \item $\vec{E}(r)=\emc{Q} \frac{r}{R^3}$, $V(r) = Q\frac{3R^2-r^2}{8\pi\epsilon_0 R^3}$ si $r<R$
    \squishend
  \item distribution surfacique:
   \squishlist
    \item $\vec{E}(r)=\emc{Q} \frac{1}{r^2} $, $V(r)=\emc{Q}\frac{1}{r}$ si $r\geqslant R$
    \item $\vec{E}(r)=0$, $V(r)=\emc{Q} \frac{1}{R}$ si $r<R$
   \squishend
    \item champ de déplacement électrique: $\vec{D}=\epsilon_0 \vec{E}$
    \item loi de circulation dans le vide: $\oint \vec{E}\cdot \vec{dl} = 0$, local $rot\vec{E}=0$	
    \item potentiel: $V(P)-V(A)= -\int_A^P \vec{E}\cdot\vec{dl}$ [V]
    \item $\vec{E}=-\vec{\nabla} V$
    \item potentiel d'une charge ponctuelle: $V(r)= \emc{1}\frac{q}{r}$
    \item équation de poisson: $\nabla^2 V(\vec{x}) = -\frac{\rho(\vec{x})}{\epsilon_0}$ (laplacien)
    \item équation de Laplace: si $\rho(\vec{x}) = 0, \ \nabla^2 V(\vec{x}) = 0$
 \squishend
 
 

\graypar{Champ électrique et conducteurs}
	\squishlist
		\item densité de charge et courant nuls dans un conducteur.
		\item $\vec{E}$ de surface: $E=\frac{\rho_s}{\epsilon_0}$
		\item force agissant sur la charge de surface: $\vec{dF}=pd\sigma \hat{\vec{n}}$
		\item pression électrostatique: $p = \rho_s^2/2\epsilon_0 = \epsilon_0 E^2/2$
	\squishend
\textbf{Capacité}
	\squishlist
		\item sphère conductrice de rayon $R$: $C=4\pi\epsilon_0 R$
		\item $C=\frac{Q_1}{V1-V2}$ (2 sphères concentriques)
		\item 2 plaques: $C=\frac{Q_1}{V1-V2} = \epsilon_0 S/d$
		\item Energie électrostatique: $E_C = \int_0^Q V(q)dq = 1/2 C V^2$
		\item Densité d'énergie: $e_c= E_C/Vol = 1/2 \epsilon_0 E^2$ [J/m$^3$]
	\squishend

\graypar{Champ électrique dans la matière diélectrique}
  \squishlist
   \item \textbf{Diplôle diélectrique}:
    \squishlist
     \item moment dipolaire él.: $\vec p = q \vec a \ [Cm]$, $\vec a: -q \rightarrow +q$
     \item $V(r) = \emc 1 \left( \frac{q}{r_1} - \frac{q}{r_2} \right)
            = \emc 1 \frac{\vec p \cdot \univec e_r}{r^2}$
     \item $\vec E = \emc 1 \frac{3 \univec e_r (\univec e_r \cdot \vec p) - \vec p}{r^3}$
     \item molécules polaires: $\vec C = \cross{p}{E}, \vec F = (\vec p \cdot \ugrad) \vec E$ \\
           $E_\textrm{pot} = - \vec p \cdot \vec E$
    \squishend

   \item \textbf{Polarisation, susceptibilité électrique}
    \squishlist
     \item polarisation: $\vec P (\vec x) = n \vec p \ [\frac{C}{m^2}]$
     \item en générale: $\vec P(\vec x) = \chi_e \varepsilon_0 \vec E$,
           $\chi_e = \varepsilon_r - 1$
     \item ferroélectriques: $\chi_e \approx 1000$, pol. sans $\vec E$
     \item piézoélectriques: déformation mécanique $\rightarrow$ pol.
     \item pyroélectriques: échauffement $\rightarrow$ pol.
    \squishend

   \item diél. ds un condensateur: $\vec E = \vec E_0 + \vec E'$,
         $\vec E' = - \frac{\vec P}{\varepsilon_0}$ \\
         $\rho_S = \pm P$
   \item Champ él. microsc.: $\vec E(\vec x) = \frac{1}{\Delta V} \int_{\Delta V}
                              \vec E_\mu (\vec x + \vec \eta) \ud^3 \eta$
   \item première loi: $\int_\Sigma (\varepsilon_0 \vec E + \vec P) \cdot \vec{\ud \sigma}= \int_\Omega \rho_l(\vec{x}) d\omega = Q_l$
   \item \textbf{Champ de déplacement}: $\vec D = \varepsilon_0 \vec E + \vec P$
   \item pol. prop. au ch. el.: $\vec F \approx \frac{\varepsilon_0 \chi_e}{2} \ugrad E^2 \cdot \textrm{vol.}$
   \item $\vec E, V, \vec F$: caluculer avec $\varepsilon_0 \rightarrow \varepsilon_0 \varepsilon_r$



   \item Milieu polarisé uniformément: ($\rho = - \udiv \vec P$)
    \squishlist
     \item $V(\vec x) = \frac{1}{\rho_0} \vec P \cdot \vec E_{\textrm{aux}}$
     \item $\vec E_{\textrm{aux}} = \emc 1 \int_\Omega \rho_0 \frac{\vec x - \vec x_0}{\| \vec x - \vec x_0 \|^3} \ud \omega$
     \item $\vec E = - \frac{1}{\rho_0} ( \vec P \cdot \ugrad) \vec E_{\textrm{aux}}$
    \squishend

   \item Sphère uniformément polarisée:
    \squishlist
     \item $V_{\textrm{int}} = \frac{\vec P \cdot \vec r}{3 \varepsilon_0},
            \quad V_{\textrm{ext}} = \frac{R^3}{3 \varepsilon_0 r^3} \vec P \cdot \vec r$
     \item $\vec E_{\textrm{int}} = - \frac{\vec P}{3 \varepsilon_0}, \quad
            \vec E_{\textrm{ext}} = \frac{R^3}{3 \varepsilon_0} \frac{3 ( \vec P \cdot \univec e_r) \univec e_r - \vec P}{r^3}$
     \item $\vec D_{\textrm{int}} = \frac{2 \vec P}{3}, \quad \vec D_{\textrm{ext}} = \varepsilon_0 \vec E_{\textrm{ext}}$
    \squishend

   \item Sphère diélectrique dans un ch. él. ext.:
    \squishlist
     \item $\vec E_{\textrm{int}} = \frac{3 \vec E_0}{3 + \chi_e},
            \vec D_{\textrm{int}} = \frac{3 (1 + \chi_e)}{3 + \chi_e} \varepsilon_0 \vec E_0, \\
            \vec P_{\textrm{int}} = \frac{3 \chi_e}{3 + \chi_e} \varepsilon_0 \vec E_0$
     \item $\vec E_{\textrm{ext}} = \vec E_0 + \frac{R^3}{3 \varepsilon_0}
              \frac{3 ( \vec P_{\textrm{int}} \cdot \univec e_r) \univec e_r - \vec P_{\textrm{int}}}{r^3}, \\
            \vec E_{\textrm{ext}} = \varepsilon_0 \vec E_{\textrm{ext}}, \vec P_{\textrm{ext}} = 0$
     \item $\chi_e = 0$: vide, $chi_e \rightarrow \infty$: conducteur
    \squishend
   \squishend

\graysec{Courant électrique stationnaire}
matériaux conducteurs
	\squishlist
		\item densité de courant: $\vec{J(x},t) = \rho \vec{v}$ [Cs$^{-1}$m$^{-2}$] = [A/m$^2$]
		\item courant électrique: $I=\int\vec{J}\cdot \vec{d\sigma} = dQ/dt$ [A]
		\item loi d'Ohm: $V_A-V_B = RI$, local $\vec{E}=\rho \vec{J}$
		\item $R=\rho L/S$ ($\rho$ la resistivité)
		\item Resistivité en fct de $T$: $\rho(T)=\rho(T_0)[1+\alpha(T-T_0]$
		\item Effet Joule: $dW_{A-B}=dq(V_A-V_B) = Idt(V_A-V_B)=IdtV$
		\item $P=dW/dt = RI^2$ [W]
		\item conductivité: $\sigma = 1/\rho$, $\vec{E} = \rho \vec{J}$ ($\rho$ la resistivité)
		\item série: $R=R_1 + \dots +R_n$, parralèle: $1/R= 1/R_1+\dots+1/R_n$
		\item loi de Kirchoff: $\sum_j I_j=0$,
	\squishend