 \graypar{Modèle d'Ising et O(n)}
  \squishlist
   \item Spin $S_i=\pm 1$, notation des plus proches voisins : $< , >$
   \item Le modèle $O(n)$ a des spins qui sont des vecteur de dim. $n$
   \item $E(c)=-J\sum_{<i,j>}S_iS_j -h\sum{S_i}$, $c=\{ S_i \} _{i=1,\dots,N}$
   \item $T=0 \to z=\sum_c \exp{(-\beta E(c))} = \sum_E \exp{(-\beta E)}\Omega{(E)}$
   \item $S(E)=k_B\ln{(\Omega{(E)})} \to \Omega{(E)}=\exp{(S(E)/k_B)}$
   \item $Z=\sum_E \exp{(-\beta (E-TS(E)))}=\sum_E \exp{(-\beta F)}$
   \item $<M(h)> =Nm(h)=\frac{1}{\beta Z}\frac{\partial Z}{\partial h}$
   \item $\vec{m}=\frac{\vec{M}}{N}=\frac{\sum \vec{S_i}}{N}$
  \squishend
 \graypar{Gaz sur réseau}
  \squishlist
   \item $n_i=1$ ou $0$, $N$ sites
   \item Grand canonique, potentiel chimique $\mu$
   \item Interaction favorable $A$ si $i$ et $j$ sont voisins
   \item $E=\mu\sum_{i=1}^N n_i - A\sum_{<i,j>}n_in_j$
   \item Changement de variable : ($z=\#$ voisins)
   \subitem $S_i=2n_i-1$
   \subitem $n_i=\frac{S_i+1}{2}$
   \item $E=cste-\frac{A}{4}\sum_{<i,j>}S_iS_j-\left[\frac{Az}{4}-\frac{\mu}{2}\right]\sum_i S_i$
   \item ... Quelque chose avec la magnétisation ... ???
  \squishend
  \graypar{Exposants critiques}
  \squishlist %J'aimerais mettre ici un schéma (cf photo)
   \item[] \tikz[baseline=-0.5ex]{
            \draw [domain=-1.5:1.5] plot(\x,{tanh(\x)});
            \draw [domain=-5:3] plot(\x/5,{tanh(\x)});
            \draw [domain=-.003:.0015] plot(\x/1000,{tanh(\x)});
            \node (node1) at (1.2,.4) {$T>T_c$};
            \node [rotate=15] (node1) at (0.7,.85) {$T=T_c$};
            
            \draw[->] (-1.5,0) -- (1.5,0);
            \draw (1.5,0) node[right] {$h$};
            \draw [->] (0,-1) -- (0,1);
            \draw (0,1) node[above] {$m$};
        }
    \item Susceptibilité $x=\frac{\partial m}{\partial h}\vert_{h=0} \propto (T-T_c)^\gamma$
    \item Chaleur spécifique $C(T)=C_{reg}(T)+B(T)|T-T_c|^{-\alpha}$
  \squishend
 \graypar{Corrélations spatiales}
  \squishlist
   \item $G_{ij}=<S_i\ S_j>-<S_i><S_j>$
   \subitem $T>T_c \to <S_i>=0$
   \subitem $T<T_c \to <S_i>=m$
   \item $R_{ij} : $ distance entre $i$ et $j$
   \subitem $ R_{ij}$ très grand $\implies <S_i\ S_j>=<S_i><S_j>$
   \subitem $\implies G_{ij}(R\to \infty) \to 0$
   \item Longueur caractéristique $\xi\propto |T-T_c|^{-\nu}$
   \item $G_{ij}(r) \propto \exp{(-r/\xi)} \Leftarrow \boxed{T\neq T_c}$
   \item $G_{ij}(R) \propto 1/R^{d-2+\eta} \Leftarrow \boxed{T= T_c}$
   
  \squishend
 \graypar{Théorème de fluctuation dissipation}
  \squishlist
   \item[] \textbf{Lien entre réponse linéaire et fluctuation}
   \item $E=\frac{1}{2} k x^2, <E>=\frac{1}{2} k <x^2> = \frac{1}{2} k_B T$
   \item $<x^2>=\frac{k_BT}{k}, k<x>=f, <x>=\frac{f}{k}=\chi f$
   \item $\boxed{\underbrace{<x^2>}_{\text{fluctuation}}=k_BT-\underbrace{\chi}_{\text{rep. lin.}}}$
  \squishend