% Matteo Veneziano -- Quantum Physics II Formula Sheet

\graypar{General single qubit state}
\begin{squishlist}
    \item $\ket{\psi} = \alpha \ket{0} + \beta \ket{1}$, or $\ket{\psi} = \cos(\theta/2)\ket{0} + e^{i\phi} \sin(\theta/2)\ket{1}$, $|\alpha|^2 + |\beta|^2 = 1$
    \item $\ket{\psi}$ is an eigenstate of $\vec{\sigma}\cdot \vec{n}$ with eigenvalue 1
\end{squishlist}

\graypar{Evolution}
\begin{squishlist}
    \item $i \D{t}{\ket{\psi(t)}} = H \ket{\psi(t)}$ (Schrödinger)
    \item if $H$ time-independent then $\ket{\psi(t)} = U(t) \ket{\psi(0)}$,\quad $U(t) = e^{-iHt}$
\end{squishlist}

\graypar{Pauli matrices}
\begin{squishlist}
    \item $\sigma_0 = \begin{pmatrix} 1 & 0\\ 0 & 1 \end{pmatrix},
        \sigma_x = \begin{pmatrix} 0 & 1 \\ 1 & 0 \end{pmatrix}, 
        \sigma_y=\begin{pmatrix} 0 & -i\\ i & 0\end{pmatrix} , 
        \sigma_z = \begin{pmatrix} 1 & 0 \\ 0 & -1\end{pmatrix}$

    \item $\Tr[\sigma_0] = 2, \Tr[\sigma_i] = 0$ for $i=1,2,3$ \hfill $S_i = \frac{\hbar}{2} \sigma_i$
    \item $\sigma_i \sigma_j = \delta_{ij} \id + i \epsilon_{ijk} \sigma_k$ \quad $[\sigma_i, \sigma_j] = 2i \epsilon_{ijk} \sigma_k$ \squishsep $V = \vec{v}\cdot \vec{\sigma} \Rightarrow V^2 = \id$
    \item $\{\sigma_i, \sigma_j \} = \sigma_i \sigma_J + \sigma_j\sigma_i = 2 \delta_{ij} \id$
    \item Paulis form an orthonormal basis with $\Tr[\sigma_i \sigma_j] = 2\delta_{ij}$
    \item The Pauli $\alpha$ operator, $\alpha \in \{x,y,z\}$, rotates the state by $\pi$ around the $\alpha$-axis.

    \item Paulis as generators: any single qubit hamiltonian can be written as $H = \omega \vec{n}\cdot \vec{\sigma}$ \\
        $U = \exp(-i\omega \vec{\sigma}\cdot \vec{n} t) = \cos(\omega t) \id - i \sin(\omega t) \vec{n}\cdot\vec{\sigma}$ \\
        Causes a qubit state to rotate around $\vec{n}$ at a rate $2\omega t$.
\end{squishlist}

\graypar{Observables}
\begin{squishlist}
    \item $M = \sum_k \lambda_k \ket{\lambda_k} \bra{\lambda_k}$ Hermitian
    \item $\braket{M} = \braket{\psi|M|\psi} = \sum_k \lambda_k P_k$, $P_k = |\braket{\lambda_k | \psi}|^2$
    \item $\Pi_k = \ket{\lambda_k}\bra{\lambda_k}$ Projector $\Rightarrow P_k = \braket{\psi| \Pi_k | \psi}$, $\sum_k \Pi_k = 1$
    \item Measurement $\Rightarrow$ State collapses to $\frac{\Pi_k \ket{\psi_k}}{\sqrt{P_k}}$
\end{squishlist}

\graypar{Composite systems \quad $\H_{ABC\ldots} = \H_a \otimes \H_b \otimes \H_c \ldots$}
\begin{squishlist}
    \item The resulting space has dimension $n_A n_B n_C \ldots$
    \item Operators $T_{AB} \ket{\lambda_ij} = (T_a \otimes T_b) (\ket{\mu_i \otimes \nu_j}) = T_A \ket{\mu_i} \otimes \ket{\nu_k}$
    \item $[T_A \otimes \id_B, \id_A \otimes T_B] = 0$, \quad $\{ T_A \otimes \id_B, \id_A \otimes T_B\} = 2 (T_A \otimes T_B)$ \\
        $e^{A \otimes \id + \id \otimes B} = e^A \otimes e^B$

    \item \textbf{Global} measurem.\ $T = T_A \otimes T_B \rightarrow T \ket{T_i} = t_i \ket{T_i}, \quad \ket{\psi} = \sum_i \ket{T_i} \bra{T_i}\ket{\psi}$ 
    \item \textbf{Partial} measurement $T_A$: if $\kpsi = \sum_{ij} \ket{T_{A,i}} \otimes \ket{T_{B,j}} = \sum_i \ket{T_{A,i}} \otimes \ket{\phi_{B,i}}$
        then $P_i = \sum_j |c_{ij}|^2$ and system collapses to $\kpsi' \propto \ket{T_{A,i}} \otimes \ket{\phi_{B,i}}$

    \item $P_{\hat{O}}(\lambda | \psi) = \braket{\psi | \Pi^{A}_{\lambda} | \psi }$ is the \textbf{probability} of a measurement of operator $\hat{O}$ yielding its eigenvalue $\lambda$, with $A$ the subspace of the meas., $\Pi_{\lambda}^A = \ket{\lambda}\bra{\lambda}_A \otimes \id_B$ and
    $\ket{\lambda}$ the eigenket corresponding to eigenvalue $\lambda$

    For measurements on both subspaces use $\Pi_{\lambda} = \ket{\lambda}\bra{\lambda}_A \otimes \ket{\kappa}\bra{\kappa}_B$

    \item \textbf{Entangled} state: its coefficients cannot be written as the product of two independent coefficients. 
        \textbf{Separable} state: the global wave function can be written as the product of two wavefunctions corresponding to subsystems $A$ and $B$ (measures performed on one part do not affect the other).
    \item Condition of separability for 2 qubits: if $c_{ij} = c^{(A)}_i c^{(B)}_j$, with $i,j\in \{0,1\}$, then
        $\mathrm{det} \begin{pmatrix} c_{00} & c_{01} \\ c_{10} & c_{11} \end{pmatrix} = 0 \Leftrightarrow $ separable. E.g. in $\left(\alpha \ket{0}_A \otimes \beta \ket{0}_B\right)$, $c_{00} = \alpha \beta$.
    \item $H_{AB} = H_A \otimes \id_B + \id_A \otimes H_B \Longrightarrow e^{-itH_{AB}} = e^{-itH_A} \otimes e^{-itH_B}$ \\ A separable unitary generates no entanglement when applied to a separable state. The reduced density matrices of each partition remain pure whenever the full state remains separable.
\end{squishlist}

\columnbreak

\graypar{Quantum eraser}
\begin{squishlist}
    \item $\ket{\nearrow} = \frac{1}{\sqrt{2}}(\ket{H}+\ket{V}),\: \ket{\swarrow} = \frac{1}{\sqrt{2}}(\ket{H} - \ket{V})$
    \item $\ket{H} = \frac{1}{\sqrt{2}}(\ket{\nearrow} + \ket{\swarrow}),\: \ket{V} = \frac{1}{\sqrt{2}}(\ket{\nearrow} - \ket{\swarrow})$
    \item $P(x) = \braket{\psi(x,t)|\left(\ket{x}\bra{x} \otimes \id\right)| \psi(x,t)}$ Probability density on screen
\end{squishlist}


\graypar{Bell states}
\begin{squishlist}
    \item $\ket{\Phi^+} = \frac{1}{\sqrt{2}} \left( \ket{0}_A \otimes \ket{0}_B + \ket{1}_A \otimes \ket{1}_B\right)$ \hfill maximally entangled\\
          $\ket{\Phi^-} = \frac{1}{\sqrt{2}} \left( \ket{0}_A \otimes \ket{0}_B - \ket{1}_A \otimes \ket{1}_B\right)$ \\
          $\ket{\Psi^+} = \frac{1}{\sqrt{2}} \left( \ket{0}_A \otimes \ket{1}_B + \ket{1}_A \otimes \ket{0}_B\right)$ \\
          $\ket{\Psi^-} = \frac{1}{\sqrt{2}} \left( \ket{0}_A \otimes \ket{1}_B - \ket{1}_A \otimes \ket{0}_B\right)$
    \item They form an orthonormal basis of the Hilbert space of the two spins $\H = \H_A \otimes \H_B$
    \item Non separable
    \item Eigenstates of $\hat{H} = \mu_x \hat{S}_x^{(A)} \otimes \hat{S}_x^{(B)} + \mu_y \hat{S}_y^{(A)} \otimes \hat{S}_y^{(B)}$
    \item $P_{\hat{S}_x^{(A)}}(\pm \frac{\hbar}{2} | \psi) = \frac{1}{2} \quad \forall \ket{\psi} $ a Bell state. 
\end{squishlist}


\graypar{CHSH Inequality}
\begin{squishlist}
    \item Bipartite system with LHS measuring device, which can measure either $A$ or $A'$, and RHS device which can measure $B$ or $B'$.
    The probability of a result combination is written as $P(l,r|L,R)$ with $L,R$ the settings on LHS and RHS device and $l$ and $r$ the results of the measures ($\pm 1$).
    \item \textbf{Bell inequalities} define a correlation coefficient $C$ and then place an upper bound on possible values this coefficient can take if you assume factorisability
    \item \textbf{Factorisability}: $p(l,r|L,R) = \int P(l|L,\lambda) P(r|R,\lambda) P(\lambda) d\lambda$ where $\lambda$ incorporates all effects from the system's shared history.
    Two \textbf{necessary conditions} for factorisability to hold: Setting Independence $P(l|L,B,\lambda) = P(l|L,B',\lambda)$ and Outcome Indipendence $P(l,A,R,r,\lambda) = P(l|A,R,r',\lambda)$.
    \item $C:= |\langle LR \rangle - \langle LR'\rangle  | + |\langle LR\rangle + \langle L'R \rangle |$ with $\langle LR \rangle = \sum_{l,r=\pm 1}lr P(l,r|L,R)$
    \item $C \leq 2$ in the classical case, violated in quantum case (Tsirelson's bound: $C \leq 2\sqrt{2}$) \\ Quantum Mechanics violates outcome indipendence.
    \item A mixture of product states does not include non-classical correlations (through entanglement) that would allow to violate Bell inequality.
\end{squishlist}


\graypar{Reduced and mixed quantum states}
\begin{squishlist}
    \item \textbf{Density operator}: $\rho = \kpsi \bra{\psi} \Longrightarrow \braket{O} = \Tr (\rho O)$ for any observable $O$
    \item \textbf{General single qubit}: $\rho = \begin{pmatrix} cos(\theta/2)^2 & \cos(\theta/2)\sin(\theta/2)e^{-i\phi} \\ \cos(\theta/2)\sin(\theta/2)e^{i\phi} & \sin(\theta/2)^2 \end{pmatrix} = $ \\
    $= \frac{1}{2}\sigma_0 + \frac{1}{2}\sum_{i=1}^3 v_i \sigma_i$ with $\vec{v} = (\sin\theta \cos\phi , \sin\theta\sin\phi, \cos\theta);\; v_i = \Tr[\rho \sigma_i]$\\
    
    Proof: the density matrix of an arbitrary 2-level system can be written as \\
    $\rho  = a \id  + b \sigma_x + c \sigma_y + d \sigma_z = a \id + \vec{\sigma}\cdot \vec{v'}$ then use $\Tr \rho = 1$ to extract $a = \frac{1}{2}$ and define $\vec{v} = 2 \vec{v'}$

    \item The eigenvalues of $\rho$ are $\frac{1}{2}(1 \pm |\vec{v}|)$ Using $\braket{\psi|\rho | \psi} \geq 0$ this yields $|\vec{v}| \leq 1$

    \item $\rho = \sum_k p_k \ket{\psi_k}\bra{\psi_k}$ System prepared in state $\ket{\psi_k}$ with prob. $p_k$ (\textbf{mixed} state)
    \item Maximally mixed: $\frac{\id}{2} = \frac{1}{2} (\ket{0}\bra{0} + \ket{1}\bra{1})$
    \item If $\rho = p \dpsi + (1-p) \ket{\phi}\bra{\phi}$ where
    $\psi, \phi$ have Bloch vectors $\vec{v}, \vec{u}$ of pure states, \textbf{the mixed state has Bloch vector} $\vec{w} = p \vec{v} + (1-p) \vec{u}$; $|\vec{w}|^2 \leq 1$

    \item More generally $\rho_{\text{mixed}} = \frac{1}{2} ( 1 + \vec{\sigma} \cdot \sum_i p_i \vec{r_i}) \Longrightarrow r_{\text{mixed}} = \sum_i p_i \vec{r_i}$

    \item $\rho_A = \sum_{k=1}^{d_B}(\id_A \otimes \bra{k})\rho_{AB}(\id_A \otimes \ket{k}) = \Tr_B[\rho_{AB}]$ \quad \textbf{Reduced} state
    
    \item $\Tr_B[\ket{ij}\bra{kl}] = \ket{i}\bra{k} \Tr[\ket{j}\bra{l}]$ \hfill \textbf{Properties of trace}
    \item $\Tr\left(\hat{C}\kpsi \bra{\psi}\hat{D}\right) = \Tr\left(\kpsi \bra{\psi}\hat{D}\hat{C}\right) = \Tr\left( \bra{\psi}\hat{D}\hat{C} \kpsi\right) =  \bra{\psi}\hat{D}\hat{C} \kpsi$

    \item \textbf{Properties}: 
    (i) $\rho_A^{\dagger} = \rho_A$ (self-adj.) \, (ii)$\Tr(\rho_A) = \sum_i  \sum_{\mu}\alpha^*_{i,\mu}\alpha_{i,\mu} = |\psi|^2 = 1$ \\ (iii)$\braket{\psi|\rho_A|\psi} \geq 0$ for all $\kpsi \in A$ i.e. positive or null eigenvalues
    \item Conseq., $\rho_A = \sum_j p_j \ket{k}\bra{j}$ where $p_j\geq 0$ and $\sum p_j = 1$.\\ $\braket{O} = \Tr(\rho_A O) = \sum_j p_j \braket{j|O|j} = \sum_{p_j} \braket{O}_{\ket{j}}$
    
    \item A \textbf{pure state} is a density matrix that has only one non zero eigenvalue. A \textbf{mixed state}
    can have more than one non-zero eigenvalue.
    \item If a density op. describes a pure state, then it is a projector ($\rho^2 = \rho$) and $\Tr[\rho^2] = 1$ (indeed, given that $\sum p_n = 1$, $\Tr[\rho^2] = 1 \text{ iff } \rho$ is pure). If $\rho$ is not pure then $\rho^2 \neq \rho$ and $\Tr[\rho^2] < 1$ (\emph{purity} of a state).

    \item The Bloch vector for a pure state has norm 1. To prove this either expand $\rho^2$ and use $\Tr[\rho^2] = 1$ or use general single qubit and retrieve $\vec{r} = (\sin \theta \cos \phi, \ldots)$
    
    \item A state of a system in $\mathcal{H}_1 \otimes \mathcal{H}_2$ is separable if we can write its density matrix as \\
    $\rho_s = \sum_k p_k \rho_k^{(1)} \otimes \rho_k^{(2)} $

    \item \textbf{Partial transpose}: $\rho^{T_B} = \sum_k \rho_k^{(1)} \otimes \left(\rho_k^{(2)}\right)^T$
    i.e. $(\rho^{T_B})_{i\mu,j\nu} = \rho_{i\nu,j\mu}$ \\
    If $\rho$ is separable, then the partial transpose is a valid density matrix and in particular all its eigenvalues have to be non-negative. Therefore if at least one eigenvalue of $\rho^{T_B}$ is negative, the state $\rho$ must be entangled (\textbf{PPT criterion}).


    \item \textbf{Evolution:} $\rho(t=0) = \sum_j \alpha_j \ket{\psi_j(0)}\bra{\psi_j(0)}$ initial state \\ $\Longrightarrow \rho(t) = \sum_j \alpha_j e^{-iHt} \ket{\psi_j(0)}\bra{\psi_j(0)}e^{iHt} \Longrightarrow i \D{t}{\rho} = [\hat{H},\rho]$
    \item \textbf{Why is signaling impossible?} No matter what is performed upon the other partitions, the reduced density matrix is unchanged. Because the statistics of local measurements are informed entirely by expected values of operators upon the reduced density matrix, they are also independent of operations on other partitions.
\end{squishlist}

\graysec{Identical multi-particle systems}
\begin{squishlist}
    \item $\P_{1,2} \psi(\vec{r_1}, \vec{r_2}) = \psi(\vec{r_2}, \vec{r_1}) = \pm \psi(\vec{r_1}, \vec{r_2})$ \quad $+1$: \textbf{bosons} \quad $-1$: \textbf{fermions}
    \item $\P_{jk} = \P_{kj}$ \squishsep $\P_{jk}^2 = \id \Leftrightarrow \P_{jk}^{-1}=\P_{jk}$ \squishsep $\P_{jk} = \P_{jk}^{\dagger}$
    \item $\braket{\psi_{12}|O|\psi_{12}} = \braket{\psi_{12}|\P_{12}^{\dagger}O\P_{12}|\psi_{12}}$ for all $\psi$ $\Rightarrow [\P_{12},O] = 0, [\P_{12},H] = 0$
    \item \textbf{Possible basis states} for a system of \\ $n$ bosons: $\ket{\psi_{\vec{x}}} = \mathcal{N} \sum_{\P \in S_n} \P\ket{x_1,x_2, \ldots, x_n}$ with $\mathcal{N} = \frac{1}{\sqrt{n!}\sqrt{\prod_k n_k !}}$ \\
    $n$ fermions $\ket{\psi_{\vec{x}}} = \frac{1}{\sqrt{n!}} \sum_{\P \in S_n} \text{sign}(\P) \P \ket{x_1, x_2, \ldots, x_n}$
    \item How to use this formula: if $\kpsi$ is a possible configuration of the system (e.g. $\ket{001}$ for three particles which can be in $0$ or $1$), apply formula to $\kpsi$ and obtain state which respects (a)symmetry (e.g. $\frac{1}{\sqrt{3}}(\ket{001} + \ket{010} + \ket{100})$ for bosons)
\end{squishlist}

\graypar{Second Quantization}
\begin{squishlist}
    \item Kets indicate the number of times a wave function is involved:\\
    for Bosons $\frac{1}{\sqrt{2}} ( \ket{\uparrow \downarrow} + \ket{\downarrow \uparrow}) \rightarrow \ket{11};\; \ket{\uparrow \uparrow} \rightarrow \ket{20};\; \ket{\downarrow \downarrow} \rightarrow \ket{02}$\\
    for Fermions $\frac{1}{\sqrt{2}} ( \ket{\uparrow \downarrow} - \ket{\downarrow \uparrow}) \rightarrow \ket{11}$ (only possible values 0,1)
    \item \textbf{Creation and annihilation operators} to increase or decrease the number of particles:
    Bosons:  $\left\{ \begin{aligned}
    \hat{c}_i^{\dagger} \ket{n_1, \cdots, n_i, \cdots} &= \sqrt{n_i + 1} \ket{n_1, \cdots, n_i +1, \cdots} \\
    \hat{c}_i \ket{n_1, \cdots, n_i, \cdots} &= \sqrt{n_i} \ket{n_1, \cdots, n_i -1, \cdots}
    \end{aligned}\right.$ \\
    $[\hat{c}_i, \hat{c}_j] = [\hat{c}_i^{\dagger}, \hat{c}_j^{\dagger}] = 0$; \qquad $[\hat{c}_i, \hat{c}_j^{\dagger}] = \delta_{ij}$ \\
    Fermions: $\left\{ \begin{aligned}
    \hat{c}_i^{\dagger} \ket{n_1, \cdots, n_i, \cdots} &= (-1)^{n_1+\cdots+n_i-1}(1-n_i) \ket{n_1, \cdots, n_i +1, \cdots} \\
    \hat{c}_i \ket{n_1, \cdots, n_i, \cdots} &= (-1)^{n_1+\cdots+n_i-1} n_i \ket{n_1, \cdots, n_i -1, \cdots}
    \end{aligned}\right.$ \\
    $\{\hat{c}_i, \hat{c}_j\} = \{\hat{c}_i^{\dagger}, \hat{c}_j^{\dagger}\} = 0$; \qquad $\{\hat{c}_i, \hat{c}_j^{\dagger}\} = \delta_{ij}$ 
\end{squishlist}

\columnbreak


\graysec{Perturbation Theory}
\begin{squishlist}
    \item $H = H_0 + \lambda V$, with $H_0 \ket{\phi_n} = \epsilon_n \ket{\phi_n}$ known,\, $\lambda \in \mathbb{R}^+$. 
    \item $H \ket{\psi_n} = E_n \ket{\psi_n}$ eigenspectrum unknown. The solution in the limit of small $\lambda$ is 
    $\ket{\psi_n} = \ket{\phi_n} + \lambda \ket{\psi_n^{(1)}} + \lambda^2 \ket{\psi_n^{(2)}} + \ldots$ \\
    $E_n = \epsilon_n + \lambda E_n^{(1)} + \lambda^2 E_n^{(2)} + \ldots$
    \item S.E.: $(H_0 + \lambda V)(\ket{\phi_n} + \lambda \ket{\psi_n^{(1)}} + \lambda^2 \ket{\psi_n^{(2)}} + \ldots) = (\epsilon_n + \lambda E_n^{(1)} + \lambda^2 E_n^{(2)} + \ldots)(\ket{\phi_n} + \lambda \ket{\psi_n^{(1)}} + \lambda^2 \ket{\psi_n^{(2)}} + \ldots)$ must be satisfied at each order in $\lambda$
\end{squishlist}
\graypar{Non-degenerate Time-Indipendent Perturbation Theory}
\begin{squishlist}
    \item \textbf{Zero-th order} $H_0 \ket{\phi_n} = \epsilon_n \ket{\phi_n}$ unperturbed eigenvalue problem
    \item \textbf{1st order} $E_n^{(1)} = \braket{\phi_n|V|\phi_n}$; \quad $\ket{\psi_n^{1}} = \sum_{m\neq n}\frac{\braket{\phi_m |V |\phi_n}}{\epsilon_n - \epsilon_m} \ket{\phi_m}$
    \item \textbf{2nd order} $E_n^{(2)} = \sum_{m\neq n}\frac{|\braket{\phi_m|V|\phi_n}|^2}{\epsilon_n - \epsilon_m}$; for approx. to be val.  $|E_n^{(2)}| \ll |E_n^{(1)}|$ 
    \item satisfied as long as $\frac{1}{\Delta} ( \braket{\phi_n | V^2|\phi_n} - \braket{\phi_n | V| \phi_n}^2) \ll \braket{\phi_n | V | \phi_n}$, \\ 
    $\Delta = \min_m | \epsilon_n - \epsilon_m|$ or more restrictive: $\left|\frac{\braket{\phi_m | V | \phi_n}}{\epsilon_n - \epsilon_m}\right| \ll 1$
\end{squishlist}

\graypar{Degenerate Time-Indipendent Perturbation Theory}
\begin{squishlist}
    \item $n$th energy state has $N$-fold degeneracy $\Rightarrow$ $H_0$ has energy $\epsilon_n$ with $\phi_{n_i}, i=1 \ldots N$
    \item $\sum_j V_{ij}c_j = E_n^{(1)}c_i \Longleftrightarrow \hat{V} \vec{c} = E_n^{(1)} \vec{c} \qquad V_{ij} = \braket{\phi_{n_i} | V | \phi_{n_j}}$ \\ 
    eigenvalue problem, must diagonalise $V$ matrix (in the degenerate subspace only!)
    \item The eigenvectors are the corrected eigenstates (to 0th order), the eigenvalues are the 1st order correction to energy. If eigenvalues are the same, degeneracy is not lifted.
    \item E.g. $V = \begin{pmatrix}
        0 & \alpha & 0 & 0 \\
        \alpha  & 0 & 0 & 0 \\
        0 & 0 & 0 & 0 \\
        0 & 0 & 0 & 0 \\
    \end{pmatrix}$ $\overset{\text{diag}}{\longrightarrow}$ 
    $\alpha,  \begin{pmatrix} \frac{1}{\sqrt{2}} \\ \frac{1}{\sqrt{2}} \\ 0 \\ 0 \end{pmatrix};
    - \alpha,  \begin{pmatrix} \frac{1}{\sqrt{2}} \\ -\frac{1}{\sqrt{2}} \\ 0 \\ 0 \end{pmatrix};
    0,  \begin{pmatrix} 0 \\ 0 \\ 1 \\ 0 \end{pmatrix};
    0,  \begin{pmatrix} 0 \\ 0 \\ 0 \\ 1 \end{pmatrix} $\\
    Thus $\ket{0} \rightarrow \frac{1}{\sqrt{2}}\left( \ket{0} + \ket{1}\right);\quad \ket{1} \rightarrow \frac{1}{\sqrt{2}}\left( \ket{0} - \ket{1}\right); \quad \ket{2},\ket{3}$ unchanged\\
    The corrected eigenstates have an energy correction $E^{(1)}$of $\alpha$, $-\alpha$, 0,0.
\end{squishlist}


\graypar{Time-Dependent Hamiltonians}
\begin{squishlist}
    \item $U(t,t_0) = \id + \sum_{n=1}^\infty \frac{(-1)^n}{n!} \int_{t_0}^t \ud t_1 \int_{t_0}^t \ud t_2 \ldots \int_{t_0}^t \ud t_n T(H(t_1)\ldots H(t_n))$\\
    where $T[H(t_1)H(t_2)\ldots H(t_n)] = H(t_{i_1})H(t_{i_2})\ldots H(t_{i_n}),\, t_{i_1} \geq t_{i_2} \geq \ldots$
    \item Alternatively: $U(t,t_0) \approx \sum_j \exp(i H(t_j) \delta t)$ with an error of \\
    $|\int_{t_0}^t \ud s H(s) - \delta t \sum_{r=1}^{n_t} H(t_0 + r \delta t)|^2$ (discretisation)
\end{squishlist}

\graypar{Interaction picture}
\begin{squishlist}
    \item Schrödinger picture: $i \D{t}{}\ket{\phi_S(t)} = H(t) \ket{\phi_S(t)}$, \quad $O_S(t) = O_S$
    \item Heisenberg picture: $O_H(t) = U_S(t,t_0)^{\dagger} O_S U_S(t,t_0)$, \quad $\ket{\phi_H(t)} = \ket{\phi_S(t_0)}$
    \item Interaction picture: $H(t) = H_H + V_S(t)$, 
    \item $O_I(t) = e^{iH_0(t-t_0)}O_S(t)e^{-iH_0(t-t_0)}$
    \item $\ket{\phi_I(t)} = U_I(t,t_0) \ket{\phi_I(t_0)}$, \quad $U_I = e^{iH_0(t-t_0)} U_S(t,t_o)$
    \item $\D{t}{U_I} = -i V_I(t) U_I(t,t_0)$
\end{squishlist}

\graypar{Time-Dependent Perturbation Theory}
\begin{squishlist}
    \item $U_I(t,t_0) \approx \id -i \int_{t_0}^t \ud t_1 V_I(t_1) + \ldots$
    \item The exact transition probability between two states is $\left| \braket{\psi|\phi} \right| ^2$

    \item $P_{i\rightarrow n}(t) = |\braket{n|\phi_S(t)}|^2 = |\braket{n|U_I(t,t_0)|i}|^2 = \\ 
    = \left| -i \int_{t_0}^t \ud t_1 e^{i(E_n-E_i)(t-t_0)} \braket{n|V(t_1,t_0)|i}\right|^2$ general expression of \textbf{transition probabilities} between eigenstates $\ket{i}$ and $\ket{n}$ of $H_0$. Note: if $V=0$ then a system in an eigenstate stays in an aigenstate.
    
    \item For a constant potential $P_{i\rightarrow n}(t) \overset{t\rightarrow \infty}{=} 2 \omega t \left|\braket{n|V|i}\right|^2 \delta(E_n - E_i)$ \\
    \D{t}{}$P_{i\rightarrow n}(t) =2 \pi \left|\braket{n|V|i}\right|^2 \delta(E_n - E_i)$
\end{squishlist}

\graypar{Variational method}
\begin{squishlist}
    \item $\dfrac{\braket{\psi | H | \psi}}{\braket{\psi|\psi}} \geq E_0$ \quad \textbf{Variational principle}
    \item The idea is to come up with a parameterised guess for the state $\kpsi$, and then we use the variational principle to find the parameter values that minimize $\psi$.\\
    It generalises to excited states orthogonal to the ground state $\phi_0$(i.e. $\braket{\phi_0|\psi} = 0$) by replacing $E_0$ with $E_1$.
    Limitation: the ground state should be known to ensure it is orthogonal to the excited state.
    \item Steps: compute $\frac{\braket{\psi | H | \psi}}{\braket{\psi|\psi}} = E(a)$ then minimise $E$ with respect to the parameter.
\end{squishlist}


\columnbreak

\graypar{Particle in a box}
\begin{squishlist}
    
    \item $V(x) = \begin{cases}
        0 & \text{if} |x| \leq L/2 \\ \infty & \text{else}
    \end{cases}$
    \qquad
    $\phi_n(x) = \begin{cases}
        \sqrt{\frac{2}{L}} \cos(\frac{n\pi }{L} x) & n \text{ odd} \\
        \sqrt{\frac{2}{L}} \sin(\frac{n\pi }{L} x) & n \text{ even} \\
    \end{cases}$
    
    \item $E_n = n^2 \frac{\pi^2 \hbar^2}{2 m L^2}$
\end{squishlist}

\graypar{Harmonic oscillator}
\begin{squishlist}
    \item$H = \frac{1}{2m}\hat{p}^2 + \frac{1}{2}m \omega^2 \hat{x}^2 = \hbar \omega (\hat{N} + \frac{1}{2}), \quad \hat{N} = \hat{a}^{\dagger} \hat{a}$
    \squishsep $\hat{p}^2 = - \frac{\hbar^2}{2m} \dder{x}{}$
    \item $\hat{a} = \sqrt{\frac{m\omega}{2\hbar}} \left( \hat{x} + \frac{i \hat{p}}{m \omega}\right)$ \quad 
     $\hat{a}^{\dagger} = \sqrt{\frac{m\omega}{2\hbar}} \left( \hat{x} - \frac{i \hat{p}}{m \omega}\right)$ \quad 
    \item $\hat{x} = \sqrt{\frac{\hbar }{2 m \omega}}(\hat{a} + \hat{a}^{\dagger})$ \quad $\hat{p} = i \sqrt{\frac{m \omega \hbar}{2}} ( \hat{a}^{\dagger} - \hat{a})$
    \item $[\hat{x}, \hat{p}] = i \hbar, \quad [\hat{a}, \hat{a}^{\dagger}] = 1, \quad [\hat{N}, \hat{a}] = - \hat{a}, \quad [\hat{N},\hat{a}^\dagger] = \hat{a}^{\dagger}$
    \item $\phi_0(x) = \left( \frac{m \omega}{\phi \hbar}   \right)^{1/4} \exp \left( - \frac{m \omega }{2 \hbar} x^2\right)$
    \item $ \phi_n = \frac{1}{\sqrt{n!}}(a^{\dagger})^n \phi_0 = \frac{1}{\sqrt{2^n n!}} \left(\frac{m\omega}{\pi \hbar}\right)^{1/4} e^{-\frac{m \omega x^2}{2 \hbar}} H_n \left(\sqrt{\frac{m \omega }{\hbar}}x \right) \\ $ with $H_n(z) = (-1)^n e^{z^2} \frac{d^n}{dz^n} (e^{-z^2})$
    \item $\hat{H} \phi_n = E_n \phi_n = \hbar \omega (n + \frac{1}{2})\phi_n$
    \item $\hat{a}^{\dagger} \phi_n = \sqrt{n+1} \phi_{n+1} \quad \hat{a}\phi_n = \sqrt{n} \phi_{n-1}$
\end{squishlist}


\graypar{Hermitian and unitary operators}
\begin{squishlist}
    \item Hermitian operator: $M = M^{\dagger}$ $\longrightarrow$ diagonalisable with real eigenvalues, linear
    \item Unitary operators: $U U^{\dagger} = U^{\dagger}U = \id$ $\longrightarrow \braket{\psi | U^{\dagger}U|\psi} = \id $ and linear
\end{squishlist}

\graypar{Tensor product}
\begin{squishlist}
\item $\vectwo{a}{b} \otimes \vectwo{\alpha}{\beta} = \begin{pmatrix} a \alpha & a \beta & b \alpha & b \beta \end{pmatrix}^T$
    
\item $A \otimes B = \begin{pmatrix} A_{11}B & A_{12}B \\ A_{21}B & A_{22}B \end{pmatrix}$
\squishsep $A \oplus B = \begin{pmatrix} A & 0 \\ 0 & B \end{pmatrix}$

\item $R(g)^{\otimes k} = R(g) \otimes \ldots \otimes R(g)$ \squishsep $\bigoplus_k R(g) = (R \bigoplus \ldots \bigoplus R)(g)$

\item $f(\hat{A} \otimes \hat{B}) \ket{a}\ket{\phi} = \ket{a} \otimes f(a\hat{B})\ket{\phi}$
\end{squishlist}


\graysec{Trigonometry}
\begin{squishlist}
    
    \item $\cos(a+b) = \cos(a)\cos(b)-\sin(a)\sin(b)$ \squishsep $\sin(2a) = 2\sin a \cos a$
    \item $\sin(a+b) = \sin(a)\cos(b) + \cos(a)\sin(b)$ \squishsep $\cos(2a) = \cos^2 a - \sin^2 a$
    \item $2\sin(a)\sin(b) = \cos(a-b) - \cos(a+b)$ \smallsquishsep $\sin(a/2) = \pm \sqrt{(1-\cos(a))/2}$
    \item $2\cos(a)\cos(b) = \cos(a+b) + \cos(a-b)$ \smallsquishsep $\cos(a/2) = \pm \sqrt{(1+\cos(a))/2}$
    \item $2\cos(a)\sin(b) = \sin(a+b) - \sin(a-b)$ \smallsquishsep $\sin^2a = (1-\cos 2a)/2$
    \item $\cos(a) + \cos(b) = 2\cos\left((a+b)/2\right)\cos\left((a-b)/2\right)$ \;\squishitem \; $\cos^2 a = (1+\cos 2a) / 2$
    \item $\sin(a) + \sin(b) = 2\sin\left((a+b)/2\right)\cos\left((a-b)/2\right)$ 
    \item $\cos(a) - \cos(b) = -2\sin\left((a+b/2\right)\sin\left((a-b/2\right)$ 
\end{squishlist} 

\graypar{Gaussian integrals}
\begin{squishlist}
    \item $\displaystyle{\int_{-\infty}^{+\infty}} \ud x \, e^{-\alpha^2(x+\beta)^2 + Bx} = \dfrac{\sqrt{\pi}}{\alpha} \displaystyle{e^{\frac{B^2}{4\alpha^2}-B\beta}} $ with Re$(\alpha^2) \ge 0, \, B, \beta \in \mathbb{C}$ 
    
    \item $      \displaystyle{\int_{-\infty}^{+\infty}} \ud x \, x^n e^{-\frac{1}{2}Ax^2} = 
    \begin{cases}
    (n-1)!!\sqrt{2\pi}A^{-\frac{n+1}{2}} \quad \text{If n is even} \\
    0 \quad \text{If n is odd}
    \end{cases}
    $
\end{squishlist}


\columnbreak


\graysec{Group Theory}

\begin{squishlist}
    \item \textbf{Group}: Set $G$ equipped with operation $*$ such that
    \begin{squishitemize}
        \setlength\itemsep{2pt}
        \item $G$ closed under $*$, i.e. if $a,b \in G$ then $a*b \in G$
        \item Associative: $\forall a,b,c \in G$ one has $(a * b) * c = a * (b*c)$
        \item Has identity, i.e an element $e$ such that $e * a = a \forall a \in G$
        \item Has inverse, i.e. $\forall a \in G$ it exists $b \in G$ such that $b*a = a*b = e$. ($b = a^{-1}$)
    \end{squishitemize}
    \item Any unitary that leaves a property invariant forms a grop with * matrix multiplic.
    \item \textbf{Finite group}: A group that contains a finite number of elements (the group \emph{order}).
    
    \begin{minipage}{0.75\columnwidth}
    \item Order 1 group (the only one, trivial group): 
    \end{minipage}
    \begin{minipage}{0.25\columnwidth}
    $\begin{array}{c | c }
        * & e \\
        \hline
        e & e 
    \end{array}$
    \end{minipage}

    \begin{minipage}{0.75\columnwidth}
    \item Order 2: parity group
    \end{minipage}
    \begin{minipage}{0.25\columnwidth}
    $\begin{array}{c | c c }
        * & e & a \\
        \hline
        e & e & a \\
        a & a & e
    \end{array}$
    \end{minipage}
    
    \begin{minipage}{0.75\columnwidth}
    \item The unique order 3 group is the cyclic $\mathbb{Z}_3$ group
    \end{minipage}
    \begin{minipage}{0.25\columnwidth}
        $\begin{array}{c| c c c}
            * & e & a & b \\
            \hline
            e & e & a & b \\
            a & a & b & e \\
            b & b & e & a \\
        \end{array}$
    \end{minipage}
        
    \item Order 4 has cyclic and symmetry of a rectangle
    
    \begin{minipage}{0.5\columnwidth}
        \item Order 6 has cyclic group $\mathbb{Z}_6$ and the C3v group (on the right)
    \end{minipage}
    \begin{minipage}{0.5\columnwidth}
        $\begin{array}{c | c c c c c c }
            * & e & a & a^2 & b & c & d \\
            \hline
            e & e & a & a^2 & b & c & d \\
            a & a & a^2 & e & c & d & b \\
            a^2 & a^2 & e & a & d    & b    & c     \\
            b   & b  & d & c & e   & a^2 & a  \\
            c   & c  & b & d & a   & e   & a^2 \\
            d   & d  & c & b & a^2 & a   & e                        
        \end{array}$
    \end{minipage}

    \begin{minipage}{0.4\columnwidth}
    \item The general cyclic group $\mathbb{Z}_n$ ($a^n = 1$) is
    \end{minipage}
    $\begin{array}{c| c c c c c}
        * & e & a^1 & a^2 & \ldots & a^{n-1} \\
        \hline
        * & e & a^1 & a^2 & \ldots & a^{n-1} \\
        a^1 & a^1 & a^2 & a^3 & \ldots & e \\
        a^2 & a^2 & a^3 & a^4 & \ldots & a^1 \\
        {\vdots} \\
        a^{n-1} & a^{n-1} & e & a^1 & \ldots & a^{n-2}
    \end{array}$

    \item Symmetric permutation group $S_n$ (all possible permutations of $n$ objects)\\
    e.g. $S_3 = \{ I,$SWAP$_{12},$SWAP$_{13}$,SWAP$_{23}$,CYCLE$_{123},$CYCLE$_{321}$\} isomorphic to C3v

    \item \textbf{Lie group}: a continous group that depends analytically on some continous parameters $\lambda$. Examples: Real $d$-dimensional rotations $SO(d)$, the orthogonal group $O(d)$, the unitary group $U(d)$, the special unitary group $SU(d)$
    
    \item \textbf{Abelian group}: $a * b = b * a \; \forall a,b \in G$
    \item \textbf{Subgroup}: a subset $H$ of the group $G$ is a subgroup of $G$ if and only if it
    is nonempty and itself forms a group. If it is neither the identity or $G$ itself then it is a \emph{proper} subgroup. \textbf{Lagrange Theorem:} the order of $H$ divides the order og $G$. Thus if the order of a group is prime there is only one possible group

    \item \textbf{Group homomorphism}: an application from $(G, *)$ to $(G, **)$ such that $\forall x,y \in G \quad f(x * y) = f(x) ** f(y)$. An isomorphism sets a one-to-one correspondance.
    
\end{squishlist}

\graypar{Representations}
\begin{squishlist}
    \item A \textbf{representation} $R$ of a group $G$ on a vector space
    $V$ is a group homomorphism from $G$ to a set of matrices that act on a vector space $V$.
    The dimension of a representation $R$ is defined to be the dimension of the vector state $V$, i.e.,
    $\text{dim}(R)=\text{dim}(V)$.
    \item All groups allow \textbf{trivial representation} $\forall g \in G, \, R(g) = \id$
    \item The \textbf{regular representation} is obtained by reordering the Caylei table so that only $e$ fills the diagonal, then to every element assign the matrix obtained by replacing 1 in the positions where the element is in the table and 0 everywhere else.
    \item \textbf{Equivalent} reps are related by a similarity transformation $R'(g) = S R(g) S^{-1}$
    \item If $R_1$ and $R_2$ are two representations for $G$, then $R_1(g) \otimes R_2(g)$ is also a rep
    \item The direct sum $R_1 \oplus R_2$ is a rep of $G$ acting on $V_1 \oplus V_2$ \\
    $(R_1 \oplus R_2)(g) = \begin{pmatrix} R_1(g) & 0 \\ 0 & R_2(g) \end{pmatrix}$

    \item Let $U_g$ be a rep of $G$ and $H$ an Hermitian operator such that $[U_g,H] = 0 \; \forall g$. Then $H$ and $U_g$ are simultaneously block diagonalised (in the same basis). \\
    E.g. $U_g \otimes U_g$, the tensor product representation of $SU(2)$, commutes with SWAP and has therefore a symmetric ($d=3$) and an asymmetric ($d=1$) invariant subspace.
    \item \textbf{Reducible representation}: a rep $R(g)$ of $G$ over a vector space $V$ is reducible if there exists an invariant subspace, i.e. if there exists a non-trivial subspace $W$ of $V$ such that for all $\ket{w} \in W$ we have $R(g) \ket{w} \in W$ for any $g$.
    \item \textbf{Completely reducible rep}: if it splits into a direct sum of irreducible reps
    
    \item \textbf{Schur's 1st lemma}: Let $R_1(g)$ and $R_(g)$ be two non-equivalent irreducible reps of $G$, acting on vector spaces $V_1, V_2$. If there is a matrix $A$ such that \\ $A R_1(g) = R_2(g) A \; \forall g$ \quad then $A=0$.\\
    Equivalently, if you can find $A$ different than 0 that satisfies the eq., then the reps are reducible.

    \item \textbf{Schur's 2nd lemma}: Let $R$ be an irreducible rep of $G$. If $A R(g) = R(g) A \; \forall g \in G$ then $A = \lambda \id $ for some $\lambda \in \mathbb{C}$. \\
    Equivalent to: if $[A,R(g)] = 0$ i.e. if they can be diagonalised in the same base.
    
    \item \textbf{Burnside's lemma}: for a finite group of order $h$ there are only a finite number $n$ of irreducible representations $a = 1 \ldots n$ of dimension $l_a$ and $\sum_{a=1}^n l_a^2 = h$
    \item In a group $G$, two elements $g$ and $g'$ are \textbf{equivalent} if there exists another element $f$ such that $g' = f^{-1} g f$. This divides $G$ in conjugacy classes.
    \item For a finite group, the number of (non-equiv.) irreps is equal to the number of conjugacy classes
    \item All irreducible reps of Abelian groups are scalar ($d=1$). An Abelian group of order $n$ has $n$ conjugacy classes and thus $n$ irreducible reps.
    
    \item \textbf{Grand Orthogonality Theorem}: Let $R_a$ and $R_b$ be two non-equiv. unitary irreducible reps of a finite group $G$ of order $N$.
    Let $n_a$ and $n_b$ be the dimensions of the vector space for $R_a$ and $R_b$. Then \\
    $\sum_{g\in G} \dfrac{n_a}{N} \left[R_a(g)^{\dagger}\right]_{jk} \left[R_b(g)\right]_{lm} = \delta_{ab} \delta_{jm} \delta_{lk}$
    \begin{squishitemize}
        \item if $a \neq b$ then $\sum_{g\in G} \left[R_a(g)^{\dagger}\right]_{jk} \left[R_b(g)\right]_{lm} = 0$ for all $i,j,k,l$
        \item if $a = b$ then $\sum_{g\in G} \left[R_a(g)^{\dagger}\right]_{jk} \left[R_a(g)\right]_{lm} = 0$ if $j \neq m$ and/or $l \neq k$
        \item if $a=b$ and $j=m$ and $l=k$ then $\sum_{g\in G} \left[R_a(g)^{\dagger}\right]_{jk} \left[R_a(g)\right]_{jk} = \dfrac{N}{n_a}$
    \end{squishitemize}

    \item A finite group can only have a finite number of inequivalent irreducible
    representations. Specifically, the \textbf{maximum number of possible irreps} is given by the order of the group.
    Proof: irreps give us 'vectors of matrices' in a vector space of dimension $|G|$ and the theorem says they must be orthogonal. But there are at most $|G|$ orthogonal vectors in a vector space of dimension $|G|$

    \item \textbf{Group averaging}: if $d$ is the dim. of vector space of rep and $U_g$ is an irreducible representation then $\braket{X}_G = \frac{1}{N} \sum_g U_g X U_g^{\dagger} = \frac{1}{d} \Tr[X] \id$  
    \item Proof: $\frac{1}{N} \sum_g U_g X U_g^{\dagger} = \frac{1}{N} \sum_g \left( \sum_{l,m}[U_g]_{lm} \ket{l}\bra{m}\right) X \left( \sum_{k,j} [U_g^{\dagger}]_{kj}\ket{k}\bra{j}\right) \\ 
    = \frac{1}{N} \sum_g \sum_{lmkj} [U_g]_{lm} X_{mk} [U^{\dagger}]_{kj} \ket{l}\bra{j}$ then apply orthogonality \\
    $= \frac{1}{n_a} \sum_{lmkj} \delta_{lj} \delta_{mk} X_{mk} \ket{l}\bra{j} $

    \item $\braket{X}_G = \int_G \ud \mu(g) U_x(g) X U_x(g)^{\dagger} = \frac{1}{d} \Tr[X] \id$ for continous groups
    \item For reducible representations $U_g$ we have \\
    $\braket{X}_G = \frac{1}{N} \sum_g U_g X U_g^{\dagger} = \sum_x \dfrac{\Tr[\Pi_x X]}{d_x} \Pi_x = \bigoplus_x \dfrac{\Tr[\Pi_x X]}{d_x} \id_x$
    \item Proof: any reducib. unitary can be writt. as $U(g) = \bigoplus_x U_x(g) = \sum_x U_x(g) \otimes I_{\bar{x}}$ where $\bar{x}$ is the subspace $U_x$ does not act on. \\ $\braket{X}_G = \frac{1}{N} \sum_g U_g X U_g^{\dagger} = \frac{1}{N} \sum_g \sum_{xx'}(U_x(g) \otimes I_{\bar{x}}) X (U_{x'}(g)^{\dagger} \otimes I_{\bar{x}}) = 
    \frac{1}{N} \sum_g \sum_{x}(U_x(g) \otimes I_{\bar{x}}) X (U_{x}(g)^{\dagger} \otimes I_{\bar{x}}) = 
    \frac{1}{d_x} \sum_x \Tr [ X \Pi_x] \Pi_x \otimes I_{\bar{x}} =
    \frac{1}{d_x} \bigoplus_x \Tr [ X \Pi_x] \Pi_x$

    \columnbreak

    \item In a representation $R$, all elements which are in the same conjugacy class have the same trace. Proof: $\Tr(R(u^{-1}y u)) = \Tr(R(u)R(u^{-1})R(y)) = \Tr R(y)$
    \item \textbf{Petit Orthogonality Theorem}: Let $R_a$ and $R_b$ be two non-equiv. unitary irreducible reps of a finite group of order $N$, then $\sum_{g \in G} \chi^*_a(g) \chi_b(g) = N \delta_{a,b}$ \\ 
    where $\chi_R(g) = \Tr[R(g)]$, or equiv. $\sum_{\mu =1}^{N_c} \eta_{\mu}\chi^*_a(g) \chi_b(g) = N \delta_{a,b}$ with $\eta_{\mu}$ the nb of elements in class $\mu$ and $N_c$ total nb of conjugacy classes

    \item The set of all traces $\{\chi_R(g)\}$ is the \textbf{character} of representation $R$. Two irreps are equivalent iff they have the same character.
    Proof by contradiction with Petit.

    \item Trace of all reps within a conj. class are the same $\rightarrow$ $\chi_R(C_{\mu}) = \sum_a b_a \chi_a ( C_{\mu})$
    \item Assuming a decomposition in irreps $R(g) = \bigoplus_{a,x}R_{a,x}(g)$ for $x = 1 \ldots b_a$, the degeneracy of conjugacy class $a$ is $b_a = \frac{1}{N} \sum_{\mu} \eta_{\mu} \chi^*_a(C_{\mu})\chi_R(C_{\mu})$ NOT CLEAR
    
    \item A necessary and sufficient condition for a representation $R$ to be an irrep is that \\
    $\sum_{\mu = 1}^{N_c} \eta_{\mu} |\chi(C_{\mu})|^2 = N$. Proof: decompose trace and use Petit.
\end{squishlist}


\graypar{Lie Algebras}
\begin{squishlist}
    \item Rotation through infinitesimal angle $R(\theta) = \id + A$ and as $R^T R = \id$ we must have $A^T = -A$
    \item $\cdots$
\end{squishlist}

\graypar{Ladder operators}
\begin{squishlist}
    \item $J_+ \ket{m} = \sqrt{(j + 1 + m) (j-m)} \ket{m + 1}$
    \item $J_- \ket{m} = \sqrt{(j+1 - m) (j+m)} \ket{m-1}$
    \item Clebsch-Gordan stuff
    \item Any representation of $SO(3)$ is also a representation of $SU(2)$
\end{squishlist}

\columnbreak
\graypar{Irreducible representations of C3v}
\begin{squishlist}
    \item[(i)]
    Trivial \qquad  (ii) 2D irrep $\downarrow$ \\
    $R(e) = \begin{pmatrix} 1 & 0 \\ 0 & 1 \end{pmatrix}
    R(c_+) = \begin{pmatrix} -\frac{1}{2} & -\frac{\sqrt{3}}{2} \\\frac{\sqrt{3}}{2} & -\frac{1}{2} \end{pmatrix}
    R(c_+) = \begin{pmatrix} -\frac{1}{2} & \frac{\sqrt{3}}{2} \\-\frac{\sqrt{3}}{2} & -\frac{1}{2} \end{pmatrix} \\
    R(\sigma) = \begin{pmatrix} -1 & 0 \\ 0 & 1 \end{pmatrix}
    R(\sigma') = \begin{pmatrix} \frac{1}{2} & \frac{\sqrt{3}}{2} \\ \frac{\sqrt{3}}{2} & -\frac{1}{2} \end{pmatrix}
    R(\sigma'') = \begin{pmatrix} \frac{1}{2} & -\frac{\sqrt{3}}{2} \\ -\frac{\sqrt{3}}{2} & -\frac{1}{2} \end{pmatrix}$

    \item[(iii)] $R(e)=1,R(c_+)=1,R(c_-)=1,R(\sigma)=-1,R(\sigma')=-1,R(\sigma'')=-1$
    \item Character table:
    $\begin{array}{c | ccc}
         & e & 2C_3 & 3\sigma_v \\
         \hline
        R_{(i)} & 1 & 1 & 1 \\
        R_{(ii)} & 2 & -1 & 0 \\
        R_{(iii)} & 1 & 1 & -1    
    \end{array}$
\end{squishlist}

\graypar{Other groups}
\begin{squishlist}
    \item $SU(2)$: single qubit rotations
\item $U(1)$: symmetry group of rotations around one axis (e.g. $x$). The associated representation is the set of rotations by an angle $\theta \in [0 , 2\pi)$ about the axis $U(\theta) = e^{-i\theta \sigma_x / 2}$. The representation is reducible in 1D reps $\{1\}$ and $\{ e^{-i\theta}\}$. For example for $\sigma_x$ $U_g = \begin{pmatrix} 1 & 0 \\ 0 & e^{-i \theta} \end{pmatrix} = \ket{+} \bra{+} + e^{-i\theta}\ket{-} \bra{-}$ \\
Averaging a state $\rho$: $\braket{\rho}_G = \braket{+|\rho|+}\ket{+}\bra{+} + \braket{-|\rho|-}\ket{-}\bra{-}$ which is a projection onto the $x$-axis
\end{squishlist}


    \vfill
\today
