Matteo Veneziano -- Quantum Physics II Formula Sheet

\graypar{General single qubit state}
\begin{squishlist}
    \item $\ket{\psi} = \alpha \ket{0} + \beta \ket{1}$, or $\ket{\psi} = \cos(\theta/2)\ket{0} + e^{i\phi} \sin(\theta/2)\ket{1}$, $|\alpha|^2 + |\beta|^2 = 1$
    \item $\ket{\psi}$ is an eigenstate of $\vec{\sigma}\cdot \vec{n}$ with eigenvalue 1
\end{squishlist}

\graypar{Evolution}
\begin{squishlist}
    \item $i \D{t}{\ket{\psi(t)}} = H \ket{\psi(t)}$ (Schrödinger)
    \item if $H$ time-independant then $\ket{\psi(t)} = U(t) \ket{\psi(0)}$, $U(t) = e^{-iHt}$
\end{squishlist}

\graypar{Pauli matrices}
\begin{squishlist}
    \item $\sigma_0 = \begin{pmatrix} 1 & 0\\ 0 & 1 \end{pmatrix},
        \sigma_x = \begin{pmatrix} 0 & 1 \\ 1 & 0 \end{pmatrix}, 
        \sigma_y=\begin{pmatrix} 0 & -i\\ i & 0\end{pmatrix} , 
        \sigma_z = \begin{pmatrix} 1 & 0 \\ 0 & -1\end{pmatrix}$

    \item $\Tr[\sigma_0] = 2, \Tr[\sigma_i] = 0$ for $i=1,2,3$ \hfill $S_i = \frac{\hbar}{2} \sigma_i$
    \item $\sigma_i \sigma_j = \delta_{ij} \id + i \epsilon_{ijk} \sigma_k$ \quad $[\sigma_i, \sigma_j] = 2i \epsilon_{ijk} \sigma_k$ \quad $V = \vec{v}\cdot \vec{\sigma} \Rightarrow V^2 = \id$
    \item $\{\sigma_i, \sigma_j \} = \sigma_i \sigma_J + \sigma_j\sigma_i = 2 \delta_{ij} \id$
    \item Paulis form an orthonormal basis with $\Tr[\sigma_i,\sigma_j] = 2\delta_{ij}$
    \item The Pauli $\alpha$ operator, $\alpha \in \{x,y,z\}$, rotates the state by $\pi$ around the $\alpha$-axis.

    \item Paulis as generators: any single qubit hamiltonian can be written as $H = \omega \vec{n}\cdot \vec{\sigma}$ \\
        $U = \exp(-i\omega \vec{\sigma}\cdot \vec{n} t) = \cos(\omega t) \id - i \sin(\omega t) \vec{n}\cdot\vec{\sigma}$ \\
        Causes a qubit state to rotate around $\vec{n}$ at a rate $2\omega t$.
\end{squishlist}

\graypar{Observables}
\begin{squishlist}
    \item $M = \sum_k \lambda_k \ket{\lambda_k} \bra{\lambda_k}$ Hermitian
    \item $\braket{M} = \braket{\psi|M|\psi} = \sum_k \lambda_k P_k$, $P_k = |\braket{\lambda_k | \psi}|^2$
    \item $\Pi_k = \ket{\lambda_k}\bra{\lambda_k}$ Projector $\Rightarrow P_k = \braket{\psi| \Pi_k | \psi}$, $\sum_k \Pi_k = 1$
    \item Measurement $\Rightarrow$ State collapses to $\frac{\Pi_k \ket{\psi_k}}{\sqrt{P_k}}$
\end{squishlist}

\graypar{Composite systems \quad $\H_{ABC\ldots} = \H_a \otimes \H_b \otimes \H_c \ldots$}
\begin{squishlist}
    \item The resulting space has dimension $n_A n_B n_C \ldots$
    \item Operators $T_{AB} \ket{\lambda_ij} = (T_a \otimes T_b) (\ket{\mu_i \otimes \nu_j}) = T_A \ket{\mu_i} \otimes \ket{\nu_k}$
    \item $[T_A \otimes \id_B, \id_A \otimes T_B] = 0$, \quad $\{ T_A \otimes \id_B, \id_A \otimes T_B\} = 2 (T_A \otimes T_B)$ \\
        $e^{A \otimes \id + \id \otimes B} = e^A \otimes e^B$

    \item Global measurem.\ $T = T_A \otimes T_B \rightarrow T \ket{T_i} = t_i \ket{T_i}, \quad \ket{\psi} = \sum_i \ket{T_i} \bra{T_i}\ket{\psi}$ 
    \item Partial measurement $T_A$: if $\kpsi = \sum_{ij} \ket{T_{A,i}} \otimes \ket{T_{B,j}} = \sum_i \ket{T_{A,i}} \otimes \ket{\phi_{B,i}}$
        then $P_i = \sum_j |c_{ij}|^2$ and system collapses to $\kpsi' \propto \ket{T_{A,i}} \otimes \ket{\phi_{B,i}}$

    \item $P_{\hat{O}}(\lambda | \psi) = \braket{\psi | \Pi^{A}_{\lambda} | \psi }$ is the probability of a measurement of operator $\hat{O}$ yielding its eigenvalue $\lambda$, with $A$ the subspace of the meas., $\Pi_{\lambda}^A = \ket{\lambda}\bra{\lambda}_A \otimes \id_B$ and
    $\ket{\lambda}$ the eigenket corresponding to eigenvalue $\lambda$

    For measurements on both subspaces use $\Pi_{\lambda} = \ket{\lambda}\bra{\lambda}_A \otimes \ket{\kappa}\bra{\kappa}_B$

    \item \emph{Entangled} state: its coefficients cannot be written as the product of two independent coefficients. 
        \emph{Separable} state: the global wave function can be written as the product of two wavefunctions corresponding to subsystems $A$ and $B$ (measures performed on one part do not affect the other).
    \item Condition of separability for 2 qubits: if $c_{ij} = c^{(A)}_i c^{(B)}_j$, with $i,j\in \{0,1\}$, then
        $\mathrm{det} \begin{pmatrix} c_{00} & c_{01} \\ c_{10} & c_{11} \end{pmatrix} = 0 \Leftrightarrow $ separable. E.g. in $\left(\alpha \ket{0}_A \otimes \beta \ket{0}_B\right)$, $c_{00} = \alpha \beta$.
    \item $H_{AB} = H_A \otimes \id_B + \id_A \otimes H_B \Longrightarrow e^{-itH_{AB}} = e^{-itH_A} \otimes e^{-ith_B}$ \\ A separable unitary generates no entanglement when applied to a separable state.
\end{squishlist}

\columnbreak

\graypar{Quantum eraser}
\begin{squishlist}
    \item $\ket{\nearrow} = \frac{1}{\sqrt{2}}(\ket{H}+\ket{V}),\: \ket{\swarrow} = \frac{1}{\sqrt{2}}(\ket{H} - \ket{V})$
    \item $\ket{H} = \frac{1}{\sqrt{2}}(\ket{\nearrow} + \ket{\swarrow}),\: \ket{V} = \frac{1}{\sqrt{2}}(\ket{\nearrow} - \ket{\swarrow})$
    \item $P(x) = \braket{\psi(x,t)|\left(\ket{x}\bra{x} \otimes \id\right)| \psi(x,t)}$ Probability density on screen
\end{squishlist}


\graypar{Bell states}
\begin{squishlist}
    \item $\ket{\Phi^+} = \frac{1}{\sqrt{2}} \left( \ket{0}_A \otimes \ket{0}_B + \ket{1}_A \otimes \ket{1}_B\right)$ \\
          $\ket{\Phi^-} = \frac{1}{\sqrt{2}} \left( \ket{0}_A \otimes \ket{0}_B - \ket{1}_A \otimes \ket{1}_B\right)$ \\
          $\ket{\Psi^+} = \frac{1}{\sqrt{2}} \left( \ket{0}_A \otimes \ket{1}_B + \ket{1}_A \otimes \ket{0}_B\right)$ \\
          $\ket{\Psi^-} = \frac{1}{\sqrt{2}} \left( \ket{0}_A \otimes \ket{1}_B - \ket{1}_A \otimes \ket{0}_B\right)$
    \item They form an orthonormal basis of the Hilbert space of the two spins $\H = \H_A \otimes \H_B$
    \item Non separable
    \item Eigenstates of $\hat{H} = \mu_x \hat{S}_x^{(A)} \otimes \hat{S}_x^{(B)} + \mu_y \hat{S}_y^{(A)} \otimes \hat{S}_y^{(B)}$
    \item $P_{\hat{S}_x^{(A)}}(\pm \frac{\hbar}{2} | \psi) = \frac{1}{2} \quad \forall \ket{\psi} $ a Bell state. 
\end{squishlist}


\graypar{CHSH Inequality}
\begin{squishlist}
    \item Bipartite system with LHS measuring device, which can measure either $A$ or $A'$, and RHS device which can measure $B$ or $B'$.
    The probability of a result combination is written as $P(l,r|L,R)$ with $L,R$ the settings on LHS and RHS device and $l$ and $r$ the results of the measures ($\pm 1$).
    \item Bell inequalities define a correlation coefficient $C$ and then place an upper bound on possible values this coefficient can take if you assume factorisability
    \item {Factorisability}: $p(l,r|L,R) = \int P(l|L,\lambda) P(r|R,\lambda) P(\lambda) d\lambda$ where $\lambda$ incorporates all effects from the system's shared history.
    Two necessary conditions for factorisability to hold: Setting Independence $P(l|L,B,\lambda) = P(l|L,B',\lambda)$ and Outcome Indipendence $P(l,A,R,r,\lambda) = P(l|A,R,r',\lambda)$.
    \item $C:= |\langle LR \rangle - \langle LR'\rangle  | + |\langle LR\rangle + \langle L'R \rangle |$ with $\langle LR \rangle = \sum_{l,r=\pm 1}lr P(l,r|L,R)$
    \item $C \leq 2$ in the classical case, violated in quantum case (Tsirelson's bound: $C \leq 2\sqrt{2}$) \\ Quantum Mechanics violates outcome indipendence.
\end{squishlist}


\graypar{Reduced and mixed quantum states}
\begin{squishlist}
    \item Density operator: $\rho = \kpsi \bra{\psi} \Longrightarrow \braket{O} = \Tr (\rho O)$ for any observable $O$
\item General single qubit: $\rho = \begin{pmatrix} cos(\theta/2)^2 & \cos(\theta/2)\sin(\theta/2)e^{-i\phi} \\ \cos(\theta/2)\sin(\theta/2)e^{i\phi} & \sin(\theta/2)^2 \end{pmatrix} = $ \\
$= \frac{1}{2}\sigma_0 + \frac{1}{2}\sum_{i=1}^3 v_i \sigma_i$ with $\vec{v} = (\sin\theta \cos\phi , \sin\theta\sin\phi, \cos\theta);\; v_i = \Tr[\rho \sigma_i]$
\item $\rho = \sum_k p_k \ket{\psi_k}\bra{\psi_k}$ System prepared in state $\ket{\psi_k}$ with prob. $p_k$ (\textbf{mixed} state)
\item If $\rho = p \dpsi + (1-p) \ket{\phi}\bra{\phi}$ where
$\psi, \phi$ have Bloch vectors $\vec{v}, \vec{u}$ of pure states, the mixed state has Bloch vector $\vec{w} = p \vec{v} + (1-p) \vec{u}$; $|\vec{w}|^2 \leq 1$
\item $\rho_A = \sum_{k=1}^{d_B}(\id_A \otimes \bra{k})\rho_{AB}(\id_A \otimes \ket{k}) = \Tr_B[\rho_{AB}]$ \quad \textbf{Reduced} state
\item $\Tr_B[\ket{ij}\bra{kl}] = \ket{i}\bra{k} \Tr[\ket{j}\bra{l}]$
\item \textbf{Props}: 
(i) $\rho_A^{\dagger} = \rho_A$ (self-adj.) \qquad (ii)$\Tr(\rho_A) = \sum_i  \sum_{\mu}\alpha^*_{i,\mu}\alpha_{i,\mu} = |\psi|^2 = 1$ \\ (iii)$\braket{\psi|\rho_A|\psi} \geq 0$ for all $\kpsi \in A$ i.e. positive or null eigenvalues
\item Conseq., $\rho_A = \sum_j p_j \ket{k}\bra{j}$ where $p_j\geq 0$ and $\sum p_j = 1$.\\ $\braket{O} = \Tr(\rho_A O) = \sum_j p_j \braket{j|O|j} = \sum_{p_j} \braket{O}_{\ket{j}}$
\item If a density op. describes a pure state, then it is a projector ($\rho^2 = \rho$). If $\rho$ is not pure then $\rho^2 \neq \rho$ and $\Tr[\rho^2] < 1$ (\emph{purity} of a state).
\item \textbf{Evolution:} $\rho(t=0) = \sum_j \alpha_j \ket{\psi_j(0)}\bra{\psi_j(0)}$ initial state \\ $\Longrightarrow \rho(t) = \sum_j \alpha_j e^{-iHt} \ket{\psi_j(0)}\bra{\psi_j(0)}e^{iHt} \Longrightarrow i \D{t}{\rho} = [\hat{H},\rho]$
\item Why is signaling impossible? No matter what is performed upon the other partitions, the reduced density matrix is unchanged. Because the statistics of local measurements are informed entirely by expected values of operators upon the reduced density matrix, they are also independent of operations on other partitions.
\end{squishlist}

\graysec{Identical multi-particle systems}
\begin{squishlist}
    \item $\P_{1,2} \psi(\vec{r_1}, \vec{r_2}) = \psi(\vec{r_2}, \vec{r_1}) = \pm \psi(\vec{r_1}, \vec{r_2})$ \quad $+1$: bosons \quad $-1$: fermions
    \item $\P_{jk} = \P_{kj}$ \squishsep $\P_{jk}^2 = \id \Leftrightarrow \P_{jk}^{-1}=\P_{jk}$ \squishsep $\P_{jk} = \P_{jk}^{\dagger}$
    \item $\braket{\psi_{12}|O|\psi_{12}} = \braket{\psi_{12}|\P_{12}^{\dagger}O\P_{12}|\psi_{12}}$ for all $\psi$ $\Rightarrow [\P_{12},O] = 0, [\P_{12},H] = 0$
    \item Possible basis states for a system of \\ $n$ bosons: $\ket{\psi_{\vec{x}}} = \mathcal{N} \sum_{\P \in S_n} \P\ket{x_1,x_2, \ldots, x_n}$ with $\mathcal{N} = \frac{1}{\sqrt{n!}\sqrt{\prod_k n_k !}}$ \\
    $n$ fermions $\ket{\psi_{\vec{x}}} = \frac{1}{\sqrt{n!}} \sum_{\P \in S_n} \text{sign}(\P) \P \ket{x_1, x_2, \ldots, x_n}$
\end{squishlist}

\graypar{Second Quantization}
\begin{squishlist}
    \item Kets indicate the number of times a wave function is involved:\\
    for Bosons $\frac{1}{\sqrt{2}} ( \ket{\uparrow \downarrow} + \ket{\downarrow \uparrow}) \rightarrow \ket{11};\; \ket{\uparrow \uparrow} \rightarrow \ket{20};\; \ket{\downarrow \downarrow} \rightarrow \ket{02}$\\
    for Fermions $\frac{1}{\sqrt{2}} ( \ket{\uparrow \downarrow} - \ket{\downarrow \uparrow}) \rightarrow \ket{11}$ (only possible values 0,1)
    \item Creation and annihilation operators to increase or decrease the number of particles:
    Bosons:  $\left\{ \begin{aligned}
    \hat{c}_i^{\dagger} \ket{n_1, \cdots, n_i, \cdots} &= \sqrt{n_i + 1} \ket{n_1, \cdots, n_i +1, \cdots} \\
    \hat{c}_i \ket{n_1, \cdots, n_i, \cdots} &= \sqrt{n_i} \ket{n_1, \cdots, n_i -1, \cdots}
    \end{aligned}\right.$ \\
    $[\hat{c}_i, \hat{c}_j] = [\hat{c}_i^{\dagger}, \hat{c}_j^{\dagger}] = 0$; \qquad $[\hat{c}_i, \hat{c}_j^{\dagger}] = \delta_{ij}$ \\
    Fermions: $\left\{ \begin{aligned}
    \hat{c}_i^{\dagger} \ket{n_1, \cdots, n_i, \cdots} &= (-1)^{n_1+\cdots+n_i-1}(1-n_i) \ket{n_1, \cdots, n_i +1, \cdots} \\
    \hat{c}_i \ket{n_1, \cdots, n_i, \cdots} &= (-1)^{n_1+\cdots+n_i-1} n_i \ket{n_1, \cdots, n_i -1, \cdots}
    \end{aligned}\right.$ \\
    $\{\hat{c}_i, \hat{c}_j\} = \{\hat{c}_i^{\dagger}, \hat{c}_j^{\dagger}\} = 0$; \qquad $\{\hat{c}_i, \hat{c}_j^{\dagger}\} = \delta_{ij}$ 
\end{squishlist}


\graysec{Perturbation Theory}
\begin{squishlist}
    \item $H = H_0 + \lambda V$, with $H_0 \ket{\phi_n} = \epsilon_n \ket{\phi_n}$ known,\, $\lambda \in \mathbb{R}^+$. 
    \item $H \ket{\psi_n} = E_n \ket{\psi_n}$ eigenspectrum unknown. The solution in the limit of small $\lambda$ is 
    $\ket{\psi_n} = \ket{\phi_n} + \lambda \ket{\psi_n^{(1)}} + \lambda^2 \ket{\psi_n^{(2)}} + \ldots$ \\
    $E_n = \epsilon_n + \lambda E_n^{(1)} + \lambda^2 E_n^{(2)} + \ldots$
    \item S.E.: $(H_0 + \lambda V)(\ket{\phi_n} + \lambda \ket{\psi_n^{(1)}} + \lambda^2 \ket{\psi_n^{(2)}} + \ldots) = (\epsilon_n + \lambda E_n^{(1)} + \lambda^2 E_n^{(2)} + \ldots)(\ket{\phi_n} + \lambda \ket{\psi_n^{(1)}} + \lambda^2 \ket{\psi_n^{(2)}} + \ldots)$ must be satisfied at each order in $\lambda$
\end{squishlist}
\graypar{Non-degenerate Time-Indipendent Perturbation Theory}
\begin{squishlist}
    \item \textbf{Zero-th order} $H_0 \ket{\phi_n} = \epsilon_n \ket{\phi_n}$ unperturbed eigenvalue problem
    \item \textbf{1st order} $E_n^{(1)} = \braket{\phi_m|V|\phi_n}$; \quad $\ket{\psi_n^{1}} = \sum_{m\neq n}\frac{\braket{\phi_m |V |\phi_n}}{\epsilon_n - \epsilon_m} \ket{\phi_m}$
    \item \textbf{2nd order} $E_n^{(2)} = \frac{|\braket{\phi_m|V|\phi_n}|^2}{\epsilon_n - \epsilon_m}$; for the approx. to be valid  $|E_n^{(2)}| \ll |E_n^{(1)}|$ 
    \item satisfied as long as $\frac{1}{\Delta} ( \braket{\phi_n | V^2|\phi_n} - \braket{\phi_n | V| \phi_n}^2) \ll \braket{\phi_n | V | \phi_n}$, \\ 
    $\Delta = \min_m | \epsilon_n - \epsilon_m|$ or more restrictive: $\left|\frac{\braket{\phi_m | V | \phi_n}}{\epsilon_n - \epsilon_m}\right| \ll 1$
\end{squishlist}



\graypar{Hermitian and unitary operators}
\begin{squishlist}
    \item Hermitian operator: $M = M^{\dagger}$ $\longrightarrow$ diagonalisable with real eigenvalues, linear
    \item Unitary operators: $U U^{\dagger} = U^{\dagger}U = \id$ $\longrightarrow \braket{\psi | U^{\dagger}U|\psi} = \id $ and linear
\end{squishlist}

\graypar{Tensor product}
\begin{squishlist}
\item $\vectwo{a}{b} \otimes \vectwo{\alpha}{\beta} = \begin{pmatrix} a \alpha & a \beta & b \alpha & b \beta \end{pmatrix}^T$
    
\item $A \otimes B = \begin{pmatrix} A_{11}B & A_{12}B \\ A_{21}B & A_{22}B \end{pmatrix}$
\item $f(\hat{A} \otimes \hat{B}) \ket{a}\ket{\phi} = \ket{a} \otimes f(a\hat{B})\ket{\phi}$
\end{squishlist}

\graypar{Different basis}

\graysec{Trigonometry}
\begin{squishlist}
    
    \item $\cos(a+b) = \cos(a)\cos(b)-\sin(a)\sin(b)$ \squishsep $\sin(2a) = 2\sin a \cos a$
    \item $\sin(a+b) = \sin(a)\cos(b) + \cos(a)\sin(b)$ \squishsep $\cos(2a) = \cos^2 a - \sin^2 a$
    \item $2\sin(a)\sin(b) = \cos(a-b) - \cos(a+b)$ \smallsquishsep $\sin(a/2) = \pm \sqrt{(1-\cos(a))/2}$
    \item $2\cos(a)\cos(b) = \cos(a+b) + \cos(a-b)$ \smallsquishsep $\cos(a/2) = \pm \sqrt{(1+\cos(a))/2}$
    \item $2\cos(a)\sin(b) = \sin(a+b) - \sin(a-b)$ \smallsquishsep $\sin^2a = (1-\cos 2a)/2$
    \item $\cos(a) + \cos(b) = 2\cos\left((a+b)/2\right)\cos\left((a-b)/2\right)$ \;\squishitem \; $\cos^2 a = (1+\cos 2a) / 2$
    \item $\sin(a) + \sin(b) = 2\sin\left((a+b)/2\right)\cos\left((a-b)/2\right)$ 
    \item $\cos(a) - \cos(b) = -2\sin\left((a+b/2\right)\sin\left((a-b/2\right)$ 
\end{squishlist} 

% \graypar{Formules d'Euler + Nombres complexes}
% $\cos(x) = \frac{e^{ix}+e^{-ix}}{2}$ \,,\, $\sin(x) = \frac{e^{ix}-e^{-ix}}{2i}$ \,,\, $\cosh(x) = \frac{e^{x}+e^{-x}}{2}$   
% $\sinh(x) = \frac{e^{x}-e^{-x}}{2}$ \,,\, $e^{i(\pi + 2k\pi)} = -1$ \,,\, $e^{i\cdot 2k\pi} = 1$ \,,\, $k \in \mathbb{Z}$  
% $e^{i(\frac{\pi}{2} + 2k\pi)} = i$ \,,\, $e^{i(-\frac{\pi}{2} + 2k\pi)} = -i$ \,,\, $k \in \mathbb{Z}$ \,,\,
% $\sqrt{i} = \frac{1+i}{\sqrt{2}}$% \,,\, $\frac{1}{i} = -i$ 
% $w^n = z = re^{i\varphi}$ => $w_k = r^{\frac{1}{n}}e^{i\frac{\varphi + 2\pi k}{n}},\, k=0,1,...,n-1 $

% \item[] \textbf{Near 0 :}
% \item $\sin(x) \simeq x - \frac{x^{3}}{6}$ \, ; \, $\cos(x) \simeq 1 - \frac{x^{2}}{2}$ \, ; \, $\tan(x) \simeq x + \frac{x^{3}}{3}$
% %\item $\cot(x) \simeq \frac{1}{x} - \frac{x}{3} - \frac{x^{3}}{45}$, $\sec(x) \simeq 1 + \frac{x^{2}}{2} + \frac{5x^{4}}{24}$
% \item $\sinh(x) \simeq x + \frac{x^{3}}{6}$  \,;\, $\cosh(x) \simeq 1 + \frac{x^{2}}{2}$   \,;\,$\tanh(x) \simeq x - \frac{x^{3}}{3}$
% %\item $\coth(x) \simeq \frac{1}{x} + \frac{x}{3} - \frac{x^{3}}{45}$ \, ; \, ${\rm sech}(x) \simeq 1 - \frac{x^{2}}{2} + \frac{5x^{4}}{24}$  
% \item $\left(1+x \right)^{\alpha} \simeq 1 + \alpha x + \frac{\alpha(\alpha -1)}{2}x^2$ \, ; \, $\ln(1+x) \simeq x - \frac{x^2}{2} $ 
% \item[] \textbf{Residuals theorem}
% \item $\displaystyle{\oint_{\partial D}} \ud z \, f(z) = 2\pi i \displaystyle{\sum_{k}} \text{Res}(f,z_k)$ \, : \, with $z_k$ singularities inside $\partial D$  %Pas de pts siguliers -> Holomorphe

% \item $\text{Res}(f,z_k) = \dfrac{1}{(n-1)!} \displaystyle{\lim_{z \rightarrow z_k}} \dfrac{\ud ^{n-1}}{\ud z^{n-1}} (z-z_k)^{n} f(z)$ \, : \, If $z_k$ poles of order n  

\columnbreak

\graysec{Group Theory}
