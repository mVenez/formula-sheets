\definecolor{lightgray}{gray}{.9}
\definecolor{darkgray}{gray}{.8}
\newcommand{\squishitem}{$\star$}
\newcommand{\squishsep}{\hspace{0.5cm} \squishitem \;}
\newcommand{\smallsquishsep}{\hspace{0.25cm} \squishitem \;}


\newcommand{\graysec}[1]{\hspace{-0.305cm}\noindent\colorbox{darkgray}{\makebox[1.045\columnwidth][l]{\textbf{#1 \hfill \hfill \putpagenumber}}}

}
\newcommand{\graypar}[1]{\hspace{-0.305cm}\noindent\colorbox{lightgray}{\makebox[1.045\columnwidth][l]{\textbf{#1 \hfill \hfill \putpagenumber}}}

}

\newcommand{\E}{\mathcal{E}}
\renewcommand{\H}{\mathcal{H}}
\renewcommand{\P}{\mathbb{P}}


\newenvironment{squishlist}{\begin{list}{\squishitem}{
		\setlength{\itemsep}{5pt}
		\setlength{\parsep}{0pt}
		\setlength{\topsep}{0pt}
		\setlength{\partopsep}{0pt}
		\setlength{\leftmargin}{0em}
		\setlength{\labelwidth}{1em}
		\setlength{\labelsep}{0.5em}}
	}{\end{list}}

\newenvironment{squishitemize}{\begin{list}{$\bullet$}{
	\setlength{\itemsep}{5pt}
	\setlength{\parsep}{0pt}
	\setlength{\topsep}{0pt}
	\setlength{\partopsep}{0pt}
	\setlength{\leftmargin}{1em}
	\setlength{\labelwidth}{1em}
	\setlength{\labelsep}{0.5em}}
}{\end{list}}

\newcommand{\ud}{\textrm{d}}
\newcommand{\ugrad}{\nabla}
\newcommand{\udiv}{\nabla \cdot}
\newcommand{\urot}{\nabla \cross}

\newcommand{\ulap}{\ensuremath{\nabla^2}}

\newcommand{\tq}{\ensuremath{\ \textrm{tq.} \ }}

\newcommand{\der}[2]{\ensuremath{\frac{\ud #2}{\ud #1}}}
\newcommand{\dder}[2]{\ensuremath{\frac{\ud^2 #2}{\ud #1^2}}}
\newcommand{\D}[2]{\ensuremath{\frac{\partial #2}{\partial #1}}}
\newcommand{\DD}[2]{\ensuremath{\frac{\partial^2 #2}{\partial #1^2}}}

\renewcommand{\vec}[1]{\mbox{\boldmath$#1$}}
\newcommand{\univec}[1]{\ensuremath{\hat{\vec #1}}}
\newcommand{\scalar}[2]{\ensuremath{\vec{#1} \cdot \vec{#2}}}
%\newcommand{\cross}[2]{\ensuremath{\vec{#1} \wedge \vec{#2}}}

\newcommand{\emc}[1]{\ensuremath{\frac{#1}{4 \pi \varepsilon_0}}}

\newcommand{\N}{\hat{\vec n}}
\newcommand{\mean}[1]{\langle#1\rangle}
\newcommand{\half}{\frac{1}{2}}

% redefined sqrt to make it "closed" root
\usepackage{letltxmacro}
\makeatletter
\let\oldr@@t\r@@t
\def\r@@t#1#2{%
\setbox0=\hbox{\ensuremath{\oldr@@t#1{#2\,}}}\dimen0=\ht0
\advance\dimen0-0.2\ht0
\setbox2=\hbox{\vrule height\ht0 depth -\dimen0}%
{\box0\lower0.4pt\box2}}
\LetLtxMacro{\oldsqrt}{\sqrt}
\renewcommand*{\sqrt}[2][\ ]{\oldsqrt[#1]{#2}}
\makeatother

\def\pa{\partial}

\newcommand{\pad}[2]{\frac{\pa #1}{\pa #2}}
%\newcommand{\padp}[3]{\frac{\pa^{#3} #1}{\pa #2^{#3}}}

%\newcommand{\bigO}[1]{\mathcal{O}\left( #1\right) }
%\newcommand{\smallo}[1]{\mathcal{o}\left( #1\right) }

%\newcommand{\iif}{\Leftrightarrow}
\newcommand{\transposeAfter}{^{\mathsf{T}}}
\newcommand{\transpose}[1]{#1\transposeAfter}
\newcommand{\complexConjugate}[1]{#1^{\star}}
\newcommand{\oneover}[1]{\frac{1}{#1}}

\newcommand{\id}{\mathbb{I}}
\newcommand{\Tr}{\mathrm{Tr}}

\newcommand{\vectwo}[2]{\begin{pmatrix} #1 \\ #2  \end{pmatrix}}

\newcommand{\kpsi}{\ket{\psi}}
\newcommand{\dpsi}{\ket{\psi}\bra{\psi}}
\newcommand{\densop}[1]{\ket{#1}\bra{#1}}