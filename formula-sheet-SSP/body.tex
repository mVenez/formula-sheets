% Matteo Veneziano -- Solid State Physics Formula Sheet
\graysec{Réseaux cristallins dans l'éspace réel et réciproque}
\begin{squishlist}
    \item Énergie électrost.\ $E = - M_d \frac{e^2}{4\pi \varepsilon_0 a}$ \textbf{Cst de Madelung} specif.\ pour une struct.\ donnée
    \item \textbf{Liaison covalente}: partage d'\elec entre les atomes, favorable en présence d'orbitales de valence partiell.\ remplies. E.g.\ diamant, hybridisation $sp^3$ $\Rightarrow$ nb de coordin.\ 4
    \item Taux de remplissage $t$: rapp.\ entre le volume occupé par les atomes (assimilés à des sphères dures de rayon $r$) et le volume de la maille considérée
\end{squishlist}
\graypar{Réseaux cristallins}
\begin{squishlist}
    \item $\vec{R} = \sum_{j=1}^{3} n_j \vec{a}_j$ un vecteur du \textbf{réseau de Bravais (RB)} (réseau de points)
    \item Une \textbf{cellulle (ou maille) primitve} est un volume d'espace qui, translaté par les vecteurs $\vR$ du RB, remplit exactement l'espace sans recouvrement. Ne contient qu'un point du RB, donc a volume $v = 1/n$ avec $n$ la densité de points du RB.
    \item La \textbf{cellule de Wigner-Seitz} est la région de l'espace qui est plus proche à un point du RB que de n'importe quel autre point. On l'obtient en tracant les plans bissecteurs des ligne qui connectent les points.
    \item Une \textbf{structure cristalline} est un RB avec une base formée de un ou plus atomes
    % \begin{minipage}{.25\columnwidth}
        \includegraphics[width=0.6\columnwidth]{figures/structure-exemple.png}
    % \end{minipage}
    \item Le \textbf{réseau réciproque} (RR) est formée des $\vG$ t.q.\ $\exp(i \vG \cdot \vR) = 1$ \\
    $\vG = \sum_{i=1}^{3}h_i \vec{b}_i $\squishsep $\vec{a}_i \cdot \vec{b}_j = 2 \pi \delta_{ij}$ \squishsep $\vec{b}_1 = 2 \pi \frac{\vec{a}_2 \times \vec{a}_3}{\vec{a}_1 \cdot (\vec{a_2 \times \vec{a}_3})}$ et perm.\ cycl.\ 
    \item Le volume d'une cell.\ prim.\ du réseau réciproque est $\vec{b}_1 \cdot (\vec{b}_2 \times \vec{b}_3) = \frac{(2\pi)^3}{v} = \frac{(2\pi)^3}{V/N}$
\end{squishlist}    

\graypar{Zones de Brillouin}
\begin{squishlist}
    \item La \textbf{1ère zone de Brillouin (1ZDB)} est la cellule de Wigner-Seitz du réseau réciproque. La \textbf{nième zone de Brillouin (nZDB)} est l'ensemble des points atteints à partir de l'origine en traversant $(n-1)$ plans de Bragg
    \item Le volume de la nZDB est égal à celui de la 1ZDB. On peut par translations de vecteur $G$ réduire une ZDB à la 1ZDB (passage d'une description en schéma de zone étendue à une de zone réduite).
\end{squishlist}

\graypar{Détermination de la structure cristalline}
Techn.\ expérim.\ pour déterminer la struct.\ cristalline: diffraction de rayons X.
\begin{squishlist}
    \item \textbf{Théorie de Bragg:} les rayons X sont réfléchis par les plans atomiques du cristal. \textbf{Condition de Bragg:} $n \lambda = 2d \sin \Theta$, $n \in \mathbb{N}$ (interférence constructive d'ondes émerg.\ de plans successifs). $\lambda < 2d$. \\
    \textbf{Condition de Laue:} $\vec{K} = \vk'-\vk = \vG$

    \item \textbf{Indices de Miller:} $(h,k,l)$ les plans du réseau sont caractérisés par trois nombres entiers. On trouve les intersections avec les axes cristallographiques (en unités de vecteurs primitifs), ensuite on prend les valeurs réciproques de ces nombres.\\
    $\vG = h \vec{b}_1 + k \vec{b}_2 + l \vec{b}_3$ vec du RR perp.\ à la famille de plans, avec plus petite norme $G_{\min} = 2\pi / d \quad G = n G_{\min}$
\end{squishlist}

\graysec{Dynamique du réseau. Notion de phonon}
\begin{squishlist}
    \item \textbf{Approximation harmonique}
    \item \textbf{Approximation adiabatique:} les vitesses \elec sont de l'ordre de la vit.\ de Fermi ($\sim 10^8$ cm/sec) et la vitesse thermique des ions est plus faible $\Rightarrow$ les \elec suivent instantanément le mouvement des ions. Les \elec se trouvent tousjouts dans l'état fondamental correspondant à la configuration ionique.
    \item $u$ amplitude de déplacement
\end{squishlist}

\graypar{Modes normaux d'un RB monoatomique 1D}
\begin{squishlist}
    \item Chaine linéaire d'ions de masse $M$ séparés à l'ex.\ par $a$ ($R = na$)\\
    $U_{\text{harm}} = \frac{1}{4} \sum_{n,n'} C_{n,n'}(u_n - u_n')^2 \quad C_{n,n'} = C_{n',n} = \phi_{xx}(na - n'a)$
    \item Interactions entre plus proches voisins: les seuls coefs non nuls sont t.q.\ $n-n'= \pm 1$
    \item \textbf{Conditions de Born von Karman:} $u_{N+1} = u_1 \Longrightarrow U_{\text{harm}} = \frac{1}{2}\sum_{n=1}^{N}C(u_{n+1}-u_n)^2$
    \item $H(p,u) = \sum_{n=1}^{N}\frac{p^2_n}{2m} + \frac{C}{2}\sum_{n=1}^{N}(u_{n+1} - u_n)^2 \Longrightarrow \ddot{u}_n = -\frac{C}{m}(-u_{n-1}+2u_n - u_{n+1})$
    \item $u_n(t) = \frac{1}{\sqrt{N}} \sum_{\nu=0}^{N-1}a_{\nu} \exp[ i (k_{\nu}na - \omega_{\nu}t)] + c.c.$ \quad $k_{\nu} = \frac{2\pi \nu}{aN} \quad \nu = 0,1,\ldots, N-1$ 
    \item $\omega_{\nu} = 2 \sqrt{\frac{C}{m}} \left| \sin \frac{k_{\nu}a}{2}\right|$ \textbf{Courbe de dispersion} (linéaire pour $k$ faible par rapport à $\pi/a$)

    \squishline

    \textbf{3D}
    \item $u_{\alpha}(\vR) = \frac{1}{\sqrt{3N}}\sum_{\nu}\sum_{s=1}^{3} a_{\nu,s} \varepsilon_{\alpha,s}(\vk_{\nu}) \exp [i \vk_{\nu} \cdot \vR - i \omega_s(\vk_{\nu}t)] + c.c.$ \\ 
    Pour chacune des $N$ valeurs $\vk_{\nu}$ dans une cellule primitive, 3 directions $\vec{\varepsilon}_s$ de polarisation, donc $3N$ modes propres

    \squishline

    \textbf{non-monoatomique}
    \item Nb de valeurs $k$ dans 1ZDB $= pN$ où $p$ nb d'atomes dans la base, $N$ nb de mailles

\end{squishlist}

\graypar{Modes normaux d'un RB 1D avec une base, $N$ mailles}
\begin{squishlist}
    \item $U_{\text{harm}} = \frac{K}{2}\sum_{n=1}^{N}(u_n - v_n)^2 + \frac{G}{2} \sum_{n=1}^{N} (u_n - v_{n-1})^2$ \quad \textbf{2 ress.\ différents, même masse}
    \item $\left\{ \begin{aligned}
        & m\ddot{u}_n = -K (u_n - v_n) - G(u_n - v_{n-1}) \\
        & m\ddot{v}_n = K (u_n - v_n) + G(u_{n+1} - v_{n})
    \end{aligned}\right.
    \Rightarrow 
    \left\{ \begin{aligned}
        & u_n = \sum_{\nu = 0}^{N-1} a_{\nu} \exp (i k_{\nu}na - i\omega_{\nu}t) + c.c \\
        & v_n = \sum_{\nu = 0}^{N-1} b_{\nu} \exp (i k_{\nu}na - i\omega_{\nu}t) + c.c
    \end{aligned}\right.$

    \item $\omega_{\nu}^2 = \frac{K+G}{m} \pm \frac{1}{m}\sqrt{(K+G)^2 - 4KG \sin^2 \frac{k_{\nu}a}{2}} \qquad \frac{a_{\nu}}{b_{\nu}} = \mp \frac{K + G\exp(ik_{\nu}a)}{|K + G\exp(ik_{\nu}a)|}$
    \item Au bord de 1ZDB, si $K > G \quad \Longrightarrow \quad \omega_{\text{opt}}= \sqrt{\frac{2K}{M}} \quad \omega_{\text{acous}} = \sqrt{\frac{2G}{M}}$ \\
    \item Pour chacune des $N$ valeurs de $k$ il y a 2 solutions $\Rightarrow 2N$ modes normaux de vibrations. Branche inférieure (\textbf{acoustique}, déplacement en phase, dynamique dominée par interactions entre cellules) et supérieure (\textbf{optique}, les ions d’une même cellule vibrent l’un par rapport à l’autre,).
    \item Dans le cas \textbf{à 3 dimensions:} pour une cellule primitive avec une base de $p$ atomes, $3N$ modes acoustiques et $(3p - 3)N$ modes optiques de vibrations.
    
    \squishline

    \textbf{2 masses, même ressort}
    \item $\omega^2(k) = K \left(\frac{1}{M_1}+\frac{1}{M_2}\right) \pm K \sqrt{\left(\frac{1}{M_1}+\frac{1}{M_2}\right) - \frac{4}{M_1 M_2} \sin^2(ka/2)}$
    \item Au bord de 1ZDB, si $M_1 > M_2 \quad \Longrightarrow \quad \omega_{\text{opt}}= \sqrt{\frac{2K}{M_2}} \quad \omega_{\text{acous}} = \sqrt{\frac{2K}{M_1}}$ \\
    La largeur de la bande interdite (gap) entre branches acoustiques et optiques se réduit lorsque les deux masses d’un système sont similaires.
\end{squishlist}

\graypar{La notion de phonons}
\begin{squishlist}
    \item La contribution à l'énergie totale d'un mode normal de fréquence $\omega_s(\vk_{\nu})$ est $\left( n_{\vk_{\nu,s}} + \frac{1}{2}\right) \hbar \omega_s(\vk_{\nu})$ \qquad $n_{\vk_{\nu,s}} = 0,1,2,\ldots$ \textbf{nb d'occupation} du mode normal $\nu,sd$
    \item On passe à une déscriptions corpusculaire: nombre $n_{\vk_{\nu,s}}$ de \textbf{phonons} de vecteur d'onde $\vk_{\nu}$ et polar.\ $s$ dans le crystal.
    \item Les phonons d'un réseau ne portent pas de quantité de mouvement
    \item Lors de l’interaction entre une particule incidente dans le cristal et les phonons, tout se passe comme si une pseudo quantité de mouvement, $\hbar \vk$ (crystal momentum), était conservée à un vecteur $\hbar \vG$ du réseau réciproque près. \\
    $\vec{p} + \sum_{\nu,s}\hbar \vk_{\nu}n_{\vk_{\nu,s}} = \vec{p}' + \sum_{\nu,s}\hbar \vk_{\nu}n'_{\vk_{\nu,s}} + \hbar \vG$ \quad plus cons.\ de l'énergie

    \item \textbf{Diffusion à un phonon:} $\frac{p'^2}{2m} = \frac{p^2}{2m} + \hbar \omega_s \left( \frac{\vec{p'}-\vec{p}}{\hbar}\right)$
    
\end{squishlist}

\graysec{Propriétés thermiques en relation avec les phonons}
\begin{squishlist}
    \item $c_v = \left.\D{T}{u}\right|_V$ chaleur spécifique du réseau
    \item \textbf{Loi de Dulong et Petit:} $c_v = 3n k_B$ ($n$ densité d'atomes). En accord avec les mesures pour une bonne partie des solides et proches de la température ambiante.
    \item $u = \frac{1}{V} \sum_{\vk,s} \hbar \omega_s(\vk) \left( \braket{n_{\vk,s}}+\frac{1}{2}\right)$ \quad $\braket{n_{\vk,s}} = \frac{1}{\exp[\beta \hbar \omega_s(\vk)]-1}$ (Bose-Einstein) \\
    À haute $T$, si $x = \frac{\hbar \omega_s}{k_B T}$, $\braket{n_{\vk,s}} \approx \frac{1}{x} - \frac{1}{2} + \frac{x}{12} \approx \frac{k_BT}{\hbar \omega_s} - \frac{1}{2}$
    
    \item \textbf{Expression génerale:} $c_v = \D{T}{} \sum_s {\displaystyle\int_{\text{1ZDB}}} \frac{\ud^3 k}{(2\pi)^3} \frac{\hbar \omega_s(\vk)}{\exp[\hbar \omega_s(\vk)/k_B T]-1}$
    
    \squishline

    \textbf{Limites }
    \item \textbf{Haute temperature:} Pour $\hbar \omega_s(\vk)/k_BT \ll 1$ on retrouve D\&P
    \item \textbf{Basse temperature:} À basse $T$ les modes tels que $\hbar \omega_s(\vk) \gg k_BT$ donnent une contribution négligeable (e.g. les modes optiques). Par contre modes acoustiques non-négligeables car $\omega_s(\vk) \rightarrow 0$ lorsque $k \rightarrow 0$ $\Longrightarrow$ $c_v = \frac{2\pi^2}{5}k_B \left(\frac{k_B T}{\hbar c}\right)^3 \propto T^3$
    
    \squishline

    \textbf{Entre les domaines de hautes et basses température}

    \item \textbf{Modèle d'Einstein:} on fait l'hypothèse que la rélation de dispersion $\omega_s(\vk)$ est const \\
    $u = 3 \frac{N}{V} \braket{n_E}\hbar \omega_E = 3n \frac{\hbar \omega_E}{\exp[\hbar \omega_E/k_BT]-1} \Longrightarrow c_v = 3nk_B \left(\frac{\hbar \omega_E}{k_BT}\right)^2 \frac{\exp[\hbar \omega_E /k_BT]}{(\exp[\hbar \omega_E /k_BT] - 1)^2}$ \\
    Pas de modes qui peuv.\ être excités à basse $T$ $\rightarrow$ $c_V$ décroit trop vite en descend.\ $T$

    \item \textbf{Modèle de Debye:} (1) on remplace toutes les branches du spectre de vibration par 3 branches ayant la même relation de dispersion $\omega = c|k|$ ($\omega \propto \sqrt{C/m}|k|$) \\
    (2) on remplace l'intégrale sur 1ZDB par une int.\ sur une sphère de rayon $k_D$, de telle sorte que la sphère contienne le même nombre de vecteurs d'onde: \\
    $\left( \Omega_{\text{Debye}} \right) \cdot \text{densité de }k = \# \text{de } k \text{ dans 1ZDB}\quad \text{densité}=\frac{\text{vol.\ du cristal}}{\text{volume d'un }k}$ e.g.\ $\frac{N a^2}{(2\pi)^2}$ en 2D \\
    $\Longrightarrow$ en 3D $n = \frac{N}{V} = \frac{k_D^3}{6\pi^2}$
    \item En définissant $\omega_D = ck_D$ et $k_B \theta_D = \hbar \omega_D$ ($T > \theta_D$: tous les modes commencent à être excités) on a
    $c_v = 9 n k_B \left(\frac{T}{\theta_D}\right)^3 {\displaystyle \int_0^{\theta_D/T}} \frac{x^4 e^x}{(e^x - 1)^2} \ud x$ où $x = \beta \hbar c k$
    \item \textbf{Basse temperature}($T/\theta_D \ll 1$): $c_v = \frac{12\pi^4}{5} n k_B \left( \frac{T}{\theta_D}\right)^3 \propto T^3$
    \item \textbf{Haute temperature:}($T/\theta_D \gg 1$) on retrouve D\&P
\end{squishlist}

\graypar{Densité de modes en $\omega$}
Combien de $k$ ($\Delta k$) pour un $\omega$ ($\Delta \omega$) donné? Beaucoup de $k$: haute densité
\begin{squishlist}
    \item $g(\omega) = \sum_s {\displaystyle \int_{\omega_s(\vk) = \omega}} \frac{\ud^3 k}{(2\pi)^3}\delta[\omega - \omega_s(\vk)] = \frac{1}{(2\pi)^3} \sum_s {\displaystyle \int_{\omega_s(\vk) = \omega}} \frac{\ud S_{\omega}}{|\nabla \omega_s(\vk)|}$ en 3D
    \item $u = \sum_s \int_{\text{1ZDB}} \frac{\ud^3 \vk}{(2\pi)^3} \hbar \omega_s(\vk) \left(\braket{n_{\vk, s}} + \frac{1}{2}\right)$
    \item Chaine 1D, Haute $T$: $c_{V, \text{Debye}} = c_{V,\text{exac}} = n K_B = K_B / a$ \\
    Basse $T$: $c_{V, \text{Debye}} = \frac{\pi^2}{3}nK_B \frac{T}{\theta_D} \quad c_{V,\text{exac}} = \frac{2\pi}{3} n k_B \frac{k_B}{\hbar \omega_{\max}} T + \frac{4 \pi^3}{15} nk_B \left(\frac{k_B}{\hbar \omega_{\max}}\right)^3 T^3$
\end{squishlist}

\graypar{Conductibilité thermique du réseau}
\begin{squishlist}
    \item \textbf{Cristal parfaitement harmonique:} les modes normaux sont états stationnaires de l'hamiltonien du cristal. Si une distribution de phonons se crée, elle ne sera pas modifiée au cours de temps. Conductibilité thermique infinie.
    \item La conductibilité thermique du réseau n'est pas infinie pour les raisons
    (1) impuretés brisent la symetrie de translation (2) la surface de l'échantillon est non-negligeable si sa dimension est de l'ordre du libre parcours moyen des phonons (3) les états stationnaires de l’hamiltonien harmonique ne sont pas états stationnaires de l’hamiltonien réel du cristal qui contient des termes anharmoniques.
    \item \textbf{Théorie cinétique:} si $\omega = ck$, $\lambda = c \tau$ libre parcours moyens ($\tau \sim \frac{1}{T}$) \\ $j_Q = \frac{1}{2}c (u[T(x-\lambda)] - u[T(x+\lambda)]) = - c^2 \tau \D{T}{u}\D{x}{T}$ \\
    En 3D $\vec{j}_Q = - \frac{1}{3}c^2 c_v(T) \tau \vec{\nabla T} \Longleftrightarrow \kappa = \frac{1}{3} c^2 \tau c_v(T)$
    \item \textbf{Processus normal} : $\sum n_{\vk,s}\vk = \sum n'_{\vk, s}\vk$ (conservation des moments) \\
    \textbf{Processus Umklapp}:  $\sum n_{\vk,s}\vk = \sum n'_{\vk, s}\vk + \vG$ (conserv.\ à un vec.\ du rés.\ recip.\ près) \\
    Umklapp peu probable à basse $T$. Seul le faible nb des phon.\ participant à  processus Umklapp contribuent à la relaxation $\tau_{\text{Umk}} \sim \exp(T_0/T)$ donc réduit la conductibilité thermique (la quantité de mouvement est alterée, direction de propagation changée).
\end{squishlist}

\columnbreak

\graysec{Gaz d'électrons libres de Fermi}
Hypoth.\: les \elec se déplacent dans un \textbf{potentiel effectif constant}. Bonne approx.\ pour beauc.\ de métaux, mais ne peut pas expliquer pourquoi un solide est un métal, un sémiconducteur, ou un isolant.
\begin{squishlist}
    \item Conditions de Born-von-Karman: $\psi(\vec{r} + N_j \vec{a}_j) = \psi(\vec{r}), \, \forall j=1,2,3$ où $\vec{a}_j$ est un vecteur primitif du réseau de Bravais, $N_j$ un entier positif. \\
    $\Rightarrow$Les comp.\ de $\vk$ sont multipl.\ de $\frac{2\pi}{L} \Longrightarrow$ dans éspace dim.\ $N$, volume de $k$ $=\left(\frac{2\pi}{L}\right)^N$
    \item $\psi_{\vk}(\vr) = \frac{1}{\sqrt{V}}\exp(i \vk \cdot \vr) \quad E(\vk) = \frac{\hbar^2 k^2}{2m} \quad k = \left(\frac{2m}{\hbar^2}\right)^{1/2} \sqrt{E} \quad \vec{p}=\hbar \vec{k}$
    \item $N$ valeurs $\vec{k}$ admises dans chaque zone de Brillouin $\rightarrow$ $2N$ états électroniques (spin). Toutes les valeurs $\vk$ t.q.\ $k\leq k_F$ sont occupées, celles t.q.\ $k > k_F$ sont vides.
    \item Trouver $k_F$: vol. sphère de Fermi $\Omega_F$ $\times$ densité de $k$ $\times \; 2$ (spin) = \# $e^-$ dans système 
    \item \textbf{Densité électronique} $n = \frac{N}{V} = \int_0^{E_F} g(E) \ud E$ \\
    E.g.\ 3D: $2 \left( \frac{4}{3}\pi k^3_F\right) \left( \frac{V}{8\pi^3}\right) = N \quad \frac{N}{V} = n = \frac{k_F^3}{3\pi^2} \qquad E_F = \frac{\hbar^2}{2m}(3\pi^2 n)^{2/3}$

    % \item $E_F = \frac{\hbar^2 k_F^2}{2m}$
    \item\textbf{Densité d'états} $g(E)$: \# d'états par unité de $V$ avec énergie entre $E$ et $E+\ud E$\\
    $g(E) \ud E = 2\; \frac{\ud \Omega}{\text{volume $k$}} \; \frac{1}{V}$ \\ 
    $g(E)_{\text{1D}} =\frac{\sqrt{2m}}{\pi \hbar}\frac{1}{\sqrt{E}}$ \quad
    $g(E)_{\text{2D}} = \frac{m}{\pi \hbar^2}$ \quad
    $g(E)_{\text{3D}} = \frac{1}{2\pi} \left(\frac{2m}{\hbar^2}\right)^{3/2}\sqrt{E}$
    % \squishsep $u(0 $K$) = \int_{0}^{E_F}E \; g(E)\; \ud E$ 
    \item \textbf{Fermi-Dirac}: $f(E,T) = \frac{1}{\exp\left(\frac{(E- \mu)}{k_B T}\right) + 1}$ \squishsep $u = \int_{0}^{\infty}E \; g(E)\; f(E) \; \ud E$
    \item \textbf{Developp.\ de Sommerfeld:} $\int_{0}^{\infty}h)E) f(E) \ud E = \int_{0}^{\mu} h(E) \ud E + \frac{\pi^2}{6}(k_BT)^2 \left.\D{E}{h}\right|_{E=\mu} + \ldots$
    \item $\mu(T) = E_F - \frac{\pi^2}{6}(k_B T)^2 \frac{1}{g(E_F)} \left.\D{E}{g}\right|_{E_F}$ \squishsep $c_V = \frac{\pi^2}{3}K_B^2 T \, g(E_F) $ (contrib.\ \elec)
    \item Champ magn.\ $\Rightarrow g_{\pm}(E) = \frac{1}{2}g(E\mp \mu_B B)$\quad $\chi_{\text{Pauli}} = \mu_0 \mu_B^2 g(E_F)$ ($\mu$ inchangé)
    \item $\vec{j} = -ne \braket{\vec{v}} = \sigma \vec{E}$ \quad En présence de $\vec{E}$ la sphère de Fermi est déplacé de $\braket{\vk}=-e\vec{E}\tau / \hbar$ \\
    $\Longrightarrow$ Conduct.\ électrique $\sigma = \frac{n e^2 \tau(E_F)}{m}$ diminue avec $T$ (parce que $\tau$ diminue avec $T$)
    \item Conductivité thermique $\kappa = \frac{1}{3}V_F^2 \tau(E_F) c_V$ $\Rightarrow$ \textbf{Wiedemann-Franz}: $\frac{\kappa}{\sigma T} = \frac{\pi^2}{3}\left(\frac{k_B}{e}\right)$
\end{squishlist}


\graysec{Les électrons dans un potentiel périodique. Structure de bande}
\graypar{Théorème de Bloch}
\begin{squishlist}
    \item $\psi(\vec{r} + \vec{R}) = \exp(i\vec{k} \cdot \vec{R}) \psi(\vec{r})$
    \item Un vecteur $\vec{k}$ ne se trouvant pas dans la 1ZDB peut être réduit à la 1ZDB: $\tilde{\vec{k}} = \vec{k} + \vec{G}$ \\
    car $\vec{k}$ est défini à un vecteur du réseau réciproque près, i.e. $\exp(i \vec{G}\cdot \vec{R}) = 1$.
    \item Alternative: $\psi_{n \vec{k}}(\vr) = \exp(i \vk \cdot \vr ) u_{n \vk}(\vr)$ où $u_{n \vk}(\vr) =  u_{n \vk}(\vr+ \vR)$
\end{squishlist}

\graypar{Équation centrale}
\begin{squishlist}
    \item On peut décomposer une fonction d'onde $\psi(\vr)$ qui satisfait Born-von-Karman comme $\psi(\vr) = \sum_{\vk} a_{\vk} \exp(i \vk \cdot \vr)$
    \squishsep $\psi_{n,\tilde{\vk}} = \sum_{\vG} a_{n, \tilde{\vk}-\vG} \exp(i (\tilde{\vk} - \vG) \cdot \vr) \quad n=1\rightarrow\infty$
    \item Le potentiel a la périodicité du réseau: $U(\vr) = \sum_{\vG} U_{\vG} \exp(i \vG \cdot \vr)$
    \item $\left(\frac{\hbar^2 k^2}{2m} - E\right)a_{\vk} + \sum_{\vG}U_{\vG}a_{\vk - \vG}  = 0$
    \item $(E_{\vk}^0 - E)a_{\mathbf{{\tilde{k}}} - \vG} + \sum_{\vG'}U_{\vG' - \vG}a_{\mathbf{\tilde{k}} - \vG'}  = 0$
    \item Periodicité de l'énergie $ E_{n,\mathbf{\tilde{k}} + \vG} = E_{n,\mathbf{\tilde{k}}}$
    \item Électrons libres: $U(\vr) = 0 \Longrightarrow E_{k - G_i}^0 = \frac{\hbar^2(k - G_i)^2}{2m}$
\end{squishlist}

\graypar{Modèle des électrons faiblement couplés au réseau (quasi-libres)}
Part de l'approximation du gaz d'électrons libres et suppose une corrugation faible du potentiel, due au fait que: 
1) les interactions sont fortes surtout à courtes distances et le principe d'exclusion de Pauli empêche les électrons de conduction de pénétrer dans le coeur et 
2) les électrons de coeur écrantent le noyau positif. Ce modèle s'applique en général bien aux électrons de valence de type s - p, donc pour les alcalins (groupe I), alcalins terreux (groupe II), métaux nobles.
\begin{squishlist}
    \item États presque dégénérés: $(E_{k-G_i}^0 - E) c_{k-G_i} + \sum_{j=1}^m U_{G_j - G_i} c_{k-G_j} = 0$ \\
    $i,j = 1 \ldots m$ tels que $|E_{k-G_i}^0 - E_{k-G_j}^0| \lesssim U$, \quad $U_G$ coefficients de Fourier de $U$. \\
    Peut être mis sous forme matricielle, trouver energies propres et vecteurs propres \\
    Levée de la dégénérescence $\rightarrow$ bandes d'énergies interdites.
    L’apparition d’un ”gap” d’énergie est donc reliée à la réflexion des fonctions d’onde électroniques (ondes planes) en bord de zone, qui crée des ondes
    stationnaires de densité de probabilité en $\cos^2$ ou $\sin^2$.
\end{squishlist}

\graypar{Modèle des liaisons fortes}
Part d'atomes neutres que l'on approche de plus en plus.
Les orbitales les plus étendues spatialement vont alors se recouvrir. Dans l'approximation
des liaisons fortes, on suppose que la fonction d'onde est bien décrite par les orbitales
atomiques, et on pose $H = H_{\text{at}} + \Delta U$. 
Cette approximation s'applique particulièrement
bien aux métaux de transition et aux isolants, qui ont un recouvrement d'orbitales pas
trop important.
\begin{squishlist}
    \item $E(\vk) = E + \frac{\beta + \sum_{\vR \neq 0}\exp(i\vk \cdot \vR) \; \gamma(\vR)}{1 + \sum_{\vR \neq 0}\exp(i\vk \cdot \vR)\; \alpha(\vR)}$
    $\left\{ \begin{aligned}
        & \alpha(\vR) = \int \phi^*(\vr)\phi(\vr - \vR) \ud^3 r \\
        & \beta = \int \phi^*(\vr)\Delta U (\vr) \phi(\vr) \ud^3 r \\
        & \gamma(\vR) = \int \phi^*(\vr)\Delta U (\vr) \phi(\vr - \vR) \ud^3 r
    \end{aligned} \right.$
    \item $\alpha(\vr)$ est positive pour fonction d'onde de type $s$, typiquement entre 0.1 et 0.4
    \item L'intégrale de champ cristallin $\beta < 0$ décrit l'effet du potentiel généré par les autres atomes.
    \item L'intégrale de transfert $\gamma(R)$ décrit le passage d'un électron du site initial $R = 0$ au site $R$, induit par la présence du potentiel $\Delta U$.
\end{squishlist}

\graypar{Métaux, isolants, semiconducteurs}
$2N$ fonctions de Bloch indépendantes dans chaque zone de Brillouin.
À chaque zone de Brillouin on peut associer une bande d'énergie qui peut donc "contenir" $2N$ électrons.
S’il y a un seul électron de valence par cellule primitive, la bande d’énergie
est à moitié remplie par les électrons.
\begin{squishlist}
    \item  nb $e^-$ par maille \textbf{impair}: forcement \textbf{métal}
    \item  nb $e^-$ par maille \textbf{pair}: \textbf{métal} si les bandes d'énergie se recouvrent (deux bandes partiellement remplies); \textbf{isolants} si elles ne se recouvrent pas (électrons remplissent enti`erement une bande d’énergie, les autres bandes étant vides) ou \textbf{semiconducteurs} si la largeur de la bande interdite est faible ($\sim 1 $ eV) (il est possible de faire passer par excitation thermique des électrons de la bande pleine à la vide, conductibilité électrique qui croît lorsque la température augmente).
    \item Metaux de transition ont plus grande $g(E_F)$ $E_F$ est sur bande $d$ qui a densité d'état plus élevée.
\end{squishlist}


\graysec{Dynamique des électrons}
\begin{squishlist}
    \item Équations du mouvement entre collisions, $n$ constante du mouvem.
    $ \left\{ \begin{aligned} 
        & \dot{\vec{r}} = \vec{v_n}(\vk) = \frac{1}{\hbar} \vec{\nabla_\vec{k}E_n}(\vk) \\
        & \hbar \dot{\vec{k}} =  -e (\vec{E} + \vec{v_n}(\vk) \times \vec{B}) =  -e (\vec{E} + \frac{1}{\hbar} \vec{\nabla_\vec{k}E_n}(\vk) \times \vec{B})
    \end{aligned} \right. $
    
    \item Pour un $e^-$ de Bloch (potentiel non-nul) $\hbar \vk \neq \vec{p}$
    \item Une bande pleine de conduit pas: \\
    $\vec{j} = -e \int_{\text{zone Brillouin}} \frac{\ud^3 k}{4 \pi^3}\vec{v} = -e \frac{1}{\hbar} \int_{\text{zone Brillouin}} \frac{\ud^3 k}{4 \pi^3} \D{\vec{k}}{E} = 0$ \\
    car pour toute fonction périodique $f(\vk) = f(\vk + \vG)$ on a $\int_{\text{cell.\ prim.}}\ud^3 k \D{\vk}{f(\vk)} = 0$
    \item Sous l’effet du champ $E$ le remplissage
    des états par les électrons n’est pas modifié, car deux états électroniques de la même bande qui diffèrent d’un vecteur du réseau réciproque doivent être considérés comme le même état.
    \item Concept de \textbf{trou}: $\vec{j} = -e \int_{\text{états occupés}} \frac{\ud^3 k}{4 \pi^3}\vec{v} = +e \int_{\text{états vides}} \frac{\ud^3 k}{4 \pi^3}\vec{v}$
    \item Niveau électron.\ \textbf{proche du sommet d'une bande}: $E(\vk) = E(\vk_0) - A(\vk - \vk_0)^2$ \\
    On définit la masse effective $m^*$ par $A = \frac{\hbar^2}{2m^*}$.
    Un trou se comporte au voisinage du sommet d’une bande
    comme une particule de charge $+e$ et de masse $m^*$. Trous lourds ont paraboles larges, trous légers ont paraboles étroites.

    \item Dans l’espace réciproque les trajectoires électroniques sont situées à l’intersection d’une surface d’énergie constante et d’un plan perpendiculaire à $B$.
    \begin{itemize}
        \item $\nabla E(\vk)$ pointe vers l'éxter.\ de la surf.\ $E(\vk) = $ cte $\Rightarrow$ trajectoire électronique
        \item $\nabla E(\vk)$ pointe vers l'inter.\ de la surf.\ $E(\vk) = $ cte $\Rightarrow$ trajectoire de type trou
    \end{itemize}
\end{squishlist}

\graysec{Les semiconducteurs}
Un semiconducteur est intrinsèque si ses proprietés électroniques sont dominées par les $e^-$ excités thermiquement de la bande de valence à la bande de conduction et extrinsèque si elles sont dominées par les $e^-$ contribués à la bande de conduction ou capturé par la bande de valence par des impuretés.
\begin{squishlist}
    \item Densité d'$e^-$ dans bande de conduction $n(T) = \int_{E_c}^{\infty} g_c(E) \frac{1}{e^{(E-\mu)/k_B T} + 1} \ud E$
    \item Densité de trous dans bande de valence $p(T) =\int_{E_c}^{\infty} g_v(E)\left( 1- \frac{1}{e^{(E-\mu)/k_B T} + 1} \right)\ud E$
    \item Pour un sc non dégénéré ($E_c - \mu \gg k_B T, \mu -E_v \gg k_B T)$ on a \\ $n(T) = N(T) e^{-(E_c - \mu) / k_B T}$ \quad $p(T) = P(T) e^{-(\mu - E_v) / k_B T}$
    \item Bande quadratique: $g_{c,v}(E) = \frac{1}{2 \pi^2}\left( \frac{2m^*_{c,v}}{\hbar^2}\right) \sqrt{E - E_{c,v}}$
    d'où \\
    $N(T) = \frac{1}{4} \left( \frac{2 m_c^* k_B T}{\pi \hbar^2}\right)^{3/2}=2.5 \left(\frac{m_c^*}{m}\right)^{3/2} \left(\frac{T}{300 \mathrm{K}}\right)^{3/2} 10^{19}/\mathrm{cm}^3; \quad P(T) = \frac{1}{4} \left( \frac{2 m_v^* k_B T}{\pi \hbar^2}\right)^{3/2}$
    \item \textbf{Loi d'action de masse:} $n(T)p(T) = N(T)P(T) e^{-\beta E_g}$
    \item sc \textbf{intrinsèque} $n(T) = p(T) = n_i(T)$ \quad densité d'$e^-$ excités thermiquement \\
    $\Longrightarrow n_i(T) = \sqrt{n(T) p(T)} = e^{-E_g / 2 k_B T} \sqrt{N(T) P(T)}$ \\
    $\Longrightarrow \mu_i(T) = E_v + \frac{1}{2}E_g + \frac{1}{2}k_B T \ln \left( \frac{P(T)}{N(T)}\right) = E_v + \frac{1}{2}E_g + \frac{3}{2}\frac{1}{2}k_B T \ln \left( \frac{m_v^*}{m_c^*}\right)$
\end{squishlist}

\graypar{Semiconducteur dopé}
Ajout d’impuretés à un SC.
L’impureté doit avoir un nombre d’électrons de valence différent de celui du SC pur. Si la valence de l’impureté est supérieure à celle du SC, nous sommes en présence d’un \textbf{donneur} (l’impureté “donne” un électron au semi-conducteur), tandis que si la valence est inférieure, nous sommes en présence d’un \textbf{accepteur} (l’impureté "prend" un électron au SC, autrement dit, lui "donne" un trou)
\begin{squishlist}
    \item Impuretés modèlisées avec atome d'H avec corrections: masse effective $m^*$ et constante diélectrique $\varepsilon_r$ du SC 
    $\Longrightarrow r_0 = \frac{\varepsilon_r}{m^*/m} a_0; \quad E=\frac{m^*/m}{\varepsilon_r^2}E_H$
    \item $n = \frac{1}{2}\sqrt{\Delta n^2 + 4 n_i^2} + \frac{1}{2}\Delta n;\quad \Delta n = n - p$
    \item $\frac{\Delta n }{n_i} = \frac{n-p}{\sqrt{np}} = 2 \sinh (\beta(\mu - \mu_i))$
    \item $\sigma = n e \nu_e + p e \nu_h$
\end{squishlist}    

\graypar{Jonction pn}
En polaris.\ directe le courant est dû aux porteurs majoritaires (\elec qui provienn,\ de la region $n$ et trous de la region $p$) par diffusion et recombinaison.
En polaris.\ inverse le courant est dû aux porteurs minoritaires générés thermiquement, favoris par champ électrique.

\graysec{Les supraconducteurs}
\begin{squishlist}
    \item \textbf{Effet Meissner}: si $T<T_c$ le flux magnétique est éjécté
    \item Au dessus d'un champ critique $H_c$ l'état supra est détruit
    \item Type II, pour $H_{c1}<H<H_{c2}$ la densité de flux à l'intérieur est non nulle (effet Meissner incomplet)
    \item \textbf{Eq.\ de London}: $\vec{j} = - \frac{1}{\mu_0 \lambda_L^2} \vec{A} \Longrightarrow \vec{\nabla}^2 \vec{B} = \frac{1}{\lambda_L^2}\vec{B}$ qui a seule solut.\ $B=0$
    \item Prédictions de la théorie BCS
\end{squishlist}

\graysec{Mixed}
\begin{squishlist}
    \item $k_B T (300$K$) \approx 26$ meV
\end{squishlist}