% Matteo Veneziano -- Solid State Physics Formula Sheet
\begin{squishlist}
    \item $\vec{R} = \sum_{j=1}^{3} n_j \vec{a}_j$ un vecteur du réseau de Bravais
    \item Conditions de Born-von-Karman: $\psi(\vec{r} + N_j \vec{a}_j) = \psi(\vec{r}), \, \forall j=1,2,3$ où $\vec{a}_j$ est un vecteur primitif du réseau de Bravais, $N_j$ un entier positif.
    \item $N$ valeurs $\vec{k}$ admises dans chaque zone de Brillouin $\rightarrow$ $2N$ états électroniques (spin)
    \item Densité électronique $n = \frac{N}{V} = \int_0^{E_F} g(E) \ud E$
\end{squishlist}

\graypar{Surface de Fermi}
\begin{squishlist}
    \item Trouver $k_F$: vol. sphère de Fermi $\Omega_F$ $\times$ densité de $k$ $\times \; 2$ (spin) = \# $e^-$ dans système 
    \item $E_F = \frac{\hbar^2 k_F^2}{2m}$
\end{squishlist}

\graysec{Électrons libres}
\begin{squishlist}
    \item Densité d'états $g(E)$ HOW IS IT FOUND
    \item Distribution de Fermi-Dirac $f(E,T) = \frac{1}{\exp\left(\frac{(E- \mu)}{k_B T}\right) + 1}$
    \item Potentiel chimique $\mu(T) = E_F - \frac{\pi^2}{6}(k_B T)^2 \frac{1}{g(E_F)} \left.\dder{g}{E}\right|_{E_F}$
    \item Chaleur specifique $c_V = \frac{\pi^2}{3}K_B^2 T \, g(E_F) $
    \item $\chi_{\text{Pauli}}$ FILL
    \item $\tau$ FILL
    \item Conductivité électrique $\sigma = \frac{n e^2 \tau(E_F)}{m}$
    \item Conductivité thermique $\kappa = \frac{1}{3}V_F^2 \tau(E_F) c_V$
    \item Loi de Wiedemann-Franz: $\frac{\kappa}{\sigma T} = \frac{\pi^2}{3}\left(\frac{k_B}{e}\right)$
\end{squishlist}

\graysec{Les électrons dans un potentiel périodique. Structure de bande}
\graypar{Théorème de Bloch}
\begin{squishlist}
    \item $\psi(\vec{r} + \vec{R}) = \exp(i\vec{k} \cdot \vec{R}) \psi(\vec{r})$
    \item Un vecteur $\vec{k}$ ne se trouvant pas dans la 1ZDB peut être réduit à la 1ZDB: $\tilde{\vec{k}} = \vec{k} + \vec{G}$ \\
    car $\vec{k}$ est défini à un vecteur du réseau réciproque près, i.e. $\exp(i \vec{G}\cdot \vec{R}) = 1$.
    \item Alternative: $\psi_{n \vec{k}}(\vr) = \exp(i \vk \cdot \vr ) u_{n \vk}(\vr)$ où $u_{n \vk}(\vr) =  u_{n \vk}(\vr+ \vR)$
\end{squishlist}

\graypar{Équation centrale}
\begin{squishlist}
    \item On peut décomposer une fonction d'onde $\psi(\vr)$ qui satisfait Born-von-Karman comme $\psi(\vr) = \sum_{\vk} a_{\vk} \exp(i \vk \cdot \vr)$
    \item Le potentiel a la périodicité du réseau: $U(\vr) = \sum_{\vG} U_{\vG} \exp(i \vG \cdot \vr)$
    \item $\left(\frac{\hbar^2 k^2}{2m} - E\right)a_{\vk} + \sum_{\vG}U_{\vG}a_{\vk - \vG}  = 0$
    \item $(E_{\vk}^0 - E)a_{\mathbf{{\tilde{k}}} - \vG} + \sum_{\vG'}U_{\vG' - \vG}a_{\mathbf{\tilde{k}} - \vG'}  = 0$
    \item Periodicité de l'énergie $ E_{n,\mathbf{\tilde{k}} + \vG} = E_{n,\mathbf{\tilde{k}}}$
    \item Électrons libres: $U(\vr) = 0$
\end{squishlist}

\graypar{Modèle des électrons faiblement couplés au réseau (quasi-libres)}
Part de l'approximation du gaz d'électrons libres et suppose une corrugation faible du potentiel, due au fait que: 
1) les interactions sont fortes surtout à courtes distances et le principe d'exclusion de Pauli empêche les électrons de conduction de pénétrer dans le coeur et 
2) les électrons de coeur écrantent le noyau positif. Ce modèle s'applique en général bien aux électrons de valence de type s - p, donc pour les alcalins (groupe I), alcalins terreux (groupe II), métaux nobles.
\begin{squishlist}
    \item 
\end{squishlist}

\graypar{Modèle des liaisons fortes}
Part d'atomes neutres que l'on approche de plus en plus.
Les orbitales les plus étendues spatialement vont alors se recouvrir. Dans l'approximation
des liaisons fortes, on suppose que la fonction d'onde est bien décrite par les orbitales
atomiques, et on pose $H = H_{\text{at}} + \Delta U$. 
Cette approximation s'applique particulièrement
bien aux métaux de transition et aux isolants, qui ont un recouvrement d'orbitales pas
trop important.
\begin{squishlist}
    \item L'intégrale de champ cristallin $\beta$ décrit l'effet du potentiel généré par les autres atomes.
    \item L'intégrale de transfert $\gamma(R)$ décrit le passage d'un électron du site initial $R = 0$ au site $R$, induit par la présence du potentiel $\Delta U$.
\end{squishlist}

\graysec{Dynamique des électrons}
\begin{squishlist}
    \item Équations du mouvement entre collisions, $n$ constante du mouvem.
    $ \left\{ \begin{aligned} 
        & \dot{\vec{r}} = \vec{v_n}(\vk) = \frac{1}{\hbar} \vec{\nabla_\vec{k}E_n}(\vk) \\
        & \hbar \dot{\vec{k}} =  -e (\vec{E} + \vec{v_n}(\vk) \times \vec{B}) =  -e (\vec{E} + \frac{1}{\hbar} \vec{\nabla_\vec{k}E_n}(\vk) \times \vec{B})
    \end{aligned} \right. $
    
    \item Pour un $e^-$ de Bloch (potentiel non-nul) $\hbar \vk \neq \vec{p}$
    \item Une bande pleine de conduit pas: \\
    $\vec{j} = -e \int_{\text{zone Brillouin}} \frac{\ud^3 k}{4 \pi^3}\vec{v} = -e \frac{1}{\hbar} \int_{\text{zone Brillouin}} \frac{\ud^3 k}{4 \pi^3} \D{\vec{k}}{E} = 0$ \\
    car pour toute fonction périodique $f(\vk) = f(\vk + \vG)$ on a $\int_{\text{cell.\ prim.}}\ud^3 k \D{\vk}{f(\vk)} = 0$
    \item Sous l’effet du champ $E$ le remplissage
    des états par les électrons n’est pas modifié, car deux états électroniques de la même bande qui diffèrent d’un vecteur du réseau réciproque doivent être considérés comme le même état.
    \item Concept de \textbf{trou}: $\vec{j} = -e \int_{\text{états occupés}} \frac{\ud^3 k}{4 \pi^3}\vec{v} = +e \int_{\text{états vides}} \frac{\ud^3 k}{4 \pi^3}\vec{v}$
    \item Niveau électron.\ \textbf{proche du sommet d'une bande}: $E(\vk) = E(\vk_0) - A(\vk - \vk_0)^2$ \\
    On définit la masse effective $m^*$ par $A = \frac{\hbar^2}{2m^*}$.
    Un trou se comporte au voisinage du sommet d’une bande
    comme une particule de charge $+e$ et de masse $m^*$

    \item Dans l’espace réciproque les trajectoires électroniques sont situées à l’intersection d’une surface d’énergie constante et d’un plan perpendiculaire à $B$.
    \begin{itemize}
        \item $\nabla E(\vk)$ pointe vers l'éxter.\ de la surf.\ $E(\vk) = $ cte $\Rightarrow$ trajectoire électronique
        \item $\nabla E(\vk)$ pointe vers l'inter.\ de la surf.\ $E(\vk) = $ cte $\Rightarrow$ trajectoire de type trou
    \end{itemize}
\end{squishlist}

\graysec{Les semiconducteurs}